% Template:     Informe/Reporte LaTeX
% Versión:      2.3.5 (28/02/2017)
% Codificación: UTF-8
%
% Autor: Pablo Pizarro R.
%        Facultad de Ciencias Físicas y Matemáticas.
%        Universidad de Chile.
%        pablo.pizarro@ing.uchile.cl, ppizarror.com
%
% Sitio web del proyecto:   [http://ppizarror.com/Template-Informe/]
% Licencia: CC BY-NC-SA 4.0 [http://creativecommons.org/licenses/by-nc-sa/4.0/]

% CREACIÓN DEL DOCUMENTO, FUENTE E IDIOMA
\documentclass[letterpaper,11pt]{article} % Articulo tamaño carta, fuente de 11pt
\usepackage[utf8]{inputenc}               % Codificación UTF-8
\usepackage[T1]{fontenc}                  % Soporta caracteres acentuados
\usepackage{lmodern}                      % Tipografía moderna
\usepackage[spanish]{babel}               % Define el idioma del documento en español


% INFORMACIÓN DEL DOCUMENTO
\newcommand{\nombredelinforme}{Título del informe}
\newcommand{\temaatratar}{Tema a tratar}
\newcommand{\fecharealizacion}{\today}
\newcommand{\fechaentrega}{\today}

\newcommand{\autordeldocumento}{Nombre del autor o grupo}
\newcommand{\nombredelcurso}{Curso}
\newcommand{\codigodelcurso}{CO-1234}

\newcommand{\nombreuniversidad}{Universidad de Chile}
\newcommand{\nombrefacultad}{Facultad de Ciencias Físicas y Matemáticas}
\newcommand{\departamentouniversidad}{Departamento de la Universidad}
\newcommand{\imagendeldepartamento}{images/departamentos/fcfm}
\newcommand{\imagendeldepartamentoescl}{0.2}
\newcommand{\localizacionuniversidad}{Santiago, Chile}


% INTEGRANTES, PROFESORES Y FECHAS
\newcommand{\tablaintegrantes}{
\begin{minipage}{1.0\textwidth}
	\begin{flushright}
		\begin{tabular}{ll}
		Integrantes: 
			& \begin{tabular}[t]{@{}l@{}}
				Integrante 1 \\
				Integrante 2 \\
				Integrante 3
			\end{tabular} \\
		Profesores: 
			& \begin{tabular}[t]{@{}l@{}}
				Profesor 1 \\
				Profesor 2
			\end{tabular} \\
		Auxiliares: 
			& \begin{tabular}[t]{@{}l@{}}
				Auxiliar 1 \\
				Auxiliar 2
			\end{tabular}\\
		Ayudantes: 
			& \begin{tabular}[t]{@{}l@{}}
				Ayudante 1 \\
				Ayudante 2
			\end{tabular}\\
		\multicolumn{2}{l}{Ayudante del laboratorio: Ayudante} \\
		& \\
		\multicolumn{2}{l}{Fecha de realización: \fecharealizacion} \\
		\multicolumn{2}{l}{Fecha de entrega: \fechaentrega} \\
		\multicolumn{2}{l}{\localizacionuniversidad}
		\end{tabular}
	\end{flushright}
\end{minipage}}


% CONFIGURACIONES
\newcommand{\defaultimagefolder}{images/}         % Directorio de las imágenes 
\newcommand{\defaultnewlinesize}{11pt}            % Tamaño del salto de línea
\newcommand{\defaultinterlind}{1.0}               % Entrelineado por defecto
\newcommand{\tipofuentetitulo}{\huge}             % Tamaño de los títulos
\newcommand{\tipofuentesubtitulo}{\Large}         % Tamaño de los subtítulos
\newcommand{\tipofuentesubsubtitulo}{\large}      % Tamaño de los sub-subtítulos
\newcommand{\tipofuentetituloi}{\LARGE}           % Tamaño de los títulos en el índice
\newcommand{\tipofuentesubtituloi}{\Large}        % Tamaño de los subtítulos en el índice
\newcommand{\tipofuentesubsubtituloi}{\large}     % Tamaño de los sub-subtítulos en el índice
\newcommand{\etipofuentetitulo}{\bfseries}        % Estilo de los títulos
\newcommand{\etipofuentesubtitulo}{\bfseries}     % Estilo de los subtítulos
\newcommand{\etipofuentesubsubtitulo}{\bfseries}  % Estilo de los sub-subtítulos
\newcommand{\etipofuentetituloi}{\bfseries}       % Estilo de los títulos en el índice
\newcommand{\etipofuentesubtituloi}{\bfseries}    % Estilo de los subtítulos en el índice
\newcommand{\etipofuentesubsubtituloi}{\bfseries} % Estilo de los sub-subtítulos en el índice
\newcommand{\tiporeferencias}{apa}                % Tipo de referencias
\newcommand{\nombltcontend}{Índice de Contenidos} % Nombre del índice de contenidos
\newcommand{\nomblttablas}{Lista de Tablas}       % Nombre de la lista de tablas
\newcommand{\nombltfiguras}{Lista de Figuras}     % Nombre de la lista de figuras
\newcommand{\nombltcodfuente}{Lista de Códigos Fuente} % Nombre del código fuente
\newcommand{\nombltwtablas}{Tabla}                % Nombre de las tablas
\newcommand{\nombltwfigura}{Figura}               % Nombre de las figuras
\newcommand{\nombltwcodfuente}{Código Fuente}     % Nombre del código fuente
\newcommand{\indexdepth}{3}                       % Profundidad del índice
\newcommand{\tablepadding}{1.1}                   % Padding de las tablas
\newcommand{\defaultcaptionmargin}{2.9}           % Márgenes de las leyendas por defecto
\newcommand{\defaultpagemarginleft}{2.5}          % Margen izquierdo de las páginas [cm]
\newcommand{\defaultpagemarginright}{2.5}         % Margen derecho de las páginas [cm]
\newcommand{\defaultpagemargintop}{3.0}           % Margen superior de las páginas [cm]
\newcommand{\defaultpagemarginbottom}{2.7}        % Margen inferior de las páginas [cm]
\newcommand{\defaultfirstpagemargintop}{3.8}      % Margen superior de la portada [cm]
\newcommand{\defaultmarginfloatimages}{-13pt}     % Margen sup. de figuras flotantes [pt]
\newcommand{\defaultmargintopimages}{0.0cm}       % Margen sup. de las figuras [cm]
\newcommand{\defaultmarginbottomimages}{-0.2cm}   % Margen inf. de las figuras [cm]
\newcommand{\nombrepaginaportada}{P}              % Etiqueta de la primera página (portada)

% CONFIGURACIONES BOOLEANAS (TRUE, FALSE)
\newcommand{\codigocursoenportada}{false}  % Muestra el código del curso en la portada
\newcommand{\showborderonlinks}{false}     % Muestra un recuadro en cada enlace del documento
\newcommand{\showfooter}{true}             % Muestra el footer (nombre informe y curso)
\newcommand{\showheadertitle}{true}        % Muestra el título de la sección en el header
\newcommand{\showindexofcontents}{true}    % Muestra la lista de contenidos
\newcommand{\showindexoffigures}{true}     % Muestra la lista de figuras en el índice
\newcommand{\showindexofsourcecode}{false} % Muestra la lista de códigos fuente
\newcommand{\showindexoftables}{true}      % Muestra la lista de tablas en el índice
\newcommand{\twocolumnreferences}{false}   % Referencias en dos columnas


% LIBRERÍAS INDEPENDIENTES
\usepackage{amsmath}                   % Fórmulas matemáticas
\usepackage{amssymb}                   % Símbolos matemáticos
\usepackage{amsthm}                    % Teoremas matemáticos
\usepackage{array}                     % Añade nuevas características a las tablas
\usepackage{bigstrut}                  % Lineas horizontales en tablas
\usepackage{booktabs}                  % Permite manejar elementos visuales en tablas
\usepackage[makeroom]{cancel}          % Cancelar términos en fórmulas
\usepackage{caption}                   % Leyendas
\usepackage{color}                     % Colores
\usepackage{colortbl}                  % Administración de color en tablas
\usepackage{datetime}                  % Fechas
\usepackage[inline]{enumitem}          % Permite enumerar ítems
\usepackage[bottom, norule]{footmisc}  % Estilo pié de página
\usepackage{fancyhdr}                  % Encabezados y pié de páginas
\usepackage{float}                     % Administrador de posiciones de objetos
\usepackage{textcomp, gensymb}         % Simbología común                
\usepackage{geometry}                  % Dimensiones y geometría del documento
\usepackage{graphicx}                  % Propiedades extra para los gráficos
\usepackage{ifthen}                    % Permite el manejo de condicionales
\usepackage{mathtools}                 % Permite utilizar notaciones matemáticas avanzadas
\usepackage[version=4]{mhchem}         % Fórmulas químicas
\usepackage{multicol}                  % Múltiples columnas
\usepackage{pdfpages}                  % Permite administrar páginas en pdf
\usepackage{lipsum}                    % Permite crear textos dummy
\usepackage{longtable}                 % Permite utilizar tablas en varias hojas
\usepackage{listings}                  % Permite añadir código fuente
\usepackage{rotating}                  % Permite rotación de objetos
\usepackage{sectsty}                   % Cambia el estilo de los títulos
\usepackage{selinput}                  % Compatibilidad con acentos
\usepackage{setspace}                  % Cambia el espacio entre líneas
\usepackage{subfig}                    % Permite agrupar imágenes
\usepackage{tikz}                      % Permite dibujar
\usepackage{ulem}                      % Permite tachar, subrayar, etc
\usepackage{url}                       % Permite añadir enlaces
\usepackage{wasysym}                   % Contiene caracteres misceláneos
\usepackage{wrapfig}                   % Permite comprimir imágenes
\usepackage{xcolor}                    % Paquete de colores avanzado

% LIBRERÍAS DEPENDIENTES
\usetikzlibrary{babel}                 % Asociado a tikz
\usepackage{chngcntr}                  % Agrega números de secciones a las leyendas
\usepackage{epstopdf}                  % Convierte archivos .eps a pdf
\usepackage{multirow}                  % Agrega nuevas opciones a las tablas
\ifthenelse{\equal{\showborderonlinks}{true}}{
	\usepackage{hyperref}              % Permite añadir enlaces y referencias
}{
	\usepackage[hidelinks]{hyperref}   % Links sin recuadro rojo
}


% DECLARACIÓN DE FUNCIONES
\newcommand{\quotes}[1]{``#1''}        % Insertar cita
\newcommand{\quotesit}[1]{\textit{\quotes{#1}}} % Insertar cita en itálico
\newcommand{\newemptypage}{            % Crea una página vacía
\newpage\null\thispagestyle{empty}\newpage
\addtocounter{page}{-1}}
\newcommand{\setcaptionmargincm}[1]{   % Cambiar el margen
	\captionsetup{margin=#1cm}}
\newcommand{\setpagemargincm}[4]{      % Cambia márgenes de las páginas [cm]
	\newgeometry{left=#1cm, top=#2cm, right=#3cm, bottom=#4cm}}
\newcommand{\newp}{                    % Inserta nueva línea
	\hbadness=10000 \vspace{\defaultnewlinesize} \par}
\newcommand{\newpar}[1]{               % Nuevo párrafo
	\hbadness=10000 #1 \newp}
\newcommand{\newparnl}[1]{#1 \par}     % Nuevo párrafo sin nueva linea al final
\newcommand{\lpow}[2]{{#1}_{#2}}       % Insertar sub-índice
\newcommand{\pow}[2]{{#1}^{#2}}        % Insertar elevado
\newcommand{\fracpartial}[2]{          % Fracción de derivadas parciales af/ax
	\frac{\partial #1}{\partial #2}}
\newcommand{\fracdpartial}[2]{         % Fracción de derivadas parciales dobles a^2/ax^2
	\frac{{\partial}^{2} #1}{\partial {#2}^{2}}}
\newcommand{\fracnpartial}[3]{         % Fracción de derivadas parciales en n a^n/ax^n
	\frac{{\partial}^{#3} #1}{\partial {#2}^{#3}}}
\newcommand{\fracderivat}[2]{          % Fracción de derivadas df/dx
	\frac{\text{d} #1}{\text{d} #2}}
\newcommand{\fracdderivat}[2]{         % Fracción de derivadas dobles d^2/dx^2
	\frac{{\text{d}}^{2} #1}{\text{d} {#2}^{2}}}
\newcommand{\fracnderivat}[3]{         % Fracción de derivadas en n d^n/dx^n
	\frac{{\text{d}}^{#3} #1}{\text{d} {#2}^{#3}}}
\newcommand{\topequal}[2]{             % Llave superior de equivalencia
	\overbrace{#1}^{\mathclap{#2}}}
\newcommand{\underequal}[2]{           % Llave inferior de equivalencia
	\underbrace{#1}_{\mathclap{#2}}}
\newcommand{\topsequal}[2]{            % Rectángulo superior de equivalencia
	\overbracket{#1}^{\mathclap{#2}}}
\newcommand{\undersequal}[2]{          % Rectángulo inferior de equivalencia
	\underbracket{#1}_{\mathclap{#2}}}
\newcommand{\resizeitem}[2]{           % Crea un resizebox de tamaño textwidth
	\resizebox{#1\textwidth}{!}{#2}
}
\newcommand{\newtitleanum}[1]{
	% Insertar un título sin número
	\addcontentsline{toc}{section}{#1}
	\section*{#1}
	\ifthenelse{\equal{\showheadertitle}{true}}{
		\fancyhead[L]{\nouppercase{#1}}}{}
	\stepcounter{section}
}
\newcommand{\newtitleanumheadless}[1]{
	% Insertar un título sin número sin alterar el header
	\addcontentsline{toc}{section}{#1}
	\section*{#1}
	\stepcounter{section}
}
\newcommand{\newsubtitleanum}[1]{
	% Insertar un subtítulo sin número
	\addcontentsline{toc}{subsection}{#1}
	\subsection*{#1}
	\stepcounter{subsection}
}
\newcommand{\newsubsubtitleanum}[1]{
	% Insertar un sub-subtítulo sin número
	\addcontentsline{toc}{subsubsection}{#1}
	\subsubsection*{#1}
	\stepcounter{subsubsection}
}
\newcommand{\newtitleanumnoi}[1]{
	% Insertar un título sin número sin indexar
	\section*{#1}
	\ifthenelse{\equal{\showheadertitle}{true}}{
		\fancyhead[L]{\nouppercase{#1}}}{}
	\stepcounter{section}
}
\newcommand{\newtitleanumnoiheadless}[1]{
	% Insertar un título sin número sin indexar sin cambiar el header
	\section*{#1}
	\ifthenelse{\equal{\showheadertitle}{true}}{
		\fancyhead[L]{\nouppercase{#1}}}{}
	\stepcounter{section}
}
\newcommand{\newsubtitleanumnoi}[1]{
	% Insertar un subtítulo sin número sin indexar
	\subsection*{#1}
	\stepcounter{subsection}
}
\newcommand{\newsubsubtitleanumnoi}[1]{
	% Insertar un sub-subtítulo sin num. sin indexar
	\addcontentsline{toc}{subsubsection}{#1}
	\subsubsection*{#1}
	\stepcounter{subsubsection}
}
\newcommand{\insertequation}[2][]{
	% Insertar una ecuación
	\vspace{-0.1cm}
	\begin{equation}
		\text{#1} #2
	\end{equation}
	\vspace{-0.23cm}
	\par
}
\newcommand{\insertequationcaptioned}[3][]{
	% Insertar una ecuación con leyenda
	\vspace{-0.1cm}
	\begin{equation}
		\text{#1} #2
	\end{equation}
	\begin{center}
		\vspace{-0.15cm}
		\textit{#3} \par
		\vspace{0.05cm}
	\end{center}
}
\newcommand{\insertequationgathered}[2][]{
	% Insertar una ecuación con el ambiente gather
	\vspace{-0.4cm}
	\begin{gather}
		\text{#1} #2
	\end{gather}
	\par
	\vspace{-0.10cm}
}
\newcommand{\insertequationgatheredcaptioned}[3][]{
	% Insertar una ecuación (gather) con leyenda
	\vspace{-0.3cm}
	\begin{gather}
		\text{#1} #2
	\end{gather}
	\begin{center}
		\vspace{-0.15cm}
		\textit{#3} \par
	\end{center}
}
\newcommand{\insertequationalign}[2][]{
	% Insertar una ecuación con el ambiente align
	\vspace{-0.4cm}
	\begin{align}
		\text{#1} #2
	\end{align}
	\par
	\vspace{-0.10cm}
}
\newcommand{\insertequationaligncaptioned}[3][]{
	% Insertar una ecuación (align) con leyenda
	\vspace{-0.1cm}
	\begin{align}
		\text{#1} #2
	\end{align}
	\begin{center}
		\vspace{-0.15cm}
		\textit{#3} \par
	\end{center}
}
\newcommand{\insertimage}[4][]{
	% Insertar una imagen
	\vspace{\defaultmargintopimages}
	\begin{figure}[H]
		\centering
		\includegraphics[#3]{\defaultimagefolder#2}
		\caption{#4 #1}
	\end{figure}
	\vspace{\defaultmarginbottomimages}
}
\newcommand{\insertimageboxed}[4][]{
	% Insertar una imagen con recuadro
	\vspace{\defaultmargintopimages}
	\begin{figure}[H]
		\centering
		\fbox{\includegraphics[#3]{\defaultimagefolder#2}}
		\caption{#4 #1}
	\end{figure}
	\vspace{\defaultmarginbottomimages}
}
\newcommand{\insertimagefixed}[5][]{
	% Insertar una imagen de ancho fijo a la página
	\vspace{\defaultmargintopimages}
	\begin{figure}[H]
		\centering
		\resizebox{#3\textwidth}{!}{
			\includegraphics[#4]{\defaultimagefolder#2}
		}
		\caption{#5 #1}
	\end{figure}
	\vspace{\defaultmarginbottomimages}
}
\newcommand{\insertimageboxedfixed}[5][]{
	% Insertar una imagen recuadrada de ancho fijo
	\vspace{\defaultmargintopimages}
	\begin{figure}[H]
		\centering
		\resizebox{#3\textwidth}{!}{
			\fbox{\includegraphics[#4]{\defaultimagefolder#2}}
		}
		\caption{#5 #1}
	\end{figure}
	\vspace{\defaultmarginbottomimages}
}
\newcommand{\insertdoubleimage}[8][]{
	% Insertar una imagen doble
	\vspace{\defaultmargintopimages}
	\captionsetup{margin=0.45cm}
	\begin{figure}[H] \centering
		\subfloat[#4]{
			\includegraphics[#3]{\defaultimagefolder#2}}
		\hspace{0.2cm}
		\subfloat[#7]{
			\includegraphics[#6]{\defaultimagefolder#5}}
		\setcaptionmargincm{\defaultcaptionmargin}
		\caption{#8 #1}
	\end{figure}
	\setcaptionmargincm{\defaultcaptionmargin}
	\vspace{\defaultmarginbottomimages}
}
\newcommand{\inserttripleimage}[8][]{
	% Insertar una imagen triple
	\vspace{\defaultmargintopimages}
	\captionsetup{margin=0.45cm}
	\begin{figure}[H] \centering
		\subfloat[]{
			\includegraphics[#3]{\defaultimagefolder#2}}
		\hspace{0.1cm}
		\subfloat[]{
			\includegraphics[#5]{\defaultimagefolder#4}}
		\hspace{0.1cm}
		\subfloat[]{
			\includegraphics[#7]{\defaultimagefolder#6}}
		\setcaptionmargincm{\defaultcaptionmargin}
		\caption{#8 #1}
	\end{figure}
	\setcaptionmargincm{\defaultcaptionmargin}
	\vspace{\defaultmarginbottomimages}
}
\newcommand{\insertimageleft}[5][]{
	% Insertar una imagen a la izquierda
	\begin{wrapfigure}[#5]{l}{#3\textwidth}
		\setcaptionmargincm{0}
		\vspace{\defaultmarginfloatimages}
		\centering\includegraphics[width=\linewidth]{\defaultimagefolder#2}
		\caption{#4 #1}
		\setcaptionmargincm{\defaultcaptionmargin}
	\end{wrapfigure}
}
\newcommand{\insertimageright}[5][]{
	% Insertar una imagen a la derecha
	\begin{wrapfigure}[#5]{r}{#3\textwidth}
		\setcaptionmargincm{0}
		\vspace{\defaultmarginfloatimages}
		\centering\includegraphics[width=\linewidth]{\defaultimagefolder#2}
		\caption{#4 #1}
		\setcaptionmargincm{\defaultcaptionmargin}
	\end{wrapfigure}
}


% DECLARACIÓN DE AMBIENTES Y ESTILOS
\definecolor{dkgreen}{rgb}{0,0.6,0}
\definecolor{gray}{rgb}{0.5,0.5,0.5}
\definecolor{mauve}{rgb}{0.58,0,0.82}
\definecolor{mygreen}{rgb}{0,0.6,0}
\definecolor{mygray}{rgb}{0.5,0.5,0.5}
\definecolor{mymauve}{rgb}{0.58,0,0.82}
\definecolor{codegreen}{rgb}{0,0.6,0}
\definecolor{codegray}{rgb}{0.5,0.5,0.5}
\definecolor{codepurple}{rgb}{0.58,0,0.82}
\definecolor{backcolour}{rgb}{0.95,0.95,0.92}
\SelectInputMappings{
	% Definición de acentos
	aacute={á},
	Ntilde={Ñ},
	Euro={€}
}
\lstdefinestyle{C}{
	% Estilo de lenguaje C
	language=C,
  	numbers=left,
  	stepnumber=1,        
  	numbersep=5pt,
  	backgroundcolor=\color{white}, 
  	showspaces=false,
  	showstringspaces=false,
  	showtabs=false,
  	tabsize=2,
  	captionpos=b,
  	breaklines=true,
  	breakatwhitespace=true,
  	title=\lstname}
\lstdefinestyle{Java}{
	% Estilo de lenguaje Java
	language=Java,
	aboveskip=3mm,
	belowskip=3mm,
	showstringspaces=false,
	columns=flexible,
	basicstyle={\small\ttfamily},
	numbers=left,
	numberstyle=\tiny\color{gray},
	keywordstyle=\color{blue},
	commentstyle=\color{dkgreen},
	stringstyle=\color{mauve},
	breaklines=true,
	breakatwhitespace=true,
	tabsize=3,
	backgroundcolor=\color{backcolour}}
\lstdefinestyle{Matlab}{
	% Estilo de lenguaje Matlab
	language=Matlab,
	breaklines=true,
	morekeywords={matlab2tikz},
	keywordstyle=\color{blue},
	morekeywords=[2]{1}, keywordstyle=[2]{\color{black}},
	backgroundcolor=\color{backcolour},
	identifierstyle=\color{black},
	stringstyle=\color{mylilas},
	commentstyle=\color{mygreen},
	showstringspaces=false,
	numbers=left,
	showstringspaces=false,
	numberstyle={\tiny \color{black}},
	numbersep=9pt,
	basicstyle={\small\ttfamily},
	tabsize=3,
	breaklines=true,
	aboveskip=3mm,
	belowskip=3mm,
	emph=[1]{for,end,break},emphstyle=[1]\color{red}}
\lstdefinestyle{Python}{
	% Estilo de lenguaje Python
	language=Python,
    backgroundcolor=\color{backcolour},   
    commentstyle=\color{codegreen},
    keywordstyle=\color{magenta},
    numberstyle=\tiny\color{codegray},
    stringstyle=\color{codepurple},
    basicstyle=\footnotesize,
    breakatwhitespace=false,         
    breaklines=true,                 
    captionpos=b,                    
    keepspaces=true,                 
    numbers=left,                    
    numbersep=5pt,                  
    showspaces=false,                
    showstringspaces=false,
    showtabs=false,                  
    tabsize=3,
    basicstyle={\small\ttfamily}
}
\lstset{literate=
	{á}{{\'a}}1 {é}{{\'e}}1 {í}{{\'i}}1 {ó}{{\'o}}1 {ú}{{\'u}}1
	{Á}{{\'A}}1 {É}{{\'E}}1 {Í}{{\'I}}1 {Ó}{{\'O}}1 {Ú}{{\'U}}1
	{à}{{\`a}}1 {è}{{\`e}}1 {ì}{{\`i}}1 {ò}{{\`o}}1 {ù}{{\`u}}1
	{À}{{\`A}}1 {È}{{\'E}}1 {Ì}{{\`I}}1 {Ò}{{\`O}}1 {Ù}{{\`U}}1
	{ä}{{\"a}}1 {ë}{{\"e}}1 {ï}{{\"i}}1 {ö}{{\"o}}1 {ü}{{\"u}}1
	{Ä}{{\"A}}1 {Ë}{{\"E}}1 {Ï}{{\"I}}1 {Ö}{{\"O}}1 {Ü}{{\"U}}1
	{â}{{\^a}}1 {ê}{{\^e}}1 {î}{{\^i}}1 {ô}{{\^o}}1 {û}{{\^u}}1
	{Â}{{\^A}}1 {Ê}{{\^E}}1 {Î}{{\^I}}1 {Ô}{{\^O}}1 {Û}{{\^U}}1
	{œ}{{\oe}}1 {Œ}{{\OE}}1 {æ}{{\ae}}1 {Æ}{{\AE}}1 {ß}{{\ss}}1
	{ű}{{\H{u}}}1 {Ű}{{\H{U}}}1 {ő}{{\H{o}}}1 {Ő}{{\H{O}}}1
	{ç}{{\c c}}1 {Ç}{{\c C}}1 {ø}{{\o}}1 {å}{{\r a}}1 {Å}{{\r A}}1
	{€}{{\EUR}}1 {£}{{\pounds}}1
}
\newcolumntype{P}[1]{>{\centering\arraybackslash}p{#1}} % Columna centrada en tablas


% CONFIGURACIÓN INICIAL DEL DOCUMENTO
\decimalpoint                              % Se define el punto decimal
\counterwithin{equation}{section}          % Añade número de sección a las ecuaciones
\counterwithin{figure}{section}            % Añade número de sección a las figuras
\counterwithin{table}{section}             % Añade número de sección a las tablas
\bibliographystyle{\tiporeferencias}       % Estilo APA para las referencias
\setcaptionmargincm{\defaultcaptionmargin} % Margen por defecto
\setlength{\headheight}{64pt}              % Tamaño de la cabecera sin fancyhdr
\setcounter{tocdepth}{\indexdepth}         % Se ajusta la profundidad del índice
\setcounter{MaxMatrixCols}{20}             % Número máximo de columnas para las matrices
\hypersetup{
	pdfauthor={\autordeldocumento},
	pdftitle={\nombredelinforme},
	pdfsubject={\temaatratar},
	pdfkeywords={\nombreuniversidad, \nombredelcurso,
	\codigodelcurso, \localizacionuniversidad},
	pdfcreator={pdfLaTeX, ppizarror},
	pdfproducer={Template LaTeX informe v2.3.5}}
\renewcommand{\baselinestretch}{\defaultinterlind} % Ajuste del entrelineado
\makeatletter
\ifthenelse{\equal{\twocolumnreferences}{true}}{
	% Bibliografía en 2 columnas
	\renewenvironment{thebibliography}[1]
	{\begin{multicols}{2}[\section*{\refname}]
		\@mkboth{\MakeUppercase\refname}{\MakeUppercase\refname}
		\list{\@biblabel{\@arabic\c@enumiv}}
		{\settowidth\labelwidth{\@biblabel{#1}}
			\leftmargin\labelwidth
			\advance\leftmargin\labelsep
			\@openbib@code
			\usecounter{enumiv}
			\let\p@enumiv\@empty
			\renewcommand\theenumiv{\@arabic\c@enumiv}}
		\sloppy
		\clubpenalty 4000
		\@clubpenalty \clubpenalty
		\widowpenalty 4000
		\sfcode`\.\@m}
		{\def\@noitemerr
			{\@latex@warning{Ambiente `thebibliography' no definido}}
			\endlist\end{multicols}
		}}{}
\makeatother
	
	
% PORTADA
\begin{document}
\newpage
\renewcommand{\thepage}{\nombrepaginaportada}
\setpagemargincm{\defaultpagemarginleft}{\defaultfirstpagemargintop}
{\defaultpagemarginright}{\defaultpagemarginbottom}
\pagestyle{fancy}
\fancyhf{}
\fancyhead[L]{\nombreuniversidad \\ \nombrefacultad \\ \departamentouniversidad}
\fancyhead[R]{\includegraphics[scale=\imagendeldepartamentoescl]{\imagendeldepartamento}}
\ifthenelse{\equal{\codigocursoenportada}{true}}{
	\vspace*{3cm}
	\begin{center}
		\huge  {\nombredelcurso} \\
		\vspace{0.3cm}
		\large  {Código del curso: \codigodelcurso} \\
		\vspace{1.5cm}
		\Huge {\nombredelinforme} \\
		\vspace{0.3cm}
		\large {\temaatratar}
	\end{center}
}{
	\vspace*{5cm}
	\begin{center}
		\huge  {\nombredelcurso} \\
		\vspace{1cm}
		\Huge {\nombredelinforme} \\
		\vspace{0.3cm}
		\large {\temaatratar}
	\end{center}
}
\vfill
\tablaintegrantes


% CONFIGURACIÓN DE PÁGINA Y ENCABEZADOS
\newpage
\pagenumbering{Roman}
\setcounter{page}{1}
\setcounter{footnote}{1}
\setpagemargincm{\defaultpagemarginleft}{\defaultpagemargintop}
{\defaultpagemarginright}{\defaultpagemarginbottom}
\def\arraystretch{\tablepadding}                     % Se ajusta el padding de las tablas
\renewcommand{\sectionmark}[1]{\markboth{#1}{}}      % Se modifica el estilo del header
\renewcommand{\listfigurename}{\nombltfiguras}       % Nombre del índice de figuras
\renewcommand{\listtablename}{\nomblttablas}         % Nombre del índice de tablas
\renewcommand{\contentsname}{\nombltcontend}         % Nombre del índice
\renewcommand{\lstlistlistingname}{\nombltcodfuente} % Nombre del índice de códigos fuente
\renewcommand{\tablename}{\nombltwtablas}            % Nombre de la leyenda de las tablas
\renewcommand{\figurename}{\nombltwfigura}           % Nombre de la leyenda de las figuras
\renewcommand{\lstlistingname}{\nombltwcodfuente}    % Nombre de la leyenda del código fuente
\pagestyle{fancy} \fancyhf{}                         % Se crean los headers y footers
\ifthenelse{\equal{\showheadertitle}{true}}{         % Header izq, nombre sección
	\fancyhead[L]{\nouppercase{\rightmark}}}{}
\fancyhead[R]{\small \rm \nouppercase{\thepage}}     % Header der, número página
\ifthenelse{\equal{\showfooter}{true}}{
	\fancyfoot[L]{\small \rm \textit{\nombredelinforme}} % Footer izq, título del informe
	\fancyfoot[R]{\small \rm \textit{\codigodelcurso \ \nombredelcurso}} % Footer der, curso
	\renewcommand{\footrulewidth}{0.5pt}             % Ancho de la barra del footer
}{}
\renewcommand{\headrulewidth}{0.5pt}                 % Ancho de la barra del header
\sectionfont{\tipofuentetitulo \etipofuentetitulo \selectfont} % Tipo de título para abstract


% ================================== RESUMEN O ABSTRACT =================================
\newtitleanumheadless{Resumen}

% Ejemplo de dos párrafos en latín, esta linea puede borrarse
\lipsum[1] \newp \lipsum[3]


% TABLA DE CONTENIDOS - ÍNDICE
\newpage
\sectionfont{\tipofuentetituloi \etipofuentetituloi \selectfont}
\subsectionfont{\tipofuentesubtituloi \etipofuentesubtituloi \selectfont}
\subsubsectionfont{\tipofuentesubsubtituloi \etipofuentesubsubtituloi \selectfont}
\ifthenelse{\equal{\showindexofcontents}{true}}{\tableofcontents}{}     % Tabla de contenidos
\ifthenelse{\equal{\showindexoffigures}{true}}{\listoffigures}{}        % Índice de figuras
\ifthenelse{\equal{\showindexoftables}{true}}{\listoftables}{}          % Índice de tablas
\ifthenelse{\equal{\showindexofsourcecode}{true}}{\lstlistoflistings}{} % Lista de códigos fuente


% CONFIGURACIONES FINALES - INICIO DE LAS SECCIONES
\newpage
\ifthenelse{\equal{\showheadertitle}{true}}{
	\fancyhead[L]{\nouppercase{\leftmark}}}{}
\sectionfont{\tipofuentetitulo \etipofuentetitulo \selectfont}
\subsectionfont{\tipofuentesubtitulo \etipofuentesubtitulo \selectfont}
\subsubsectionfont{\tipofuentesubsubtitulo \etipofuentesubsubtitulo \selectfont}
\renewcommand{\thepage}{\arabic{page}}
\setcounter{page}{1}
\setcounter{section}{0}
\setcounter{footnote}{0}


% ================================ INICIO DEL DOCUMENTO ================================

% SE INCLUYE EL EJEMPLO DEL DOCUMENTO, ESTA LÍNEA PUEDE BORRARSE SIN PROBLEMA
% Template:     Archivo de ejemplo
% Versión:      2.0.1 (06/09/2016)
% Codificación: UTF-8
%
% Autor: Pablo Pizarro R.
%        Facultad de Ciencias Físicas y Matemáticas.
%        Universidad de Chile.
%
% Licencia: CC BY-NC-SA 4.0 (http://creativecommons.org/licenses/by-nc-sa/4.0/)

% NUEVA SECCIÓN
% Las secciones se inician con \section, si se quiere una sección sin "número" se pueden usar las funciones \newtitleanum (nuevo titulo sin numeración) o la función \newtitleanumnoi para crear el mismo título sin numerar pero sin aparecer en el índice.
\section{Informes con \LaTeX}

	% SUB-SECCIÓN
	% Las subsecciones se inician con \subsection, si se quiere una subsección sin "número" se pueden usar las funciones \newsubtitleanum (nuevo subtitulo sin numeración) o la función \newsubtitleanumnoi para crear el mismo subtítulo sin numerar pero sin aparecer en el índice.
	\subsection{Una breve introducción}
	
		% Esta función sirve para rellenar con un párrafo.
		\lipsum[2]
		
		% Se inserta una ecuación, el primer parámetro entre corchetes es opcional (permite identificar con una etiqueta para poder referenciarlo después con \ref), seguido de aquello se escribe la ecuación en modo bruto sin signos peso.
		\insertequation[\label{eqn:identidad-imposible}]{\pow{a}{k}=\pow{b}{k}+\pow{c}{k}, \forall k>2}
		
		% Los párrafos se pueden añadir con \newpar o simplemente se puede escribir, esta función se hizo para evitar errores y warnings por parte del compilador. Además da la opción de poder cambiar la estructura de todos los párrafos del documento directamente al cambiar las propiedades de la función (definida en la sección DECLARACIÓN DE FUNCIONES).
		\newpar{Este es un párrafo, puede contener múltiples \quotes{Expresiones} así como \quotesit{Citas en itálico} o referencias \footnote{Las referencias se hacen utilizando la expresión \textbackslash label\{etiqueta\}} a fórmulas como (\ref{formulasinsentido}), los párrafos añaden una entrada libre por defecto, a continuación se muestra un ejemplo de inserción de imágenes (como la Figura \ref{testimage}):}
		
		% Para insertar una imagen se puede usar la funcion \insertimage la cual toma un primer parámetro opcional para definir una etiqueta, luego toma la dirección de la imagen, sus parámetros (en este caso se definió la escala de 0.2) y un título de imagen (el cual va debajo de la imagen)
		\insertimage[\label{testimage}]{ejemplos/test-image.png}{scale=0.2}{Where are you? de \quotesit{Internet}}
		
		\newparnl{Este es un párrafo sin nueva linea (de ahí el \textit{nl}, esto se puede usar para terminar un tema, o si vienen imágenes nuevas), si no te gustan los comandos \textbf{newpar} o \textbf{newparnl} simplemente puedes usar los salto de línea convencionales. Además puedes cambiarle el nombre a las funciones, así puedes tener comandos más intuitivos para ti.}
		
	% SUB-SECCIÓN
	\subsection{Tablas!}

		\newparnl{También puedes usar tablas, insertarlas es muy fácil, puedes usar directamente el \quotes{conversor de tablas} \textsuperscript{\cite{conversortabla}}, ahí puedes convertir en un solo clic tablas Excel, o crearlas tú mismo sin tener que hacer todo el aburridísimo código.}
		
		\begin{longtable}{ccc}
			\caption{Esta es una tabla que se \quotes{corta} en varias páginas si es que le falta espacio, útil para cuando tienes que pegar tablas con más de 60 o 1000? filas.}\label{foo}\\
			\hline
			Columna 1 & Columna 2 & Columna 3\\\hline
			\endfirsthead
			\hline
			Columna 1 & Columna 2 & Columna 3\\
			\hline
			\endhead
			\hline
			\endfoot
			\hline
			\endlastfoot
			$\omega$ & $\nu$ & $\delta$\\     
			$\partial$ & $\nabla$ & $\mho$\\
			$\beta$ & $\gamma$ & $\epsilon$\\   
			$\varepsilon$ & $\upsilon$ & $\varphi$\\
			$\Phi$ & $\Theta$ & $\varSigma$\\
			$\omega$ & $\nu$ & $\delta$\\     
			$\partial$ & $\nabla$ & $\mho$\\ 
		\end{longtable}
		
% NUEVA SECCIÓN
\section{Aquí un nuevo tema}
		
	% SUB-SECCIÓN
	\subsection{Haciendo informes como un profesional}
	
		% Se inserta una imagen fija en la izquierda del documento con \insertimageleft, al igual que las demás funciones, el primer parámetro es opcional, luego viene la ubicación de la imagen, seguido de la escala, un título y por último el número de líneas que 'cortará' hacia abajo. Para insertar una imagen fija en la derecha se utiliza \insertimageright usando los mismos parámetros.
		\insertimageleft[\label{imagen-izquierda}]{ejemplos/test-image-wrap}{0.22}{Apolo}{14}
		
		Test es una palabra inglesa aceptada por la Real Academia Española (RAE). Este concepto hace referencia a las pruebas destinadas a evaluar conocimientos, aptitudes o funciones. La palabra test puede utilizarse como sinónimo de examen. Los exámenes son muy frecuentes en el ámbito educativo ya que permiten evaluar los conocimientos adquiridos por los estudiantes. Los exámenes pueden ser orales o escritos, con preguntas de respuestas abiertas (donde el estudiante responde libremente) o preguntas de respuestas múltiples (el estudiante debe seleccionar la respuesta correcta de un listado).
		
		% Se inserta un nuevo párrafo de modo inteligente, evita errores de hbox, crea un nuevo par e espacia de manera fija
		\newp
		
		\lipsum[5]
		
		\insertequationcaptioned[\label{formulasinsentido}]{\int_{a}^{b} f(x) dx = \fracnpartial{f(x)}{x}{\eta} \cdotp \textstyle \sum_{x=a}^{b} f(x)\cancelto{1+\frac{\epsilon}{k}}{(1+\Delta x)}}{Ecuación sin sentido}
		
		\begin{wrapfigure}[16]{r}{0.4\textwidth}
			\setcaptionmargincm{0}
			\vspace{-1.0cm}
			\begin{center}
				\begin{tikzpicture}[domain=0:4] 
					\draw[very thin,color=gray] (-0.1,-1.1) grid (3.9,3.9);
					\draw[->] (-0.2,0) -- (4.2,0) node[right] {$x$}; 
					\draw[->] (0,-1.2) -- (0,4.2) node[above] {$f(x)$};
					\draw[color=red]    plot (\x,\x)             node[right] {$f(x) =x$}; 
					\draw[color=blue]   plot (\x,{sin(\x r)})    node[right] {$f(x) = \sin x$}; 
					\draw[color=orange] plot (\x,{0.05*exp(\x)}) node[right] {$f(x) = \frac{1}{20} \mathrm e^x$};
				 \end{tikzpicture}
			\end{center}
			\caption{También puedes graficar con \LaTeX !}
			\setcaptionmargincm{\defaultcaptionmargin}
		\end{wrapfigure}
		
		\lipsum[2]
		\vspace{\defaultnewlinesize}
		\lipsum[6]

% REFERENCIAS
\newpage
\begin{thebibliography}{99}
	\addcontentsline{toc}{section}{Referencias}
	
	\bibitem{referencia1} 
		\hbadness=10000 Autor 1, Autor 2.
		\textit{Título}.
		Editorial, versión(revisión):página1:página2, año.
		\\\url{https://www.enlace.com/}
		
	\bibitem{einstein}
		\hbadness=10000 El mismísimo Albert Einstein. 
		\textit{Otro titulo complicado e interesante}. 
		Análisis de físicas de informes, 322(10):891–921, 1905.
	
	\bibitem{pjd} 
		\hbadness=10000 Pedro, Juan y Diego. 
		\textit{Como hacer informes en \LaTeX}. 
		Universidad de Chile, Facultad de Ciencias Físicas y Matemáticas, 2016.
	
	\bibitem{conversortabla}
		\hbadness=10000 Tables Generator.
		\textit{Convierte fácilmente tus tablas, o crea unas con un intuitivo editor de tablas.}
		\\\url{http://www.tablesgenerator.com/}
	
\end{thebibliography}


% FIN DEL DOCUMENTO
\end{document}