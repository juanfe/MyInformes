% Template:     Informe/Reporte LaTeX
% Documento:    Configuración inicial del template
% Versión:      3.2.4 (04/05/2017)
% Codificación: UTF-8
%
% Autor: Pablo Pizarro R.
%        Facultad de Ciencias Físicas y Matemáticas.
%        Universidad de Chile.
%        pablo.pizarro@ing.uchile.cl, ppizarror.com
%
% Sitio web del proyecto: [http://ppizarror.com/Template-Informe/]
% Licencia: MIT           [https://opensource.org/licenses/MIT]

\decimalpoint                      % Se define el punto decimal
\counterwithin{equation}{section}  % Añade número de sección a las ecuaciones
\counterwithin{figure}{section}    % Añade número de sección a las figuras
\counterwithin{table}{section}     % Añade número de sección a las tablas
\bibliographystyle{\typereference} % Estilo APA para las referencias
\setlength{\headheight}{64pt}      % Tamaño de la cabecera sin fancyhdr
\setcounter{tocdepth}{\indexdepth} % Se ajusta la profundidad del índice
\setcounter{MaxMatrixCols}{20}     % Número máximo de columnas en matrices

% Configuración caption
\captionsetup{                     % Configuración de márgenes en captions
	belowskip=\captionbottommargin pt,
	aboveskip=\captiontopmargin pt}
\ifthenelse{\equal{\figurecaptiontop}{true}}{
	% Se dejan los caption en la parte superior para las figuras
	\floatsetup[figure]{capposition=top}
}{}

% Definición de espaciados
\hfuzz=100pt
\vfuzz=100pt
\hbadness=2000
\vbadness=\maxdimen

\hypersetup{
	pdfauthor={\autordeldocumento},
	pdftitle={\nombredelinforme},
	pdfsubject={\temaatratar},
	pdfkeywords={\nombreuniversidad, \nombredelcurso,
	\codigodelcurso, \localizacionuniversidad},
	pdfcreator={pdfLaTeX, ppizarror},
	pdfproducer={Template LaTeX informe v\templateversion}
}
\renewcommand{\baselinestretch}{\defaultinterlind} % Ajuste del entrelineado
\setcaptionmargincm{\defaultcaptionmargin}         % Margen por defecto
\ifthenelse{\equal{\referenceassection}{true}}{
	% Se configura las referencias como una sección
	\patchcmd{\thebibliography}{*}{}{}{}
}{}
\makeatletter
\ifthenelse{\equal{\twocolumnreferences}{true}}{
	% Bibliografía en 2 columnas
	\renewenvironment{thebibliography}[1]
	{\begin{multicols}{2}[\section*{\refname}]
		\@mkboth{\MakeUppercase\refname}{\MakeUppercase\refname}
		\list{\@biblabel{\@arabic\c@enumiv}}
		{\settowidth\labelwidth{\@biblabel{#1}}
			\leftmargin\labelwidth
			\advance\leftmargin\labelsep
			\@openbib@code
			\usecounter{enumiv}
			\let\p@enumiv\@empty
			\renewcommand\theenumiv{\@arabic\c@enumiv}}
		\sloppy
		\clubpenalty 4000
		\@clubpenalty \clubpenalty
		\widowpenalty 4000
		\sfcode`\.\@m}
		{\def\@noitemerr
		{\@latex@warning{Ambiente `thebibliography' no definido}}
		\endlist\end{multicols}}}{}
\makeatother