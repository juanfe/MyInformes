% Template:     Informe/Reporte LaTeX
% Documento:    Funciones para insertar títulos
% Versión:      4.6.7 (04/11/2017)
% Codificación: UTF-8
%
% Autor: Pablo Pizarro R.
%        Facultad de Ciencias Físicas y Matemáticas
%        Universidad de Chile
%        pablo.pizarro@ing.uchile.cl, ppizarror.com
%
% Manual template: [http://latex.ppizarror.com/Template-Informe/]
% Licencia MIT:    [https://opensource.org/licenses/MIT/]

\newcommand{\sectionanum}[1]{
	% Insertar un título sin número
	% 	#1	Título
	\emptyvarerr{\sectionanum}{#1}{Titulo no definido}
	\phantomsection
	\needspace{3\baselineskip}
	\section*{#1}
	\addcontentsline{toc}{section}{#1}
	\ifthenelse{\equal{\anumsecaddtocounter}{true}}{\stepcounter{section}}{}
	\markboth{#1}{}
}

\newcommand{\sectionanumnoi}[1]{
	% Insertar un título sin número y sin indexar
	% 	#1	Título
	\emptyvarerr{\sectionanumnoi}{#1}{Titulo no definido}
	\phantomsection
	\needspace{3\baselineskip}
	\section*{#1}
	\ifthenelse{\equal{\anumsecaddtocounter}{true}}{\stepcounter{section}}{}
	\markboth{#1}{}
}

\newcommand{\sectionanumheadless}[1]{
	% Insertar un título sin número sin cambiar el título del header
	% 	#1	Título
	\emptyvarerr{\sectionanumnoheadless}{#1}{Titulo no definido}
	\section*{#1}
	\addcontentsline{toc}{section}{#1}
	\ifthenelse{\equal{\anumsecaddtocounter}{true}}{\stepcounter{section}}{}
}

\newcommand{\sectionanumnoiheadless}[1]{
	% Insertar un título sin número, sin indexar y sin cambiar el título del header
	% 	#1	Título
	\emptyvarerr{\sectionanumnoi}{#1}{Titulo no definido}
	\section*{#1}
	\ifthenelse{\equal{\anumsecaddtocounter}{true}}{\stepcounter{section}}{}
}

\newcommand{\subsectionanum}[1]{
	% Insertar un subtítulo sin número
	% 	#1	Subtítulo
	\emptyvarerr{\subsectionanum}{#1}{Subtitulo no definido}
	\subsection*{#1}
	\addcontentsline{toc}{subsection}{#1}
	\ifthenelse{\equal{\anumsecaddtocounter}{true}}{\stepcounter{subsection}}{}
}

\newcommand{\subsectionanumnoi}[1]{
	% Insertar un subtítulo sin número y sin indexar
	% 	#1	Subtítulo
	\emptyvarerr{\subsectionanumnoi}{#1}{Subtitulo no definido}
	\subsection*{#1}
	\ifthenelse{\equal{\anumsecaddtocounter}{true}}{\stepcounter{subsection}}{}
}

\newcommand{\subsubsectionanum}[1]{
	% Insertar un sub-subtítulo sin número
	% 	#1	Sub-subtítulo
	\emptyvarerr{\subsubsectionanum}{#1}{Sub-subtitulo no definido}
	\subsubsection*{#1}
	\addcontentsline{toc}{subsubsection}{#1}
	\ifthenelse{\equal{\anumsecaddtocounter}{true}}{\stepcounter{subsubsection}}{}
}

\newcommand{\subsubsectionanumnoi}[1]{
	% Insertar un sub-subtítulo sin número y sin indexar
	% 	#1	Sub-subtítulo
	\emptyvarerr{\subsubsectionanumnoi}{#1}{Sub-subtitulo no definido}
	\subsubsection*{#1}
	\ifthenelse{\equal{\anumsecaddtocounter}{true}}{\stepcounter{subsubsection}}{}
}

\newcommand{\insertindextitle}[2]{
	% Insertar un título en un índice
	%	#1	Título
	%	#2	Margen superior en pt. (opcional), 10pt por defecto
	\emptyvarerr{\insertindextitle}{#1}{Titulo no definido}
	\ifx\hfuzz#2\hfuzz
		\addtocontents{toc}{\protect\addvspace{\indextitlemargin pt}}
	\else
		\addtocontents{toc}{\protect\addvspace{#2 pt}}
	\fi
	\addtocontents{toc}{\noindent\hyperref[swpn]{\textbf{#1}}}
}

\newcommand{\newchapter}[1]{
	% Crear un capítulo como una sección
	%	#1	Título
	\emptyvarerr{\newchapter}{#1}{Titulo no definido}
	\newpage
	\stepcounter{section}
	\phantomsection
	\needspace{3\baselineskip}
	\vspace* {3cm}
	\noindent {\huge{\textbf{\nomchapter\ \thesection}}} \\
	\vspace* {0.5cm} \\
	\noindent {\Huge{\textbf{#1}}} \\
	\vspace {0.5cm} \\
	\addcontentsline{toc}{section}{\protect\numberline{\thesection}#1}
	\markboth{#1}{}
}
