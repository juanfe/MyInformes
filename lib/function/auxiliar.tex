% Template:     Template auxiliar LaTeX
% Documento:    Funciones exclusivas Template-Auxiliar
% Versión:      --
% Codificación: UTF-8
%
% Autor: Pablo Pizarro R.
%        Facultad de Ciencias Físicas y Matemáticas
%        Universidad de Chile
%        pablo.pizarro@ing.uchile.cl, ppizarror.com
%
% Sitio web del proyecto: [http://ppizarror.com/Template-Auxiliares/]
% Licencia: MIT           [https://opensource.org/licenses/MIT]

\newcommand{\newquestion}[1]{
	% Insertar nuevo título de pregunta
	%	#1	Título
	\emptyvarerr{\newquestion}{#1}{Titulo pregunta no definido}
	\sectionanum{#1}
}

\newenvironment{references}{
	% Crea una sección de referencias solo para bibtex
	\ifthenelse{\equal{\stylecitereferences}{bibtex}}{
	}{
		\throwerror{}{Solo se puede usar entorno references con estilo citas \noexpand\stylecitereferences=bibtex}
	}
	\begingroup
	% Se configura las referencias como una sección
	\ifthenelse{\equal{\referencenumsection}{true}}{
		\section{\namereferences}
	}{
		\sectionanum{\namereferences}
	}
	\renewcommand{\section}[2]{}
	\begin{thebibliography}{99}
	}
	{ 
	\end{thebibliography}
	\endgroup
}

% Crea una sección de anexos
\newenvironment{anexo}{
	\begingroup
	\begin{appendices}
		\counterwithin{equation}{section}
		\counterwithin{figure}{section}
		\counterwithin{lstlisting}{section}
		\counterwithin{table}{section}
	}
	{
	\end{appendices}
	\endgroup
}