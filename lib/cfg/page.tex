% Template:     Informe/Reporte LaTeX
% Documento:    Configuración de página
% Versión:      DEV
% Codificación: UTF-8
%
% Autor: Pablo Pizarro R. @ppizarror
%        Facultad de Ciencias Físicas y Matemáticas
%        Universidad de Chile
%        pablo.pizarro@ing.uchile.cl, ppizarror.com
%
% Manual template: [https://latex.ppizarror.com/Template-Informe/]
% Licencia MIT:    [https://opensource.org/licenses/MIT/]

% -----------------------------------------------------------------------------
% Numeración de páginas
% -----------------------------------------------------------------------------
\newpage
\ifthenelse{\equal{\predocuseromannumber}{true}}{
	\ifthenelse{\equal{\romanpageuppercase}{true}}{
		\pagenumbering{Roman}
	}{
		\pagenumbering{roman}
	}}{
	\pagenumbering{arabic}
}
\setcounter{page}{1}
\setcounter{footnote}{0}

% -----------------------------------------------------------------------------
% Márgenes de páginas y tablas
% -----------------------------------------------------------------------------
\setpagemargincm{\pagemarginleft}{\pagemargintop}{\pagemarginright}{\pagemarginbottom}
\def\arraystretch {\tablepaddingv} % Se ajusta el padding vertical de las tablas
\setlength{\tabcolsep}{\tablepaddingh em} % Se ajusta el padding horizontal de las tablas

% -----------------------------------------------------------------------------
% Se define el punto decimal
% -----------------------------------------------------------------------------
\ifthenelse{\equal{\pointdecimal}{true}}{
	\decimalpoint}{
}

% -----------------------------------------------------------------------------
% Definición de nombres de objetos
% -----------------------------------------------------------------------------
\renewcommand{\appendixname}{\nomltappendixsection} % Nombre del anexo (título)
\renewcommand{\appendixpagename}{\nameappendixsection} % Nombre del anexo en índice
\renewcommand{\appendixtocname}{\nameappendixsection} % Nombre del anexo en índice
\renewcommand{\contentsname}{\nomltcont} % Nombre del índice
\renewcommand{\figurename}{\nomltwfigure} % Nombre de la leyenda de las fig.
\renewcommand{\listfigurename}{\nomltfigure} % Nombre del índice de figuras
\renewcommand{\listtablename}{\nomlttable} % Nombre del índice de tablas
\renewcommand{\lstlistingname}{\nomltwsrc} % Nombre leyenda del código fuente
\renewcommand{\lstlistlistingname}{\nomltsrc} % Nombre índice código fuente
\renewcommand{\refname}{\namereferences} % Nombre de las referencias
\renewcommand{\tablename}{\nomltwtable} % Nombre de la leyenda de tablas

% -----------------------------------------------------------------------------
% Estilo de títulos
% -----------------------------------------------------------------------------
\sectionfont{\color{\titlecolor} \fontsizetitle \styletitle \selectfont}
\subsectionfont{\color{\subtitlecolor} \fontsizesubtitle \stylesubtitle \selectfont}
\subsubsectionfont{\color{\subsubtitlecolor} \fontsizesubsubtitle \stylesubsubtitle \selectfont}
\titleformat{\subsubsubsection}{\color{\ssstitlecolor} \normalfont \fontsizessstitle \stylessstitle}{\thesubsubsubsection}{1em}{}
\titlespacing*{\subsubsubsection}{0pt}{3.25ex plus 1ex minus .2ex}{1.5ex plus .2ex}

% -----------------------------------------------------------------------------
% Se crean los header-footer
% -----------------------------------------------------------------------------
\ifthenelse{\equal{\hfstyle}{style1}}{
	\pagestyle{fancy} \fancyhf{}
	\ifthenelse{\equal{\disablehfrightmark}{false}}{
		\fancyhead[L]{\nouppercase{\rightmark}}
	}{}
	\fancyhead[R]{\small \thepage}
	\fancyfoot[L]{
		\begin{minipage}[t]{\hftitlewidth\linewidth}
			\begin{flushleft}
				\small \textit{\titulodelinforme}
			\end{flushleft}
		\end{minipage}
	}
	\fancyfoot[R]{
		\begin{minipage}[t]{\hfcoursewidth\linewidth}
			\begin{flushright}
				\small \textit{\codigodelcurso \nombredelcurso}
			\end{flushright}
		\end{minipage}
	}
	\renewcommand{\headrulewidth}{0.5pt}
	\renewcommand{\footrulewidth}{0.5pt}
	\renewcommand{\sectionmark}[1]{\markboth{#1}{}}
}{
\ifthenelse{\equal{\hfstyle}{style2}}{
	\pagestyle{fancy} \fancyhf{}
	\ifthenelse{\equal{\disablehfrightmark}{false}}{
		\fancyhead[L]{\nouppercase{\rightmark}}
	}{}
	\fancyhead[R]{\small \thepage}
	\fancyfoot[L]{
		\begin{minipage}[t]{\hftitlewidth\linewidth}
			\begin{flushleft}
				\small \textit{\titulodelinforme}
			\end{flushleft}
		\end{minipage}
	}
	\fancyfoot[R]{
		\begin{minipage}[t]{\hfcoursewidth\linewidth}
			\begin{flushright}
				\small \textit{\codigodelcurso \nombredelcurso}
			\end{flushright}
		\end{minipage}
	}
	\renewcommand{\headrulewidth}{0.5pt}
	\renewcommand{\footrulewidth}{0pt}
	\renewcommand{\sectionmark}[1]{\markboth{#1}{}}
}{
\ifthenelse{\equal{\hfstyle}{style3}}{
	\pagestyle{fancy} \fancyhf{}
	\fancyhead[L]{
		\begin{minipage}[t]{\hftitlewidth\linewidth}
			\begin{flushleft}
				\small \textit{\codigodelcurso \nombredelcurso}
			\end{flushleft}
		\end{minipage}
	}
	\fancyhead[R]{
		\includegraphics[width=1.2cm]{\imagendepartamento}
		\vspace{-0.15cm}
	}
	\fancyfoot[C]{\thepage}
	\renewcommand{\headrulewidth}{0.5pt}
	\renewcommand{\footrulewidth}{0pt}
}{
\ifthenelse{\equal{\hfstyle}{style4}}{
	\pagestyle{fancy} \fancyhf{}
	\ifthenelse{\equal{\disablehfrightmark}{false}}{
		\fancyhead[L]{\nouppercase{\rightmark}}
	}{}
	\fancyhead[R]{}
	\fancyfoot[C]{\small \thepage}
	\renewcommand{\headrulewidth}{0.5pt}
	\renewcommand{\footrulewidth}{0pt}
	\renewcommand{\sectionmark}[1]{\markboth{#1}{}}
}{
\ifthenelse{\equal{\hfstyle}{style5}}{
	\pagestyle{fancy} \fancyhf{}
	\fancyhead[L]{
		\begin{minipage}[t]{\hfcoursewidth\linewidth}
			\begin{flushleft}
				\codigodelcurso \nombredelcurso
			\end{flushleft}
		\end{minipage}
	}
	\ifthenelse{\equal{\disablehfrightmark}{false}}{
		\fancyhead[R]{
			\begin{minipage}[t]{\hftitlewidth\linewidth}
				\begin{flushright}
					\nouppercase{\rightmark}
				\end{flushright}
			\end{minipage}
		}
	}{}
	\fancyfoot[L]{\departamentouniversidad, \nombreuniversidad}
	\fancyfoot[R]{\small \thepage}
	\renewcommand{\headrulewidth}{0pt}
	\renewcommand{\footrulewidth}{0pt}
	\renewcommand{\sectionmark}[1]{\markboth{#1}{}}
}{
\ifthenelse{\equal{\hfstyle}{style6}}{
	\pagestyle{fancy} \fancyhf{}
	\fancyfoot[L]{\departamentouniversidad}
	\fancyfoot[C]{\thepage}
	\fancyfoot[R]{\nombreuniversidad}
	\renewcommand{\headrulewidth}{0pt}
	\renewcommand{\footrulewidth}{0pt}
	\setlength{\headheight}{49pt}
}{
\ifthenelse{\equal{\hfstyle}{style7}}{
	\pagestyle{fancy} \fancyhf{}
	\fancyfoot[C]{\thepage}
	\renewcommand{\headrulewidth}{0pt}
	\renewcommand{\footrulewidth}{0pt}
	\setlength{\headheight}{49pt}
}{
\ifthenelse{\equal{\hfstyle}{style8}}{
	\pagestyle{fancy} \fancyhf{}
	\fancyfoot[R]{\thepage}
	\renewcommand{\headrulewidth}{0pt}
	\renewcommand{\footrulewidth}{0pt}
	\setlength{\headheight}{49pt}
}{
\ifthenelse{\equal{\hfstyle}{style9}}{
	\pagestyle{fancy} \fancyhf{}
	\ifthenelse{\equal{\disablehfrightmark}{false}}{
		\fancyhead[L]{\nouppercase{\rightmark}}
	}{}
	\fancyhead[R]{}
	\fancyfoot[L]{\small \textit{\titulodelinforme}}
	\fancyfoot[R]{\small \thepage}
	\renewcommand{\headrulewidth}{0.5pt}
	\renewcommand{\footrulewidth}{0.5pt}
	\renewcommand{\sectionmark}[1]{\markboth{#1}{}}
}{
\ifthenelse{\equal{\hfstyle}{style10}}{
	\pagestyle{fancy} \fancyhf{}
	\ifthenelse{\equal{\disablehfrightmark}{false}}{
		\fancyhead[L]{
			\begin{minipage}[t]{\hftitlewidth\linewidth}
				\begin{flushleft}
					\nouppercase{\rightmark}
				\end{flushleft}
			\end{minipage}
		}
	}{}
	\fancyhead[R]{
		\begin{minipage}[t]{\hfcoursewidth\linewidth}
			\begin{flushright}
				\small \textit{\titulodelinforme}
			\end{flushright}
		\end{minipage}
	}
	\fancyfoot[L]{}
	\fancyfoot[R]{\small \thepage}
	\renewcommand{\headrulewidth}{0.5pt}
	\renewcommand{\footrulewidth}{0.5pt}
	\renewcommand{\sectionmark}[1]{\markboth{#1}{}}
}{
\ifthenelse{\equal{\hfstyle}{style11}}{ % Similar a 1
	\pagestyle{fancy} \fancyhf{}
	\ifthenelse{\equal{\disablehfrightmark}{false}}{
		\fancyhead[L]{\nouppercase{\rightmark}}
	}{}
	\fancyhead[R]{\small \thepage \nomnpageof \pageref{LastPage}}
	\fancyfoot[L]{
		\begin{minipage}[t]{\hftitlewidth\linewidth}
			\begin{flushleft}
				\small \textit{\titulodelinforme}
			\end{flushleft}
		\end{minipage}
	}
	\fancyfoot[R]{
		\begin{minipage}[t]{\hfcoursewidth\linewidth}
			\begin{flushright}
				\small \textit{\codigodelcurso \nombredelcurso}
			\end{flushright}
		\end{minipage}
	}
	\renewcommand{\headrulewidth}{0.5pt}
	\renewcommand{\footrulewidth}{0.5pt}
	\renewcommand{\sectionmark}[1]{\markboth{#1}{}}
}{
\ifthenelse{\equal{\hfstyle}{style12}}{ % Similar a 6
	\pagestyle{fancy} \fancyhf{}
	\fancyfoot[L]{\departamentouniversidad}
	\fancyfoot[C]{\thepage \nomnpageof \pageref{LastPage}}
	\fancyfoot[R]{\nombreuniversidad}
	\renewcommand{\headrulewidth}{0pt}
	\renewcommand{\footrulewidth}{0pt}
	\setlength{\headheight}{49pt}
}{
\ifthenelse{\equal{\hfstyle}{style13}}{ % Similar a 3
	\pagestyle{fancy} \fancyhf{}
	\fancyhead[L]{
		\begin{minipage}[t]{\hftitlewidth\linewidth}
			\begin{flushleft}
				\small \textit{\codigodelcurso \nombredelcurso}
			\end{flushleft}
		\end{minipage}
	}
	\fancyhead[R]{
		\includegraphics[width=1.2cm]{\imagendepartamento}
		\vspace{-0.15cm}
	}
	\fancyfoot[C]{\thepage \nomnpageof \pageref{LastPage}}
	\renewcommand{\headrulewidth}{0.5pt}
	\renewcommand{\footrulewidth}{0pt}
}{
\ifthenelse{\equal{\hfstyle}{style14}}{ % Similar a 4
	\pagestyle{fancy} \fancyhf{}
	\ifthenelse{\equal{\disablehfrightmark}{false}}{
		\fancyhead[L]{\nouppercase{\rightmark}}
	}{}
	\fancyhead[R]{}
	\fancyfoot[C]{\small \thepage \nomnpageof \pageref{LastPage}}
	\renewcommand{\headrulewidth}{0.5pt}
	\renewcommand{\footrulewidth}{0pt}
	\renewcommand{\sectionmark}[1]{\markboth{#1}{}}
}{
\ifthenelse{\equal{\hfstyle}{style15}}{ % Similar a 1
	\pagestyle{fancy} \fancyhf{}
	\ifthenelse{\equal{\disablehfrightmark}{false}}{
		\fancyhead[L]{\nouppercase{\rightmark}}
	}{}
	\fancyhead[R]{}
	\fancyfoot[L]{
		\small \codigodelcurso \nombredelcurso
	}
	\fancyfoot[R]{
		\small \thepage
	}
	\renewcommand{\headrulewidth}{0.5pt}
	\renewcommand{\footrulewidth}{0.5pt}
	\renewcommand{\sectionmark}[1]{\markboth{#1}{}}
}{
	\throwbadconfigondoc{Estilo de header-footer incorrecto}{\hfstyle}{style1 .. style15}}}}}}}}}}}}}}}
}

% -----------------------------------------------------------------------------
% Muestra los números de línea
% -----------------------------------------------------------------------------
\ifthenelse{\equal{\showlinenumbers}{true}}{
	\linenumbers}{
}

% END