% Template:     Informe/Reporte LaTeX
% Documento:    Configuraciones finales
% Versión:      DEV
% Codificación: UTF-8
%
% Autor: Pablo Pizarro R. @ppizarror
%        Facultad de Ciencias Físicas y Matemáticas
%        Universidad de Chile
%        pablo.pizarro@ing.uchile.cl, ppizarror.com
%
% Manual template: [https://latex.ppizarror.com/Template-Informe/]
% Licencia MIT:    [https://opensource.org/licenses/MIT/]

% -----------------------------------------------------------------------------
% Se reestablecen headers y footers
% -----------------------------------------------------------------------------
\markboth{}{}
\newpage

\ifthenelse{\equal{\disablehfrightmark}{false}}{
	\ifthenelse{\equal{\hfstyle}{style1}}{
		\fancyhead[L]{\nouppercase{\leftmark}}}{
	}
	\ifthenelse{\equal{\hfstyle}{style2}}{
		\fancyhead[L]{\nouppercase{\leftmark}}}{
	}
	\ifthenelse{\equal{\hfstyle}{style4}}{
		\fancyhead[L]{\nouppercase{\leftmark}}}{
	}
	\ifthenelse{\equal{\hfstyle}{style5}}{
		\ifthenelse{\equal{\hfwidthwrap}{true}}{
			\fancyhead[R]{
				\begin{minipage}[t]{\hfwidthtitle\linewidth}
					\begin{flushright}
						\nouppercase{\leftmark}
					\end{flushright}
				\end{minipage}
			}
		}{
			\fancyhead[R]{\nouppercase{\leftmark}}
		}}{
	}
	\ifthenelse{\equal{\hfstyle}{style9}}{
		\fancyhead[L]{\nouppercase{\leftmark}}}{
	}
	\ifthenelse{\equal{\hfstyle}{style10}}{
		\ifthenelse{\equal{\hfwidthwrap}{true}}{
			\fancyhead[L]{
				\begin{minipage}[t]{\hfwidthtitle\linewidth}
					\begin{flushleft}
						\nouppercase{\leftmark}
					\end{flushleft}
				\end{minipage}
			}
		}{
			\fancyhead[L]{\nouppercase{\leftmark}}
		}}{
	}
	\ifthenelse{\equal{\hfstyle}{style11}}{ % Similar a 1
		\fancyhead[L]{\nouppercase{\leftmark}}}{
	}
	\ifthenelse{\equal{\hfstyle}{style14}}{ % Similar a 4
		\fancyhead[L]{\nouppercase{\leftmark}}}{
	}
	\ifthenelse{\equal{\hfstyle}{style15}}{ % Similar a 1
		\fancyhead[L]{\nouppercase{\leftmark}}}{
	}
	}{
}

% -----------------------------------------------------------------------------
% Estilo de títulos - reestablece estilos por el índice
% -----------------------------------------------------------------------------
\sectionfont{\color{\titlecolor} \fontsizetitle \styletitle \selectfont}
\subsectionfont{\color{\subtitlecolor} \fontsizesubtitle \stylesubtitle \selectfont}
\subsubsectionfont{\color{\subsubtitlecolor} \fontsizesubsubtitle \stylesubsubtitle \selectfont}
\titleformat{\subsubsubsection}{\color{\ssstitlecolor} \normalfont \fontsizessstitle \stylessstitle}{\thesubsubsubsection}{1em}{}
\titlespacing*{\subsubsubsection}{0pt}{3.25ex plus 1ex minus .2ex}{1.5ex plus .2ex}

% -----------------------------------------------------------------------------
% Crea funciones para numerar objetos
% -----------------------------------------------------------------------------

% Numeración de la sección en los objetos CÓDIGO FUENTE
\ifthenelse{\equal{\showsectioncaptioncode}{none}}{
	\def\sectionobjectnumcode {}
}{
\ifthenelse{\equal{\showsectioncaptioncode}{sec}}{
	\def\sectionobjectnumcode {\thesection\sectioncaptiondelimiter}
}{
\ifthenelse{\equal{\showsectioncaptioncode}{ssec}}{
	\def\sectionobjectnumcode {\thesubsection\sectioncaptiondelimiter}
}{
\ifthenelse{\equal{\showsectioncaptioncode}{sssec}}{
	\def\sectionobjectnumcode {\thesubsubsection\sectioncaptiondelimiter}
}{
\ifthenelse{\equal{\showsectioncaptioncode}{ssssec}}{
	\def\sectionobjectnumcode {\thesubsubsubsection\sectioncaptiondelimiter}
}{
	\throwbadconfig{Valor configuracion incorrecto}{\showsectioncaptioncode}{none,sec,ssec,sssec,ssssec}
}}}}}

% -----------------------------------------------------------------------------
% Modifica numeración de objetos
% -----------------------------------------------------------------------------

% Figuras, INCLUIR SECCIÓN
\ifthenelse{\equal{\captionnumequation}{arabic}}{
	\renewcommand{\theequation}{\sectionobjectnum\arabic{figure}}
}{
\ifthenelse{\equal{\captionnumequation}{alph}}{
	\renewcommand{\theequation}{\sectionobjectnum\alph{figure}}
}{
\ifthenelse{\equal{\captionnumequation}{Alph}}{
	\renewcommand{\theequation}{\sectionobjectnum\Alph{figure}}
}{
\ifthenelse{\equal{\captionnumequation}{roman}}{
	\renewcommand{\theequation}{\sectionobjectnum\roman{figure}}
}{
\ifthenelse{\equal{\captionnumequation}{Roman}}{
	\renewcommand{\theequation}{\sectionobjectnum\Roman{figure}}
}{
	\throwbadconfig{Tipo numero ecuacion desconocido}{\captionnumequation}{arabic,alph,Alph,roman,Roman}}}}}
}

% Figuras, INCLUIR SECCIÓN
\ifthenelse{\equal{\captionnumfigure}{arabic}}{
	\renewcommand{\thefigure}{\sectionobjectnum\arabic{figure}}
}{
\ifthenelse{\equal{\captionnumfigure}{alph}}{
	\renewcommand{\thefigure}{\sectionobjectnum\alph{figure}}
}{
\ifthenelse{\equal{\captionnumfigure}{Alph}}{
	\renewcommand{\thefigure}{\sectionobjectnum\Alph{figure}}
}{
\ifthenelse{\equal{\captionnumfigure}{roman}}{
	\renewcommand{\thefigure}{\sectionobjectnum\roman{figure}}
}{
\ifthenelse{\equal{\captionnumfigure}{Roman}}{
	\renewcommand{\thefigure}{\sectionobjectnum\Roman{figure}}
}{
	\throwbadconfig{Tipo numero figura desconocido}{\captionnumfigure}{arabic,alph,Alph,roman,Roman}}}}}
}

% Subfiguras, NO USAR SECCIONES YA QUE SON HIJAS DE FIGURA
\ifthenelse{\equal{\captionnumsubfigure}{arabic}}{
	\renewcommand{\thesubfigure}{\arabic{subfigure}}
}{
\ifthenelse{\equal{\captionnumsubfigure}{alph}}{
	\renewcommand{\thesubfigure}{\alph{subfigure}}
}{
\ifthenelse{\equal{\captionnumsubfigure}{Alph}}{
	\renewcommand{\thesubfigure}{\Alph{subfigure}}
}{
\ifthenelse{\equal{\captionnumsubfigure}{roman}}{
	\renewcommand{\thesubfigure}{\roman{subfigure}}
}{
\ifthenelse{\equal{\captionnumsubfigure}{Roman}}{
	\renewcommand{\thesubfigure}{\Roman{subfigure}}
}{
	\throwbadconfig{Tipo numero subfigura desconocido}{\captionnumsubfigure}{arabic,alph,Alph,roman,Roman}}}}}
}

% Tablas, INCLUIR SECCIÓN
\ifthenelse{\equal{\captionnumtable}{arabic}}{
	\renewcommand{\thetable}{\sectionobjectnum\arabic{table}}
}{
\ifthenelse{\equal{\captionnumtable}{alph}}{
	\renewcommand{\thetable}{\sectionobjectnum\alph{table}}
}{
\ifthenelse{\equal{\captionnumtable}{Alph}}{
	\renewcommand{\thetable}{\sectionobjectnum\Alph{table}}
}{
\ifthenelse{\equal{\captionnumtable}{roman}}{
	\renewcommand{\thetable}{\sectionobjectnum\roman{table}}
}{
\ifthenelse{\equal{\captionnumtable}{Roman}}{
	\renewcommand{\thetable}{\sectionobjectnum\Roman{table}}
}{
	\throwbadconfig{Tipo numero tabla desconocido}{\captionnumtable}{arabic,alph,Alph,roman,Roman}}}}}
}

% Subtablas, NO INCLUIR SECCIÓN YA QUE SON HIJAS DE LAS TABLAS
\ifthenelse{\equal{\captionnumsubtable}{arabic}}{
	\renewcommand{\thesubtable}{\arabic{subtable}}
}{
\ifthenelse{\equal{\captionnumsubtable}{alph}}{
	\renewcommand{\thesubtable}{\alph{subtable}}
}{
\ifthenelse{\equal{\captionnumsubtable}{Alph}}{
	\renewcommand{\thesubtable}{\Alph{subtable}}
}{
\ifthenelse{\equal{\captionnumsubtable}{roman}}{
	\renewcommand{\thesubtable}{\roman{subtable}}
}{
\ifthenelse{\equal{\captionnumsubtable}{Roman}}{
	\renewcommand{\thesubtable}{\Roman{subtable}}
}{
	\throwbadconfig{Tipo numero subtabla desconocido}{\captionnumsubtable}{arabic,alph,Alph,roman,Roman}}}}}
}

% Código fuente, INCLUIR SECCIÓN
\ifthenelse{\equal{\captionnumcode}{arabic}}{
	\renewcommand{\thelstlisting}{\sectionobjectnum\arabic{lstlisting}}
}{
\ifthenelse{\equal{\captionnumcode}{alph}}{
	\renewcommand{\thelstlisting}{\sectionobjectnum\alph{lstlisting}}
}{
\ifthenelse{\equal{\captionnumcode}{Alph}}{
	\renewcommand{\thelstlisting}{\sectionobjectnum\Alph{lstlisting}}
}{
\ifthenelse{\equal{\captionnumcode}{roman}}{
	\renewcommand{\thelstlisting}{\sectionobjectnum\roman{lstlisting}}
}{
\ifthenelse{\equal{\captionnumcode}{Roman}}{
	\renewcommand{\thelstlisting}{\sectionobjectnum\Roman{lstlisting}}
}{
	\throwbadconfig{Tipo numero codigo fuente desconocido}{\captionnumcode}{arabic,alph,Alph,roman,Roman}}}}}
}

% -----------------------------------------------------------------------------
% Se reestablecen números de página y secciones
% -----------------------------------------------------------------------------
\ifthenelse{\equal{\predocuseromannumber}{true}}{
	\renewcommand{\thepage}{\arabic{page}}}{
}
\ifthenelse{\equal{\resetpagnumafterindex}{true}}{
	\setcounter{page}{1}}{
}
\setcounter{section}{0}
\setcounter{footnote}{0}

% -----------------------------------------------------------------------------
% Muestra los números de línea
% -----------------------------------------------------------------------------
\ifthenelse{\equal{\showlinenumbers}{true}}{
	\linenumbers}{
}

% END