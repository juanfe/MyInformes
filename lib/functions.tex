% Template:     Informe/Reporte LaTeX
% Documento:    Definición de funciones
% Versión:      3.0.0 (01/04/2017)
% Codificación: UTF-8
%
% Autor: Pablo Pizarro R.
%        Facultad de Ciencias Físicas y Matemáticas.
%        Universidad de Chile.
%        pablo.pizarro@ing.uchile.cl, ppizarror.com
%
% Sitio web del proyecto: [http://ppizarror.com/Template-Informe/]
% Licencia: MIT           [https://opensource.org/licenses/MIT]

\newcommand{\quotes}[1]{
	% Insertar cita
	''#1''
}
\newcommand{\quotesit}[1]{
	% Insertar cita en itálico
	\textit{\quotes{#1}}
}
\newcommand{\newemptypage}{
	% Crea una página vacía
	\newpage\null\thispagestyle{empty}\newpage
	\addtocounter{page}{-1}
}
\newcommand{\setcaptionmargincm}[1]{
	% Cambiar el margen
	\captionsetup{margin=#1cm}
}
\newcommand{\setpagemargincm}[4]{
	% Cambia márgenes de las páginas [cm]
	\newgeometry{left=#1cm, top=#2cm, right=#3cm, bottom=#4cm}
}
\newcommand{\newp}{
	% Inserta nueva línea
	\hbadness=10000 \vspace{\defaultnewlinesize} \par
}
\newcommand{\newpar}[1]{
	% Nuevo párrafo
	\hbadness=10000 #1 \newp
}
\newcommand{\newparnl}[1]{
	% Nuevo párrafo sin nueva línea al final
	#1 \par
}
\newcommand{\lpow}[2]{
	% Insertar sub-índice
	{#1}_{#2}
}
\newcommand{\pow}[2]{
	% Insertar elevado
	{#1}^{#2}
}
\newcommand{\fracpartial}[2]{
	% Fracción de derivadas parciales af/ax
	\frac{\partial #1}{\partial #2}
}
\newcommand{\fracdpartial}[2]{
	% Fracción de derivadas parciales dobles a^2/ax^2
	\frac{{\partial}^{2} #1}{\partial {#2}^{2}}
}
\newcommand{\fracnpartial}[3]{
	% Fracción de derivadas parciales en n a^n/ax^n
	\frac{{\partial}^{#3} #1}{\partial {#2}^{#3}}
}
\newcommand{\fracderivat}[2]{
	% Fracción de derivadas df/dx
	\frac{\text{d} #1}{\text{d} #2}
}
\newcommand{\fracdderivat}[2]{
	% Fracción de derivadas dobles d^2/dx^2
	\frac{{\text{d}}^{2} #1}{\text{d} {#2}^{2}}
}
\newcommand{\fracnderivat}[3]{
	% Fracción de derivadas en n d^n/dx^n
	\frac{{\text{d}}^{#3} #1}{\text{d} {#2}^{#3}}
}
\newcommand{\topequal}[2]{
	% Llave superior de equivalencia
	\overbrace{#1}^{\mathclap{#2}}
}
\newcommand{\underequal}[2]{
	% Llave inferior de equivalencia
	\underbrace{#1}_{\mathclap{#2}}
}
\newcommand{\topsequal}[2]{
	% Rectángulo superior de equivalencia
	\overbracket{#1}^{\mathclap{#2}}
}
\newcommand{\undersequal}[2]{
	% Rectángulo inferior de equivalencia
	\underbracket{#1}_{\mathclap{#2}}
}
\newcommand{\resizeitem}[2]{
	% Crea un resizebox de tamaño textwidth
	\resizebox{#1\textwidth}{!}{#2}
}
\newcommand{\newtitleanum}[1]{
	% Insertar un título sin número
	\addcontentsline{toc}{section}{#1}
	\section*{#1}
	\ifthenelse{\equal{\showheadertitle}{true}}{
		\fancyhead[L]{\nouppercase{#1}}}{}
	\stepcounter{section}
}
\newcommand{\newtitleanumheadless}[1]{
	% Insertar un título sin número sin alterar el header
	\addcontentsline{toc}{section}{#1}
	\section*{#1}
	\stepcounter{section}
}
\newcommand{\newsubtitleanum}[1]{
	% Insertar un subtítulo sin número
	\addcontentsline{toc}{subsection}{#1}
	\subsection*{#1}
	\stepcounter{subsection}
}
\newcommand{\newsubsubtitleanum}[1]{
	% Insertar un sub-subtítulo sin número
	\addcontentsline{toc}{subsubsection}{#1}
	\subsubsection*{#1}
	\stepcounter{subsubsection}
}
\newcommand{\newtitleanumnoi}[1]{
	% Insertar un título sin número sin indexar
	\section*{#1}
	\ifthenelse{\equal{\showheadertitle}{true}}{
		\fancyhead[L]{\nouppercase{#1}}}{}
	\stepcounter{section}
}
\newcommand{\newtitleanumnoiheadless}[1]{
	% Insertar un título sin número sin indexar sin cambiar el header
	\section*{#1}
	\ifthenelse{\equal{\showheadertitle}{true}}{
		\fancyhead[L]{\nouppercase{#1}}}{}
	\stepcounter{section}
}
\newcommand{\newsubtitleanumnoi}[1]{
	% Insertar un subtítulo sin número sin indexar
	\subsection*{#1}
	\stepcounter{subsection}
}
\newcommand{\newsubsubtitleanumnoi}[1]{
	% Insertar un sub-subtítulo sin número sin indexar
	\addcontentsline{toc}{subsubsection}{#1}
	\subsubsection*{#1}
	\stepcounter{subsubsection}
}
\newcommand{\insertequation}[2][]{
	% Insertar una ecuación
	\vspace{-0.1cm}
	\begin{equation}
		\text{#1} #2
	\end{equation}
	\vspace{-0.23cm}
	\par
}
\newcommand{\insertequationcaptioned}[3][]{
	% Insertar una ecuación con leyenda
	\vspace{-0.1cm}
	\begin{equation}
		\text{#1} #2
	\end{equation}
	\begin{center}
		\vspace{-0.15cm}
		\textit{#3} \par
		\vspace{0.05cm}
	\end{center}
}
\newcommand{\insertequationgathered}[2][]{
	% Insertar una ecuación con el ambiente gather
	\vspace{-0.4cm}
	\begin{gather}
		\text{#1} #2
	\end{gather}
	\par
	\vspace{-0.10cm}
}
\newcommand{\insertequationgatheredcaptioned}[3][]{
	% Insertar una ecuación (gather) con leyenda
	\vspace{-0.3cm}
	\begin{gather}
		\text{#1} #2
	\end{gather}
	\begin{center}
		\vspace{-0.15cm}
		\textit{#3} \par
	\end{center}
}
\newcommand{\insertequationalign}[2][]{
	% Insertar una ecuación con el ambiente align
	\vspace{-0.4cm}
	\begin{align}
		\text{#1} #2
	\end{align}
	\par
	\vspace{-0.10cm}
}
\newcommand{\insertequationaligncaptioned}[3][]{
	% Insertar una ecuación (align) con leyenda
	\vspace{-0.1cm}
	\begin{align}
		\text{#1} #2
	\end{align}
	\begin{center}
		\vspace{-0.15cm}
		\textit{#3} \par
	\end{center}
}
\newcommand{\insertimage}[4][]{
	% Insertar una imagen
	\vspace{\defaultmargintopimages}
	\begin{figure}[H]
		\centering
		\includegraphics[#3]{\defaultimagefolder#2}
		\caption{#4 #1}
	\end{figure}
	\vspace{\defaultmarginbottomimages}
}
\newcommand{\insertimageboxed}[4][]{
	% Insertar una imagen con recuadro
	\vspace{\defaultmargintopimages}
	\begin{figure}[H]
		\centering
		\fbox{\includegraphics[#3]{\defaultimagefolder#2}}
		\caption{#4 #1}
	\end{figure}
	\vspace{\defaultmarginbottomimages}
}
\newcommand{\insertimagefixed}[5][]{
	% Insertar una imagen de ancho fijo a la página
	\vspace{\defaultmargintopimages}
	\begin{figure}[H]
		\centering
		\resizebox{#3\textwidth}{!}{
			\includegraphics[#4]{\defaultimagefolder#2}
		}
		\caption{#5 #1}
	\end{figure}
	\vspace{\defaultmarginbottomimages}
}
\newcommand{\insertimageboxedfixed}[5][]{
	% Insertar una imagen recuadrada de ancho fijo
	\vspace{\defaultmargintopimages}
	\begin{figure}[H]
		\centering
		\resizebox{#3\textwidth}{!}{
			\fbox{\includegraphics[#4]{\defaultimagefolder#2}}
		}
		\caption{#5 #1}
	\end{figure}
	\vspace{\defaultmarginbottomimages}
}
\newcommand{\insertdoubleimage}[8][]{
	% Insertar una imagen doble
	\vspace{\defaultmargintopimages}
	\captionsetup{margin=0.45cm}
	\begin{figure}[H] \centering
		\subfloat[#4]{
			\includegraphics[#3]{\defaultimagefolder#2}}
		\hspace{0.2cm}
		\subfloat[#7]{
			\includegraphics[#6]{\defaultimagefolder#5}}
		\setcaptionmargincm{\defaultcaptionmargin}
		\caption{#8 #1}
	\end{figure}
	\setcaptionmargincm{\defaultcaptionmargin}
	\vspace{\defaultmarginbottomimages}
}
\newcommand{\insertdoubleeqimage}[7][]{
	% Insertar una imagen doble, igual propiedades
	\insertdoubleimage[#1]{#2}{#6}{#3}{#4}{#6}{#5}{#7}
}
\newcommand{\inserttripleimage}[8][]{
	% Insertar una imagen triple
	\vspace{\defaultmargintopimages}
	\captionsetup{margin=0.45cm}
	\begin{figure}[H] \centering
		\subfloat[]{
			\includegraphics[#3]{\defaultimagefolder#2}}
		\hspace{0.1cm}
		\subfloat[]{
			\includegraphics[#5]{\defaultimagefolder#4}}
		\hspace{0.1cm}
		\subfloat[]{
			\includegraphics[#7]{\defaultimagefolder#6}}
		\setcaptionmargincm{\defaultcaptionmargin}
		\caption{#8 #1}
	\end{figure}
	\setcaptionmargincm{\defaultcaptionmargin}
	\vspace{\defaultmarginbottomimages}
}
\newcommand{\inserttripleeqimage}[6][]{
	% Insertar una imagen triple, igual propiedades
	\inserttripleimage[#1]{#2}{#5}{#3}{#5}{#4}{#5}{#6}
}
\newcommand{\insertquadimage}[7][]{
	% Insertar una imagen cuádruple, igual propiedades
	\vspace{\defaultmargintopimages}
	\captionsetup{margin=0.45cm}
	\begin{figure}[H] \centering
		\subfloat[]{
			\includegraphics[#6]{\defaultimagefolder#2}}
		\hspace{0.1cm}
		\subfloat[]{
			\includegraphics[#6]{\defaultimagefolder#3}}
		\hspace{0.1cm}
		\subfloat[]{
			\includegraphics[#6]{\defaultimagefolder#4}}
		\hspace{0.1cm}
		\subfloat[]{
			\includegraphics[#6]{\defaultimagefolder#5}}
		\setcaptionmargincm{\defaultcaptionmargin}
		\caption{#7 #1}
	\end{figure}
	\setcaptionmargincm{\defaultcaptionmargin}
	\vspace{\defaultmarginbottomimages}
}
\newcommand{\insertpentaimage}[8][]{
	% Insertar una imagen quíntuple, igual propiedades
	\vspace{\defaultmargintopimages}
	\captionsetup{margin=0.45cm}
	\begin{figure}[H] \centering
		\subfloat[]{
			\includegraphics[#7]{\defaultimagefolder#2}}
		\hspace{0.1cm}
		\subfloat[]{
			\includegraphics[#7]{\defaultimagefolder#3}}
		\hspace{0.1cm}
		\subfloat[]{
			\includegraphics[#7]{\defaultimagefolder#4}}
		\hspace{0.1cm}
		\subfloat[]{
			\includegraphics[#7]{\defaultimagefolder#5}}
		\hspace{0.1cm}
		\subfloat[]{
			\includegraphics[#7]{\defaultimagefolder#6}}
		\setcaptionmargincm{\defaultcaptionmargin}
		\caption{#8 #1}
	\end{figure}
	\setcaptionmargincm{\defaultcaptionmargin}
	\vspace{\defaultmarginbottomimages}
}
\newcommand{\insertimageleft}[5][]{
	% Insertar una imagen a la izquierda
	\begin{wrapfigure}[#5]{l}{#3\textwidth}
		\setcaptionmargincm{0}
		\vspace{\defaultmarginfloatimages}
		\centering
		\includegraphics[width=\linewidth]{\defaultimagefolder#2}
		\caption{#4 #1}
		\setcaptionmargincm{\defaultcaptionmargin}
	\end{wrapfigure}
}
\newcommand{\insertimageright}[5][]{
	% Insertar una imagen a la derecha
	\begin{wrapfigure}[#5]{r}{#3\textwidth}
		\setcaptionmargincm{0}
		\vspace{\defaultmarginfloatimages}
		\centering
		\includegraphics[width=\linewidth]{\defaultimagefolder#2}
		\caption{#4 #1}
		\setcaptionmargincm{\defaultcaptionmargin}
	\end{wrapfigure}
}