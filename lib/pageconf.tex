% Template:     Informe/Reporte LaTeX
% Documento:    Configuración de página
% Versión:      3.4.1 (11/05/2017)
% Codificación: UTF-8
%
% Autor: Pablo Pizarro R.
%        Facultad de Ciencias Físicas y Matemáticas.
%        Universidad de Chile.
%        pablo.pizarro@ing.uchile.cl, ppizarror.com
%
% Sitio web del proyecto: [http://ppizarror.com/Template-Informe/]
% Licencia: MIT           [https://opensource.org/licenses/MIT]

% Se crea la numeración y los márgenes
\newpage \pagenumbering{Roman}
\setcounter{page}{1}
\setcounter{footnote}{1}
\setpagemargincm{\pagemarginleft}{\pagemargintop}
{\pagemarginright}{\pagemarginbottom}
\def\arraystretch{\tablepadding} % Se ajusta el padding de las tablas

% Definición de nombres a objetos
\renewcommand{\sectionmark}[1]{\markboth{#1}{}} % Se modifica el estilo del header
\renewcommand{\listfigurename}{\nomltfigure}  % Nombre del índice de figuras
\renewcommand{\listtablename}{\nomlttable}    % Nombre del índice de tablas
\renewcommand{\contentsname}{\nomltcont}      % Nombre del índice
\renewcommand{\lstlistlistingname}{\nomltsrc} % Nombre índice código fuente
\renewcommand{\tablename}{\nomltwtable}       % Nombre de la leyenda de las tablas
\renewcommand{\figurename}{\nomltwfigure}     % Nombre de la leyenda de las fig.
\renewcommand{\lstlistingname}{\nomltwsrc}    % Nombre leyenda del código fuente
\renewcommand\refname{\namereferences}        % Nombre de las referencias

% Se crean los estílos de página
\pagestyle{fancy} \fancyhf{}
\ifthenelse{\equal{\showheadertitle}{true}}{ % Header izq, nombre sección
	\fancyhead[L]{\nouppercase{\rightmark}}
}{}
\fancyhead[R]{\small \rm \nouppercase{\thepage}} % Header der, número página
\ifthenelse{\equal{\showfooter}{true}}{
	\fancyfoot[L]{
		\small \rm \textit{\nombredelinforme} % Footer izq, título del informe
	}
	\fancyfoot[R]{
		\small \rm \textit{\codigodelcurso \ \nombredelcurso} % Footer der, curso
	}
	\renewcommand{\footrulewidth}{0.5pt}
}{}
\renewcommand{\headrulewidth}{0.5pt}

% Tipo de título para abstract
\sectionfont{
	\typefonttitle \etypefonttitle \selectfont
}