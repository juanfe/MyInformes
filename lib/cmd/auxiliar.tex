% Template:     Template Auxiliares LaTeX
% Documento:    Funciones exclusivas de Template-Auxiliar
% Versión:      --
% Codificación: UTF-8
%
% Autor: Pablo Pizarro R. @ppizarror
%        Facultad de Ciencias Físicas y Matemáticas
%        Universidad de Chile
%        pablo.pizarro@ing.uchile.cl, ppizarror.com
%
% Sitio web del proyecto: [http://ppizarror.com/Template-Auxiliares/]
% Licencia: MIT           [https://opensource.org/licenses/MIT]

% COMPILACION
% - environments.tex -> {references, anexo, columna}

% Insertar nuevo título de pregunta
%	#1	Título
\newcommand{\newquestion}[1]{
	\emptyvarerr{\newquestion}{#1}{Titulo pregunta no definido}
	\sectionanum{#1}
}

% Insertar nuevo título de pregunta encerrado en un recuadro
%	#1	Número pregunta
\newcommand{\newboxquestion}[1]{
	\emptyvarerr{\newquestion}{#1}{Titulo pregunta no definido}
	\phantomsection
	\newp \fbox{\ \textbf{#1}.-\ } \noindent
	\pdfbookmark[1]{#1}{toc}
}

% Crea una sección de imágenes múltiples
%	#1	Label (opcional)
%	#2	Caption
\newenvironment{images}[2][]{
	\def\envimageslabelvar {#1}
	\def\envimagescaptionvar {#2}
	\def\envimagesinitialized {true}
	\vspace{\margintopimages cm}
	\captionsetup{margin=\captionmarginmultimg cm}
	\begin{figure}[H] \centering
		\vspace{-\marginrightmultimage cm}
		\vspace{-\marginrightmultimage cm}
		\vspace{-\marginrightmultimage cm}
	}{
		\setcaptionmargincm{\captionlrmargin}
		\ifx\hfuzz\envimagescaptionvar\hfuzz
		\vspace{\captionlessmarginimage cm}
		\else
		\caption{\envimagescaptionvar\envimageslabelvar}
		\fi
	\end{figure}
	\setcaptionmargincm{\captionlrmargin}
	\vspace{\marginbottomimages cm}
	\def\envimagesinitialized {false}
}
\def\envimagesinitialized {false}

% END