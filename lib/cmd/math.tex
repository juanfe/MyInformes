% Template:     Informe/Reporte LaTeX
% Documento:    Funciones matemáticas
% Versión:      5.4.1 (27/05/2018)
% Codificación: UTF-8
%
% Autor: Pablo Pizarro R. @ppizarror
%        Facultad de Ciencias Físicas y Matemáticas
%        Universidad de Chile
%        pablo.pizarro@ing.uchile.cl, ppizarror.com
%
% Manual template: [http://latex.ppizarror.com/Template-Informe/]
% Licencia MIT:    [https://opensource.org/licenses/MIT/]

% Insertar sub-índice, a_b
% 	#1	Elemento inferior (a)
%	#2	Elemento superior (b)
\newcommand{\lpow}[2]{
	\ensuremath{{#1}_{#2}}
}

% Insertar elevado, a^b
% 	#1	Elemento inferior (a)
%	#2	Elemento superior (b)
\newcommand{\pow}[2]{
	\ensuremath{{#1}^{#2}}
}

% Inserta inverso función seno, sin^-1
%	#1	Elemento
\newcommand{\aasin}[1][]{
	\ifx\hfuzz#1\hfuzz
		\ensuremath{\sin^{-1}#1}
	\else
		\ensuremath{{\sin}^{-1}}
	\fi
}

% Inserta inverso función coseno, cos^-1
%	#1	Elemento
\newcommand{\aacos}[1][]{
	\ifx\hfuzz#1\hfuzz
		\ensuremath{\cos^{-1}#1}
	\else
		\ensuremath{\cos^{-1}}
	\fi
}

% Inserta inverso función tangente, tan^-1
%	#1	Elemento
\newcommand{\aatan}[1][]{
	\ifx\hfuzz#1\hfuzz
		\ensuremath{\tan^{-1}#1}
	\else
		\ensuremath{\tan^{-1}}
	\fi
}

% Inserta inverso función cosecante, csc^-1
%	#1	Elemento
\newcommand{\aacsc}[1][]{
	\ifx\hfuzz#1\hfuzz
		\ensuremath{\csc^{-1}#1}
	\else
		\ensuremath{\csc^{-1}}
	\fi
}

% Inserta inverso función secante, sec^-1
%	#1	Elemento
\newcommand{\aasec}[1][]{
	\ifx\hfuzz#1\hfuzz
		\ensuremath{\sec^{-1}#1}
	\else
		\ensuremath{\sec^{-1}}
	\fi
}

% Inserta inverso función cotangente, cot^-1
%	#1	Elemento
\newcommand{\aacot}[1][]{
	\ifx\hfuzz#1\hfuzz
		\ensuremath{\cot^{-1}#1}
	\else
		\ensuremath{\cot^{-1}}
	\fi
}

% Fracción de derivadas parciales af/ax
% 	#1	Función a derivar (f)
%	#2	Variable a derivar (x)
\newcommand{\fracpartial}[2]{
	\ensuremath{\pdv{#1}{#2}}
}

% Fracción de derivadas parciales dobles a^2f/ax^2
% 	#1	Función a derivar (f)
%	#2	Variable a derivar (x)
\newcommand{\fracdpartial}[2]{
	\ensuremath{\pdv[2]{#1}{#2}}
}

% Fracción de derivadas parciales en n, a^nf/ax^n
% 	#1	Función a derivar (f)
%	#2	Variable a derivar (x)
%	#3	Orden (n)
\newcommand{\fracnpartial}[3]{
	\ensuremath{\pdv[#3]{#1}{#2}}
}

% Fracción de derivadas df/dx
% 	#1	Función a derivar (f)
%	#2	Variable a derivar (x)
\newcommand{\fracderivat}[2]{
	\ensuremath{\dv{#1}{#2}}
}

% Fracción de derivadas dobles d^2/dx^2
% 	#1	Función a derivar (f)
%	#2	Variable a derivar (x)
\newcommand{\fracdderivat}[2]{
	\ensuremath{\dv[2]{#1}{#2}}
}

% Fracción de derivadas en n d^nf/dx^n
% 	#1	Función a derivar (f)
%	#2	Variable a derivar (x)
%	#3	Orden de la derivada (n)
\newcommand{\fracnderivat}[3]{
	\ensuremath{\dv[#3]{#1}{#2}}
}

% Llave superior de equivalencia
% 	#1	Elemento a igualar
%	#2	Igualdad
\newcommand{\topequal}[2]{
	\ensuremath{\overbrace{#1}^{\mathclap{#2}}}
}

% Llave inferior de equivalencia
% 	#1	Elemento a igualar
%	#2	Igualdad
\newcommand{\underequal}[2]{
	\ensuremath{\underbrace{#1}_{\mathclap{#2}}}
}

% Rectángulo superior de equivalencia
% 	#1	Elemento a igualar
%	#2	Igualdad
\newcommand{\topsequal}[2]{
	\ensuremath{\overbracket{#1}^{\mathclap{#2}}}
}

% Rectángulo inferior de equivalencia
% 	#1	Elemento a igualar
%	#2	Igualdad
\newcommand{\undersequal}[2]{
	\ensuremath{\underbracket{#1}_{\mathclap{#2}}}
}
