% Template:     Informe/Reporte LaTeX
% Documento:    Funciones para insertar elementos
% Versión:      DEV
% Codificación: UTF-8
%
% Autor: Pablo Pizarro R. @ppizarror
%        Facultad de Ciencias Físicas y Matemáticas
%        Universidad de Chile
%        pablo.pizarro@ing.uchile.cl, ppizarror.com
%
% Manual template: [https://latex.ppizarror.com/Template-Informe/]
% Licencia MIT:    [https://opensource.org/licenses/MIT/]

% Insertar párrafo
\newcommand{\newp}{\hbadness=10000 \vspace{\defaultnewlinesize pt} \par}

% Insertar párrafo
% 	#1	Párrafo
\newcommand{\newpar}[1]{\hbadness=10000 #1 \newp}

% Insertar párrafo sin nueva línea al final
% 	#1	Párrafo
\newcommand{\newparnl}[1]{#1 \par}

% Redimensiona un ítem
% 	#1	Tamaño del nuevo objeto (En textwidth)
%	#2	Objeto a redimensionar
\newcommand{\itemresize}[2]{
	\emptyvarerr{\itemresize}{#1}{Tamano del nuevo objeto no definido}
	\emptyvarerr{\itemresize}{#2}{Objeto a redimensionar no definido}
	\resizebox{#1\textwidth}{!}{#2}
}

% Crea una página vacía
\newcommand{\insertemptypage}{
	\newpage
	\setcounter{templatepagecounter}{\thepage}
	\pagenumbering{gobble}
	\null
	\thispagestyle{empty}
	\newpage
	\pagenumbering{arabic}
	\setcounter{page}{\thetemplatepagecounter}
}

% Añade un archivo pdf con el header
%	#1	Parámetros (opcional)
%	#2	Nombre del archivo pdf
\newcommand{\includehfpdf}[2][]{
	\includepdf[pagecommand={\pagestyle{fancy}},#1]{#2}
}

% Añade un archivo pdf con el header
%	#1	Parámetros (opcional)
%	#2	Nombre del archivo pdf
\newcommand{\includefullhfpdf}[2][]{
	\includepdf[pages=-,pagecommand={\pagestyle{fancy}},#1]{#2}
}

% Inserta un texto entre comillas
%	#1 	Texto
\newcommand{\quotes}[1]{\enquote*{#1}}

% Inserta una cita con texto elevado
%	#1	Cita
\newcommand{\scite}[1]{\textsuperscript{\cite{#1}}}

% Inserta un texto con el formato de enlace
% 	#1 	Enlace
\newcommand{\hreftext}[1]{
	\ifthenelse{\equal{\fonturl}{same}}{#1}{
	\ifthenelse{\equal{\fonturl}{tt}}{\texttt{#1}}{
	\ifthenelse{\equal{\fonturl}{rm}}{\textrm{#1}}{
	\ifthenelse{\equal{\fonturl}{sf}}{\textsf{#1}}{	
	}}}}
}

% Parcha href para incluir el estilo
% \let\oldhref=\href
% \renewcommand{\href}[2]{
% 	\oldhref{#1}{\insertHREFtext{#2}}
% }

% Inserta un email con un link cliqueable
%	#1 	Dirección email
\newcommand{\insertemail}[1]{\href{mailto:#1}{\hreftext{#1}}}

% Inserta un teléfono celular
%	#1	Teléfono celular
\newcommand{\insertphone}[1]{\href{tel:#1}{\hreftext{#1}}}

% Reinicia el número de ecuaciones
\newcommand{\restartequation}{\setcounter{equation}{0}}

% Desactiva el margen de las leyendas
\newcommand{\disablecaptionmargin}{\setcaptionmargincm{0}}

% Reinicia el margen de las leyendas
\newcommand{\resetcaptionmargin}{\setcaptionmargincm{\captionlrmargin}}

% Modifica el color de las tablas
%	#1	Posición inicial del inicio de colores
\newcommand{\settablerowcolors}[1]{
	\emptyvarerr{\settablerowcolors}{#1}{Posicion de fila no definida}
	
	% Usa colores normales
	\ifthenelse{\equal{\GLOBALtablerowcolorswitch}{false}}{
		\ifthenelse{\equal{\tablerowfirstcolor}{none}}{
			\ifthenelse{\equal{\tablerowsecondcolor}{none}}{
				\rowcolors{#1}{}{}
			}{
				\rowcolors{#1}{\tablerowsecondcolor}{}
			}
		}{
			\ifthenelse{\equal{\tablerowsecondcolor}{none}}{
				\rowcolors{#1}{}{\tablerowfirstcolor}
			}{
				\rowcolors{#1}{\tablerowsecondcolor}{\tablerowfirstcolor}
			}
		}
	% Usa colores alternados
	}{
		\ifthenelse{\equal{\tablerowfirstcolor}{none}}{
			\ifthenelse{\equal{\tablerowsecondcolor}{none}}{
				\rowcolors{#1}{}{}
			}{
				\rowcolors{#1}{}{\tablerowsecondcolor}
			}
		}{
			\ifthenelse{\equal{\tablerowsecondcolor}{none}}{
				\rowcolors{#1}{\tablerowfirstcolor}{}
			}{
				\rowcolors{#1}{\tablerowfirstcolor}{\tablerowsecondcolor}
			}
		}
	}

	% Actualiza el índice previo
	\def\GLOBALtablerowcolorindex{#1}
}

% Modifica el color de las tablas
%	#1	Posición inicial del inicio de colores
\newcommand{\settablerowcolorslast}{
	% Usa colores normales
	\ifthenelse{\equal{\GLOBALtablerowcolorswitch}{false}}{
		\ifthenelse{\equal{\tablerowfirstcolor}{none}}{
			\ifthenelse{\equal{\tablerowsecondcolor}{none}}{
				\rowcolors{\GLOBALtablerowcolorindex}{}{}
			}{
				\rowcolors{\GLOBALtablerowcolorindex}{\tablerowsecondcolor}{}
			}
		}{
			\ifthenelse{\equal{\tablerowsecondcolor}{none}}{
				\rowcolors{\GLOBALtablerowcolorindex}{}{\tablerowfirstcolor}
			}{
				\rowcolors{\GLOBALtablerowcolorindex}{\tablerowsecondcolor}{\tablerowfirstcolor}
			}
		}
	% Usa colores alternados
	}{
		\ifthenelse{\equal{\tablerowfirstcolor}{none}}{
			\ifthenelse{\equal{\tablerowsecondcolor}{none}}{
				\rowcolors{\GLOBALtablerowcolorindex}{}{}
			}{
				\rowcolors{\GLOBALtablerowcolorindex}{}{\tablerowsecondcolor}
			}
		}{
			\ifthenelse{\equal{\tablerowsecondcolor}{none}}{
				\rowcolors{\GLOBALtablerowcolorindex}{\tablerowfirstcolor}{}
			}{
				\rowcolors{\GLOBALtablerowcolorindex}{\tablerowfirstcolor}{\tablerowsecondcolor}
			}
		}
	}
}

% Activa el color de las filas de las tablas
%	#1	Posición inicial del inicio de colores
\newcommand{\enabletablerowcolor}[1][]{
	\ifx\hfuzz#1\hfuzz
		\settablerowcolors{2}
	\else
		\settablerowcolors{#1}
	\fi
}

% Desactiva el color de las filas de las tablas
\newcommand{\disabletablerowcolor}{\rowcolors{2}{}{}}

% Alterna los colores de las filas de las tablas
\newcommand{\switchtablerowcolors}{
	\ifthenelse{\equal{\GLOBALtablerowcolorswitch}{false}}{
		\def\GLOBALtablerowcolorswitch{true}
	}{
		\def\GLOBALtablerowcolorswitch{false}
	}
	\settablerowcolorslast
}

% END