% Template:     Informe/Reporte LaTeX
% Documento:    Funciones para insertar imágenes
% Versión:      6.0.5 (29/10/2018)
% Codificación: UTF-8
%
% Autor: Pablo Pizarro R. @ppizarror
%        Facultad de Ciencias Físicas y Matemáticas
%        Universidad de Chile
%        pablo.pizarro@ing.uchile.cl, ppizarror.com
%
% Manual template: [https://latex.ppizarror.com/Template-Informe/]
% Licencia MIT:    [https://opensource.org/licenses/MIT/]

% Añade una imagen a un env 'image' con borde
%	#1	Dirección de la imagen
%	#2	Parámetros de la imagen
%	#3	Leyenda de la imagen (opcional)
\newcommand{\addimage}[3]{
	\addimageboxed{#1}{#2}{0}{#3}
}

% Añade una imagen a un env 'image' con borde
%	#1	Dirección de la imagen
%	#2	Parámetros de la imagen
%	#3	Ancho de la línea (en pt)
%	#4	Leyenda de la imagen (opcional)
\newcommand{\addimageboxed}[4]{
	\checkonlyonenvimage
	\begingroup
		\setlength{\fboxsep}{0 pt}
		\setlength{\fboxrule}{#3 pt}
		\hspace{\marginrightmultimage cm}
		\subfloat[#4]{
			\fbox{\includegraphics[#2]{#1}}
		}
	\endgroup
}

% Insertar una imagen
% 	#1	Label (opcional)
%	#2	Dirección de la imagen
%	#3	Parámetros de la imagen
%	#4	Leyenda de la imagen (opcional)
\newcommand{\insertimage}[4][]{
	\insertimageboxed[#1]{#2}{#3}{0}{#4}
}

% Insertar una imagen con recuadro
% 	#1	Label (opcional)
%	#2	Dirección de la imagen
%	#3	Parámetros de la imagen
%	#4	Ancho de la línea (en pt)
%	#5	Leyenda de la imagen (opcional)
\newcommand{\insertimageboxed}[5][]{
	\emptyvarerr{\insertimageboxed}{#2}{Direccion de la imagen no definida}
	\emptyvarerr{\insertimageboxed}{#3}{Parametros de la imagen no definidos}
	\emptyvarerr{\insertimageboxed}{#4}{Ancho de la linea no definido}
	\checkoutsideenvimage
	\vspace{\margintopimages cm}
	\begin{figure}[H]
		\begingroup
			\setlength{\fboxsep}{0 pt}
			\setlength{\fboxrule}{#4 pt}
			\centering
			\fbox{\includegraphics[#3]{#2}}
		\endgroup
		\ifx\hfuzz#5\hfuzz
			\vspace{\captionlessmarginimage cm}
		\else
			\hspace{0cm}\caption{#5 #1}
		\fi
	\end{figure}
	\vspace{\marginbottomimages cm}
}

% Insertar una imagen a la izquierda, escalada, ancho fijo
% 	#1	Label (opcional)
%	#2	Dirección de la imagen
%	#3	Ancho de la imagen (en textwidth)
%	#4	Leyenda de la imagen (opcional)
\newcommand{\insertimageleft}[4][]{
	\insertimageleftboxed[#1]{#2}{#3}{0}{#4}
}

% Insertar una imagen a la izquierda, escalada, ancho fijo
% 	#1	Label (opcional)
%	#2	Dirección de la imagen
%	#3	Ancho de la imagen (en textwidth)
%	#4	Ancho de la línea (en pt)
%	#5	Leyenda de la imagen (opcional)
\newcommand{\insertimageleftboxed}[5][]{
	\emptyvarerr{\insertimageleftboxed}{#2}{Direccion de la imagen no definida}
	\emptyvarerr{\insertimageleftboxed}{#3}{Ancho de la imagen no definido}
	\emptyvarerr{\insertimageleftboxed}{#4}{Ancho de la linea no definido}
	\checkoutsideenvimage
	~
	\vspace{-\baselineskip}
	\par
	\begin{wrapfigure}{l}{#3\textwidth}
		\setcaptionmargincm{0}
		\ifthenelse{\equal{\figurecaptiontop}{true}}{}{
			\vspace{\marginfloatimages pt}
		}
		\begingroup
			\setlength{\fboxsep}{0 pt}
			\setlength{\fboxrule}{#4 pt}
			\centering
			\fbox{\includegraphics[width=\linewidth]{#2}}
		\endgroup
		\ifx\hfuzz#5\hfuzz
			\vspace{\captionlessmarginimage cm}
		\else
			\caption{#5 #1}
		\fi
	\end{wrapfigure}
	\setcaptionmargincm{\captionlrmargin}
}

% Insertar una imagen a la izquierda, ajustada en un número de líneas, escalada, ancho fijo
% 	#1	Label (opcional)
%	#2	Dirección de la imagen
%	#3	Ancho de la imagen (en textwidth)
%	#4	Altura en líneas de la imagen
%	#5	Leyenda de la imagen (opcional)
\newcommand{\insertimageleftline}[5][]{
	\insertimageleftlineboxed[#1]{#2}{#3}{0}{#4}{#5}
}

% Insertar una imagen recuadrada a la izquierda, ajustada en un número de líneas, escalada, ancho fijo
% 	#1	Label (opcional)
%	#2	Dirección de la imagen
%	#3	Ancho de la imagen (en textwidth)
%	#4	Ancho de la línea (en pt)
%	#5	Altura en líneas de la imagen
%	#6	Leyenda de la imagen (opcional)
\newcommand{\insertimageleftlineboxed}[6][]{
	\emptyvarerr{\insertimageleftlineboxed}{#2}{Direccion de la imagen no definida}
	\emptyvarerr{\insertimageleftlineboxed}{#3}{Ancho de la imagen no definido}
	\emptyvarerr{\insertimageleftlineboxed}{#4}{Ancho de la linea no definido}
	\emptyvarerr{\insertimageleftlineboxed}{#5}{Altura en lineas de la imagen flotante izquierda no definida}
	\checkoutsideenvimage
	~
	\vspace{-\baselineskip}
	\par
	\begin{wrapfigure}[#5]{l}{#3\textwidth}
		\setcaptionmargincm{0}
		\ifthenelse{\equal{\figurecaptiontop}{true}}{}{
			\vspace{\marginfloatimages pt}}
		\begingroup
			\setlength{\fboxsep}{0 pt}
			\setlength{\fboxrule}{#4 pt}
			\centering
			\fbox{\includegraphics[width=\linewidth]{#2}}
		\endgroup
		\ifx\hfuzz#6\hfuzz
			\vspace{\captionlessmarginimage cm}
		\else
			\caption{#6 #1}
		\fi
	\end{wrapfigure}
	\setcaptionmargincm{\captionlrmargin}
}

% Insertar una imagen a la derecha, escalada, ancho fijo
% 	#1	Label (opcional)
%	#2	Dirección de la imagen
%	#3	Ancho de la imagen (en textwidth)
%	#4	Leyenda de la imagen (opcional)
\newcommand{\insertimageright}[4][]{
	\insertimagerightboxed[#1]{#2}{#3}{0}{#4}
}

% Insertar una imagen recuadrada a la derecha, escalada, ancho fijo
% 	#1	Label (opcional)
%	#2	Dirección de la imagen
%	#3	Ancho de la imagen (en textwidth)
%	#4	Ancho de la línea (en pt)
%	#5	Leyenda de la imagen (opcional)
\newcommand{\insertimagerightboxed}[5][]{
	\emptyvarerr{\insertimagerightboxed}{#2}{Direccion de la imagen no definida}
	\emptyvarerr{\insertimagerightboxed}{#3}{Ancho de la imagen no defindo}
	\emptyvarerr{\insertimagerightboxed}{#4}{Ancho de la linea no definido}
	\checkoutsideenvimage
	~
	\vspace{-\baselineskip}
	\par
	\begin{wrapfigure}{r}{#3\textwidth}
		\setcaptionmargincm{0}
		\ifthenelse{\equal{\figurecaptiontop}{true}}{}{
			\vspace{\marginfloatimages pt}
		}
		\begingroup
			\setlength{\fboxsep}{0 pt}
			\setlength{\fboxrule}{#4 pt}
			\centering
			\fbox{\includegraphics[width=\linewidth]{#2}}
		\endgroup
		\ifx\hfuzz#5\hfuzz
			\vspace{\captionlessmarginimage cm}
		\else
			\caption{#5 #1}
		\fi
	\end{wrapfigure}
	\setcaptionmargincm{\captionlrmargin}
}

% Insertar una imagen a la derecha, ajustada en un número de líneas, escalada, ancho fijo
% 	#1	Label (opcional)
%	#2	Dirección de la imagen
%	#3	Ancho de la imagen (en textwidth)
%	#4	Altura en líneas de la imagen
%	#5	Leyenda de la imagen (opcional)
\newcommand{\insertimagerightline}[5][]{
	\insertimagerightlineboxed[#1]{#2}{#3}{0}{#4}{#5}
}

% Insertar una imagen recuadrada a la derecha, ajustada en un número de líneas, escalada, ancho fijo
% 	#1	Label (opcional)
%	#2	Dirección de la imagen
%	#3	Ancho de la imagen (en textwidth)
%	#4	Ancho de la línea (en pt)
%	#5	Altura en líneas de la imagen
%	#6	Leyenda de la imagen (opcional)
\newcommand{\insertimagerightlineboxed}[6][]{
	\emptyvarerr{\insertimagerightlineboxed}{#2}{Direccion de la imagen no definida}
	\emptyvarerr{\insertimagerightlineboxed}{#3}{Ancho de la imagen no defindo}
	\emptyvarerr{\insertimagerightlineboxed}{#4}{Ancho de la linea no definido}
	\emptyvarerr{\insertimagerightlineboxed}{#5}{Altura en lineas de la imagen flotante derecha no definida}
	\checkoutsideenvimage
	~
	\vspace{-\baselineskip}
	\par
	\begin{wrapfigure}[#5]{r}{#3\textwidth}
		\setcaptionmargincm{0}
		\ifthenelse{\equal{\figurecaptiontop}{true}}{}{
			\vspace{\marginfloatimages pt}
		}
		\begingroup
			\setlength{\fboxsep}{0 pt}
			\setlength{\fboxrule}{#4 pt}
			\centering
			\fbox{\includegraphics[width=\linewidth]{#2}}
		\endgroup
		\ifx\hfuzz#6\hfuzz
			\vspace{\captionlessmarginimage cm}
		\else
			\caption{#6 #1}
		\fi
	\end{wrapfigure}
	\setcaptionmargincm{\captionlrmargin}
}

% Insertar una imagen a la izquierda, propiedades variables
% 	#1	Label (opcional)
%	#2	Dirección de la imagen
%	#3	Ancho del objeto
%	#4	Propiedades de la imagen
%	#5	Leyenda de la imagen (opcional)
\newcommand{\insertimageleftp}[5][]{
	\xspace~\\
	\vspace{-2\baselineskip}
	\par
	\insertimageleftboxedp[#1]{#2}{#3}{#4}{0}{#5}
}

% Insertar una imagen a la izquierda, propiedades variables
% 	#1	Label (opcional)
%	#2	Dirección de la imagen
%	#3	Ancho del objeto
%	#4	Propiedades de la imagen
%	#5	Ancho de la línea (en pt)
%	#6	Leyenda de la imagen (opcional)
\newcommand{\insertimageleftboxedp}[6][]{
	\emptyvarerr{\insertimageleftboxedp}{#2}{Direccion de la imagen no definida}
	\emptyvarerr{\insertimageleftboxedp}{#3}{Ancho del objeto no definido}
	\emptyvarerr{\insertimageleftboxedp}{#4}{Propiedades de la imagen no defindos}
	\emptyvarerr{\insertimageleftboxedp}{#5}{Ancho de la linea no definido}
	\checkoutsideenvimage
	~
	\vspace{-\baselineskip}
	\par
	\begin{wrapfigure}{l}{#3}
		\setcaptionmargincm{0}
		\ifthenelse{\equal{\figurecaptiontop}{true}}{}{
			\vspace{\marginfloatimages pt}
		}
		\begingroup
			\setlength{\fboxsep}{0 pt}
			\setlength{\fboxrule}{#5 pt}
			\centering
			\fbox{\includegraphics[#4]{#2}}
		\endgroup
		\ifx\hfuzz#6\hfuzz
			\vspace{\captionlessmarginimage cm}
		\else
			\caption{#6 #1}
		\fi
	\end{wrapfigure}
	\setcaptionmargincm{\captionlrmargin}
}

% Insertar una imagen a la izquierda, ajustada en un número de líneas, propiedades variables
% 	#1	Label (opcional)
%	#2	Dirección de la imagen
%	#3	Ancho del objeto
%	#4	Propiedades de la imagen
%	#5	Altura en líneas de la imagen
%	#6	Leyenda de la imagen (opcional)
\newcommand{\insertimageleftlinep}[6][]{
	\insertimageleftlineboxedp[#1]{#2}{#3}{#4}{0}{#5}{#6}
}

% Insertar una imagen recuadrada a la izquierda, ajustada en un número de líneas, propiedades variables
% 	#1	Label (opcional)
%	#2	Dirección de la imagen
%	#3	Ancho del objeto
%	#4	Propiedades de la imagen
%	#5	Ancho de la línea (en pt)
%	#6	Altura en líneas de la imagen
%	#7	Leyenda de la imagen (opcional)
\newcommand{\insertimageleftlineboxedp}[7][]{
	\emptyvarerr{\insertimageleftlineboxedp}{#2}{Direccion de la imagen no definida}
	\emptyvarerr{\insertimageleftlineboxedp}{#3}{Ancho del objeto no definido}
	\emptyvarerr{\insertimageleftlineboxedp}{#4}{Propiedades de la imagen no definidos}
	\emptyvarerr{\insertimageleftlineboxedp}{#5}{Ancho de la linea no definido}
	\emptyvarerr{\insertimageleftlineboxedp}{#6}{Altura en lineas de la imagen flotante izquierda no definida}
	\checkoutsideenvimage
	~
	\vspace{-\baselineskip}
	\par
	\begin{wrapfigure}[#6]{l}{#3}
		\setcaptionmargincm{0}
		\ifthenelse{\equal{\figurecaptiontop}{true}}{}{
			\vspace{\marginfloatimages pt}
		}
		\begingroup
			\setlength{\fboxsep}{0 pt}
			\setlength{\fboxrule}{#5 pt}
			\centering
			\fbox{\includegraphics[#4]{#2}}
		\endgroup
		\ifx\hfuzz#7\hfuzz
			\vspace{\captionlessmarginimage cm}
		\else
			\caption{#7 #1}
		\fi
	\end{wrapfigure}
	\setcaptionmargincm{\captionlrmargin}
}

% Insertar una imagen a la derecha, propiedades variables
% 	#1	Label (opcional)
%	#2	Dirección de la imagen
%	#3	Ancho del objeto (en cm)
%	#4	Propiedades de la imagen
%	#5	Leyenda de la imagen (opcional)
\newcommand{\insertimagerightp}[5][]{
	\xspace~\\
	\vspace{-2\baselineskip}
	\par
	\insertimagerightboxedp[#1]{#2}{#3}{#4}{0}{#5}
}

% Insertar una imagen recuadrada a la derecha, propiedades variables
% 	#1	Label (opcional)
%	#2	Dirección de la imagen
%	#3	Ancho del objeto
%	#4	Propiedades de la imagen
%	#5	Ancho de la línea (en pt)
%	#6	Leyenda de la imagen (opcional)
\newcommand{\insertimagerightboxedp}[6][]{
	\emptyvarerr{\insertimagerightboxedp}{#2}{Direccion de la imagen no definida}
	\emptyvarerr{\insertimagerightboxedp}{#3}{Ancho del objeto no definido}
	\emptyvarerr{\insertimagerightboxedp}{#4}{Propiedades de la imagen no definidos}
	\emptyvarerr{\insertimagerightboxedp}{#5}{Ancho de la linea no definido}
	\checkoutsideenvimage
	~
	\vspace{-\baselineskip}
	\par
	\begin{wrapfigure}{r}{#3}
		\setcaptionmargincm{0}
		\ifthenelse{\equal{\figurecaptiontop}{true}}{}{
			\vspace{\marginfloatimages pt}
		}
		\begingroup
			\setlength{\fboxsep}{0 pt}
			\setlength{\fboxrule}{#5 pt}
			\centering
			\fbox{\includegraphics[#4]{#2}}
		\endgroup
		\ifx\hfuzz#6\hfuzz
			\vspace{\captionlessmarginimage cm}
		\else
			\caption{#6 #1}
		\fi
	\end{wrapfigure}
	\setcaptionmargincm{\captionlrmargin}
}

% Insertar una imagen a la derecha, ajustada en un número de líneas, propiedades variables
% 	#1	Label (opcional)
%	#2	Dirección de la imagen
%	#3	Ancho del objeto (en cm)
%	#4	Propiedades de la imagen
%	#5	Altura en líneas de la imagen
%	#6	Leyenda de la imagen (opcional)
\newcommand{\insertimagerightlinep}[6][]{
	\insertimagerightlineboxedp[#1]{#2}{#3}{#4}{0}{#5}{#6}
}

% Insertar una imagen recuadrada a la derecha, ajustada en un número de líneas, propiedades variables
% 	#1	Label (opcional)
%	#2	Dirección de la imagen
%	#3	Ancho del objeto
%	#4	Propiedades de la imagen
%	#5	Ancho de la línea (en pt)
%	#6	Altura en líneas de la imagen
%	#7	Leyenda de la imagen (opcional)
\newcommand{\insertimagerightlineboxedp}[7][]{
	\emptyvarerr{\insertimagerightlineboxedp}{#2}{Direccion de la imagen no definida}
	\emptyvarerr{\insertimagerightlineboxedp}{#3}{Ancho del objeto no definido}
	\emptyvarerr{\insertimagerightlineboxedp}{#4}{Propiedades de la imagen no definidos}
	\emptyvarerr{\insertimagerightlineboxedp}{#5}{Ancho de la linea no definido}
	\emptyvarerr{\insertimagerightlineboxedp}{#6}{Altura en lineas de la imagen flotante derecha no definida}
	\checkoutsideenvimage
	~
	\vspace{-\baselineskip}
	\par
	\begin{wrapfigure}[#6]{r}{#3}
		\setcaptionmargincm{0}
		\ifthenelse{\equal{\figurecaptiontop}{true}}{}{
			\vspace{\marginfloatimages pt}
		}
		\begingroup
			\setlength{\fboxsep}{0 pt}
			\setlength{\fboxrule}{#5 pt}
			\centering
			\fbox{\includegraphics[#4]{#2}}
		\endgroup
		\ifx\hfuzz#7\hfuzz
			\vspace{\captionlessmarginimage cm}
		\else
			\caption{#7 #1}
		\fi
	\end{wrapfigure}
	\setcaptionmargincm{\captionlrmargin}
}

% END