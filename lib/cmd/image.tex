% Template:     Informe/Reporte LaTeX
% Documento:    Funciones para insertar imágenes
% Versión:      5.2.2 (21/05/2018)
% Codificación: UTF-8
%
% Autor: Pablo Pizarro R. @ppizarror
%        Facultad de Ciencias Físicas y Matemáticas
%        Universidad de Chile
%        pablo.pizarro@ing.uchile.cl, ppizarror.com
%
% Manual template: [http://latex.ppizarror.com/Template-Informe/]
% Licencia MIT:    [https://opensource.org/licenses/MIT/]

% Añade una imagen a un env 'image' con borde
%	#1	Dirección de la imagen
%	#2	Parámetros de la imagen
%	#3	Leyenda de la imagen (opcional)
\newcommand{\addimage}[3]{
	\addimageboxed{#1}{#2}{0}{#3}
}

% Añade una imagen a un env 'image' con borde
%	#1	Dirección de la imagen
%	#2	Parámetros de la imagen
%	#3	Ancho de la línea (en pt)
%	#4	Leyenda de la imagen (opcional)
\newcommand{\addimageboxed}[4]{
	\checkonlyonenvimage
	\begingroup
		\setlength{\fboxsep}{0 pt}
		\setlength{\fboxrule}{#3 pt}
		\hspace{\marginrightmultimage cm}
		\subfloat[#4]{
			\fbox{\includegraphics[#2]{#1}}
		}
	\endgroup
}

% Insertar una imagen
% 	#1	Label (opcional)
%	#2	Dirección de la imagen
%	#3	Parámetros de la imagen
%	#4	Leyenda de la imagen (opcional)
\newcommand{\insertimage}[4][]{
	\insertimageboxed[#1]{#2}{#3}{0}{#4}
}

% Insertar una imagen con recuadro
% 	#1	Label (opcional)
%	#2	Dirección de la imagen
%	#3	Parámetros de la imagen
%	#4	Ancho de la línea (en pt)
%	#5	Leyenda de la imagen (opcional)
\newcommand{\insertimageboxed}[5][]{
	\emptyvarerr{\insertimageboxed}{#2}{Direccion de la imagen no definida}
	\emptyvarerr{\insertimageboxed}{#3}{Parametros de la imagen no definidos}
	\emptyvarerr{\insertimageboxed}{#4}{Ancho de la linea no definido}
	\checkoutsideenvimage
	\vspace{\margintopimages cm}
	\begin{figure}[H]
		\begingroup
			\setlength{\fboxsep}{0 pt}
			\setlength{\fboxrule}{#4 pt}
			\centering
			\fbox{\includegraphics[#3]{#2}}
		\endgroup
		\ifx\hfuzz#5\hfuzz
			\vspace{\captionlessmarginimage cm}
		\else
			\hspace{0cm}\caption{#5 #1}
		\fi
	\end{figure}
	\vspace{\marginbottomimages cm}
}

% Insertar una imagen doble
% [DEPRECADA]
% 	#1	Label (opcional)
%	#2	Dirección de la imagen 1
%	#3	Parámetros de la imagen 1
%	#4	Leyenda de la imagen 1
%	#5	Dirección de la imagen 2
%	#6	Parámetros de la imagen 2
%	#7	Leyenda de la imagen 2
%	#8	Leyenda de la imagen doble (opcional)
\newcommand{\insertdoubleimage}[8][]{
	\emptyvarerr{\insertdoubleimage}{#2}{Direccion de la imagen 1 no definida}
	\emptyvarerr{\insertdoubleimage}{#3}{Parametros de la imagen 1 no definidos}
	\emptyvarerr{\insertdoubleimage}{#5}{Direccion de la imagen 2 no definida}
	\emptyvarerr{\insertdoubleimage}{#6}{Parametros de la imagen 2 no definidos}
	\checkoutsideenvimage
	\alertdeprecatedcmdimage
	\vspace{\margintopimages cm}
	\captionsetup{margin=\captionmarginmultimg cm}
	\begin{figure}[H] \centering
		\subfloat[#4]{
			\includegraphics[#3]{#2}
		}
		\hspace{\marginrightmultimage cm}
		\subfloat[#7]{
			\includegraphics[#6]{#5}
		}
		\setcaptionmargincm{\captionlrmargin}
		\ifx\hfuzz#8\hfuzz
			\vspace{\captionlessmarginimage cm}
		\else
			\caption{#8 #1}
		\fi
	\end{figure}
	\setcaptionmargincm{\captionlrmargin}
	\vspace{\marginbottomimages cm}
}

% Insertar una imagen doble, igual propiedades
% [DEPRECADA]
% 	#1	Label (opcional)
%	#2	Dirección de la imagen 1
%	#3	Leyenda de la imagen 1
%	#4	Dirección de la imagen 2
%	#5	Leyenda de la imagen 2
%	#6	Propiedades de las imágenes
%	#7 	Leyenda de la imagen doble (opcional)
\newcommand{\insertdoubleeqimage}[7][]{
	\insertdoubleimage[#1]{#2}{#6}{#3}{#4}{#6}{#5}{#7}
}

% Insertar una imagen triple
% [DEPRECADA]
% 	#1	Label (opcional)
%	#2	Dirección de la imagen 1
%	#3	Parámetros de la imagen 1
%	#4	Dirección de la imagen 2
%	#5	Parámetros de la imagen 2
%	#6	Dirección de la imagen 3
%	#7	Parámetros de la imagen 3
%	#8	Leyenda de la imagen triple (opcional)
\newcommand{\inserttripleimage}[8][]{
	\emptyvarerr{\inserttripleimage}{#2}{Direccion de la imagen 1 no definida}
	\emptyvarerr{\inserttripleimage}{#3}{Parametros de la imagen 1 no definidos}
	\emptyvarerr{\inserttripleimage}{#4}{Direccion de la imagen 2 no definida}
	\emptyvarerr{\inserttripleimage}{#5}{Parametros de la imagen 2 no definidos}
	\emptyvarerr{\inserttripleimage}{#6}{Direccion de la imagen 3 no definida}
	\emptyvarerr{\inserttripleimage}{#7}{Parametros de la imagen 3 no definidos}
	\checkoutsideenvimage
	\alertdeprecatedcmdimage
	\vspace{\margintopimages cm}
	\captionsetup{margin=\captionmarginmultimg cm}
	\begin{figure}[H] \centering
		\subfloat[]{
			\includegraphics[#3]{#2}
		}
		\hspace{\marginrightmultimage cm}
		\subfloat[]{
			\includegraphics[#5]{#4}
		}
		\hspace{\marginrightmultimage cm}
		\subfloat[]{
			\includegraphics[#7]{#6}
		}
		\setcaptionmargincm{\captionlrmargin}
		\ifx\hfuzz#8\hfuzz
			\vspace{\captionlessmarginimage cm}
		\else
			\caption{#8 #1}
		\fi
	\end{figure}
	\setcaptionmargincm{\captionlrmargin}
	\vspace{\marginbottomimages cm}
}

% Insertar una imagen triple, igual propiedades
% [DEPRECADA]
% 	#1	Label (opcional)
%	#2	Dirección de la imagen 1
%	#3	Dirección de la imagen 2
%	#4	Dirección de la imagen 3
%	#5	Propiedades de las imágenes
%	#6	Leyenda de la imagen triple (opcional)
\newcommand{\inserttripleeqimage}[6][]{
	\inserttripleimage[#1]{#2}{#5}{#3}{#5}{#4}{#5}{#6}
}

% Insertar una imagen cuádruple, igual propiedades
% [DEPRECADA]
% 	#1	Label (opcional)
%	#2	Dirección de la imagen 1
%	#3	Dirección de la imagen 2
%	#4	Dirección de la imagen 3
%	#5	Dirección de la imagen 4
%	#6	Propiedades de las imágenes
%	#7	Leyenda de la imagen cuádruple (opcional)
\newcommand{\insertquadimage}[7][]{
	\emptyvarerr{\insertquadimage}{#2}{Direccion de la imagen 1 no definida}
	\emptyvarerr{\insertquadimage}{#3}{Direccion de la imagen 2 no definida}
	\emptyvarerr{\insertquadimage}{#4}{Direccion de la imagen 3 no definida}
	\emptyvarerr{\insertquadimage}{#5}{Direccion de la imagen 4 no definida}
	\emptyvarerr{\insertquadimage}{#6}{Propiedades de las imagenes no definidos}
	\checkoutsideenvimage
	\alertdeprecatedcmdimage
	\vspace{\margintopimages cm}
	\captionsetup{margin=\captionmarginmultimg cm cm}
	\begin{figure}[H] \centering
		\subfloat[]{
			\includegraphics[#6]{#2}
		}
		\hspace{\marginrightmultimage cm}
		\subfloat[]{
			\includegraphics[#6]{#3}
		}
		\hspace{\marginrightmultimage cm}
		\subfloat[]{
			\includegraphics[#6]{#4}
		}
		\hspace{\marginrightmultimage cm}
		\subfloat[]{
			\includegraphics[#6]{#5}
		}
		\setcaptionmargincm{\captionlrmargin}
		\ifx\hfuzz#7\hfuzz
			\vspace{\captionlessmarginimage cm}
		\else
			\caption{#7 #1}
		\fi
	\end{figure}
	\setcaptionmargincm{\captionlrmargin}
	\vspace{\marginbottomimages cm}
}

% Insertar una imagen quíntuple, igual propiedades
% [DEPRECADA]
% 	#1	Label (opcional)
%	#2	Dirección de la imagen 1
%	#3	Dirección de la imagen 2
%	#4	Dirección de la imagen 3
%	#5	Dirección de la imagen 4
%	#6	Dirección de la imagen 5
%	#7	Propiedades de las imágenes
%	#8	Leyenda de la imagen quíntuple (opcional)
\newcommand{\insertpentaimage}[8][]{
	\emptyvarerr{\insertpentaimage}{#2}{Direccion de la imagen 1 no definida}
	\emptyvarerr{\insertpentaimage}{#3}{Direccion de la imagen 2 no definida}
	\emptyvarerr{\insertpentaimage}{#4}{Direccion de la imagen 3 no definida}
	\emptyvarerr{\insertpentaimage}{#5}{Direccion de la imagen 4 no definida}
	\emptyvarerr{\insertpentaimage}{#6}{Direccion de la imagen 5 no definida}
	\emptyvarerr{\insertpentaimage}{#7}{Propiedades de las imagenes no definidas}
	\checkoutsideenvimage
	\alertdeprecatedcmdimage
	\vspace{\margintopimages cm}
	\captionsetup{margin=\captionmarginmultimg cm cm}
	\begin{figure}[H] \centering
		\subfloat[]{
			\includegraphics[#7]{#2}
		}
		\hspace{\marginrightmultimage cm}
		\subfloat[]{
			\includegraphics[#7]{#3}
		}
		\hspace{\marginrightmultimage cm}
		\subfloat[]{
			\includegraphics[#7]{#4}
		}
		\hspace{\marginrightmultimage cm}
		\subfloat[]{
			\includegraphics[#7]{#5}
		}
		\hspace{\marginrightmultimage cm}
		\subfloat[]{
			\includegraphics[#7]{#6}
		}
		\setcaptionmargincm{\captionlrmargin}
		\ifx\hfuzz#8\hfuzz
			\vspace{\captionlessmarginimage cm}
		\else
			\caption{#8 #1}
		\fi
	\end{figure}
	\setcaptionmargincm{\captionlrmargin}
	\vspace{\marginbottomimages cm}
}

% Insertar una imagen con 6 imágenes, igual propiedades
% [DEPRECADA]
% 	#1	Label (opcional)
%	#2	Dirección de la imagen 1
%	#3	Dirección de la imagen 2
%	#4	Dirección de la imagen 3
%	#5	Dirección de la imagen 4
%	#6	Dirección de la imagen 5
%	#7	Dirección de la imagen 6
%	#8	Propiedades de las imágenes
%	#9	Leyenda de la imagen global (opcional)
\newcommand{\inserthexaimage}[9][]{
	\emptyvarerr{\inserthexaimage}{#2}{Direccion de la imagen 1 no definida}
	\emptyvarerr{\inserthexaimage}{#3}{Direccion de la imagen 2 no definida}
	\emptyvarerr{\inserthexaimage}{#4}{Direccion de la imagen 3 no definida}
	\emptyvarerr{\inserthexaimage}{#5}{Direccion de la imagen 4 no definida}
	\emptyvarerr{\inserthexaimage}{#6}{Direccion de la imagen 5 no definida}
	\emptyvarerr{\inserthexaimage}{#7}{Direccion de la imagen 6 no definida}
	\emptyvarerr{\inserthexaimage}{#8}{Propiedades de las imagenes no definidas}
	\checkoutsideenvimage
	\alertdeprecatedcmdimage
	\vspace{\margintopimages cm}
	\captionsetup{margin=\captionmarginmultimg cm}
	\begin{figure}[H] \centering
		\subfloat[]{
			\includegraphics[#8]{#2}
		}
		\hspace{\marginrightmultimage cm}
		\subfloat[]{
			\includegraphics[#8]{#3}
		}
		\hspace{\marginrightmultimage cm}
		\subfloat[]{
			\includegraphics[#8]{#4}
		}
		\hspace{\marginrightmultimage cm}
		\subfloat[]{
			\includegraphics[#8]{#5}
		}
		\hspace{\marginrightmultimage cm}
		\subfloat[]{
			\includegraphics[#8]{#6}
		}
		\hspace{\marginrightmultimage cm}
		\subfloat[]{
			\includegraphics[#8]{#7}
		}
		\setcaptionmargincm{\captionlrmargin}
		\ifx\hfuzz#9\hfuzz
			\vspace{\captionlessmarginimage cm}
		\else
			\caption{#9 #1}
		\fi
	\end{figure}
	\setcaptionmargincm{\captionlrmargin}
	\vspace{\marginbottomimages cm}
}

% Insertar una imagen a la izquierda, escalada, ancho fijo
% 	#1	Label (opcional)
%	#2	Dirección de la imagen
%	#3	Ancho de la imagen (en textwidth)
%	#4	Leyenda de la imagen (opcional)
\newcommand{\insertimageleft}[4][]{
	\insertimageleftboxed[#1]{#2}{#3}{0}{#4}
}

% Insertar una imagen a la izquierda, escalada, ancho fijo
% 	#1	Label (opcional)
%	#2	Dirección de la imagen
%	#3	Ancho de la imagen (en textwidth)
%	#4	Ancho de la línea (en pt)
%	#5	Leyenda de la imagen (opcional)
\newcommand{\insertimageleftboxed}[5][]{
	\emptyvarerr{\insertimageleftboxed}{#2}{Direccion de la imagen no definida}
	\emptyvarerr{\insertimageleftboxed}{#3}{Ancho de la imagen no definido}
	\emptyvarerr{\insertimageleftboxed}{#4}{Ancho de la linea no definido}
	\checkoutsideenvimage
	~
	\vspace{-\baselineskip}
	\par
	\begin{wrapfigure}{l}{#3\textwidth}
		\setcaptionmargincm{0}
		\ifthenelse{\equal{\figurecaptiontop}{true}}{}{
			\vspace{\marginfloatimages pt}
		}
		\begingroup
			\setlength{\fboxsep}{0 pt}
			\setlength{\fboxrule}{#4 pt}
			\centering
			\fbox{\includegraphics[width=\linewidth]{#2}}
		\endgroup
		\ifx\hfuzz#5\hfuzz
			\vspace{\captionlessmarginimage cm}
		\else
			\caption{#5 #1}
		\fi
	\end{wrapfigure}
	\setcaptionmargincm{\captionlrmargin}
}

% Insertar una imagen a la izquierda, ajustada en un número de líneas, escalada, ancho fijo
% 	#1	Label (opcional)
%	#2	Dirección de la imagen
%	#3	Ancho de la imagen (en textwidth)
%	#4	Altura en líneas de la imagen
%	#5	Leyenda de la imagen (opcional)
\newcommand{\insertimageleftline}[5][]{
	\insertimageleftlineboxed[#1]{#2}{#3}{0}{#4}{#5}
}

% Insertar una imagen recuadrada a la izquierda, ajustada en un número de líneas, escalada, ancho fijo
% 	#1	Label (opcional)
%	#2	Dirección de la imagen
%	#3	Ancho de la imagen (en textwidth)
%	#4	Ancho de la línea (en pt)
%	#5	Altura en líneas de la imagen
%	#6	Leyenda de la imagen (opcional)
\newcommand{\insertimageleftlineboxed}[6][]{
	\emptyvarerr{\insertimageleftlineboxed}{#2}{Direccion de la imagen no definida}
	\emptyvarerr{\insertimageleftlineboxed}{#3}{Ancho de la imagen no definido}
	\emptyvarerr{\insertimageleftlineboxed}{#4}{Ancho de la linea no definido}
	\emptyvarerr{\insertimageleftlineboxed}{#5}{Altura en lineas de la imagen flotante izquierda no definida}
	\checkoutsideenvimage
	~
	\vspace{-\baselineskip}
	\par
	\begin{wrapfigure}[#5]{l}{#3\textwidth}
		\setcaptionmargincm{0}
		\ifthenelse{\equal{\figurecaptiontop}{true}}{}{
			\vspace{\marginfloatimages pt}}
		\begingroup
			\setlength{\fboxsep}{0 pt}
			\setlength{\fboxrule}{#4 pt}
			\centering
			\fbox{\includegraphics[width=\linewidth]{#2}}
		\endgroup
		\ifx\hfuzz#6\hfuzz
			\vspace{\captionlessmarginimage cm}
		\else
			\caption{#6 #1}
		\fi
	\end{wrapfigure}
	\setcaptionmargincm{\captionlrmargin}
}

% Insertar una imagen a la derecha, escalada, ancho fijo
% 	#1	Label (opcional)
%	#2	Dirección de la imagen
%	#3	Ancho de la imagen (en textwidth)
%	#4	Leyenda de la imagen (opcional)
\newcommand{\insertimageright}[4][]{
	\insertimagerightboxed[#1]{#2}{#3}{0}{#4}
}

% Insertar una imagen recuadrada a la derecha, escalada, ancho fijo
% 	#1	Label (opcional)
%	#2	Dirección de la imagen
%	#3	Ancho de la imagen (en textwidth)
%	#4	Ancho de la línea (en pt)
%	#5	Leyenda de la imagen (opcional)
\newcommand{\insertimagerightboxed}[5][]{
	\emptyvarerr{\insertimagerightboxed}{#2}{Direccion de la imagen no definida}
	\emptyvarerr{\insertimagerightboxed}{#3}{Ancho de la imagen no defindo}
	\emptyvarerr{\insertimagerightboxed}{#4}{Ancho de la linea no definido}
	\checkoutsideenvimage
	~
	\vspace{-\baselineskip}
	\par
	\begin{wrapfigure}{r}{#3\textwidth}
		\setcaptionmargincm{0}
		\ifthenelse{\equal{\figurecaptiontop}{true}}{}{
			\vspace{\marginfloatimages pt}
		}
		\begingroup
			\setlength{\fboxsep}{0 pt}
			\setlength{\fboxrule}{#4 pt}
			\centering
			\fbox{\includegraphics[width=\linewidth]{#2}}
		\endgroup
		\ifx\hfuzz#5\hfuzz
			\vspace{\captionlessmarginimage cm}
		\else
			\caption{#5 #1}
		\fi
	\end{wrapfigure}
	\setcaptionmargincm{\captionlrmargin}
}

% Insertar una imagen a la derecha, ajustada en un número de líneas, escalada, ancho fijo
% 	#1	Label (opcional)
%	#2	Dirección de la imagen
%	#3	Ancho de la imagen (en textwidth)
%	#4	Altura en líneas de la imagen
%	#5	Leyenda de la imagen (opcional)
\newcommand{\insertimagerightline}[5][]{
	\insertimagerightlineboxed[#1]{#2}{#3}{0}{#4}{#5}
}

% Insertar una imagen recuadrada a la derecha, ajustada en un número de líneas, escalada, ancho fijo
% 	#1	Label (opcional)
%	#2	Dirección de la imagen
%	#3	Ancho de la imagen (en textwidth)
%	#4	Ancho de la línea (en pt)
%	#5	Altura en líneas de la imagen
%	#6	Leyenda de la imagen (opcional)
\newcommand{\insertimagerightlineboxed}[6][]{
	\emptyvarerr{\insertimagerightlineboxed}{#2}{Direccion de la imagen no definida}
	\emptyvarerr{\insertimagerightlineboxed}{#3}{Ancho de la imagen no defindo}
	\emptyvarerr{\insertimagerightlineboxed}{#4}{Ancho de la linea no definido}
	\emptyvarerr{\insertimagerightlineboxed}{#5}{Altura en lineas de la imagen flotante derecha no definida}
	\checkoutsideenvimage
	~
	\vspace{-\baselineskip}
	\par
	\begin{wrapfigure}[#5]{r}{#3\textwidth}
		\setcaptionmargincm{0}
		\ifthenelse{\equal{\figurecaptiontop}{true}}{}{
			\vspace{\marginfloatimages pt}
		}
		\begingroup
			\setlength{\fboxsep}{0 pt}
			\setlength{\fboxrule}{#4 pt}
			\centering
			\fbox{\includegraphics[width=\linewidth]{#2}}
		\endgroup
		\ifx\hfuzz#6\hfuzz
			\vspace{\captionlessmarginimage cm}
		\else
			\caption{#6 #1}
		\fi
	\end{wrapfigure}
	\setcaptionmargincm{\captionlrmargin}
}

% Insertar una imagen a la izquierda, propiedades variables
% 	#1	Label (opcional)
%	#2	Dirección de la imagen
%	#3	Ancho del objeto
%	#4	Propiedades de la imagen
%	#5	Leyenda de la imagen (opcional)
\newcommand{\insertimageleftp}[5][]{
	\xspace~\\
	\vspace{-2\baselineskip}
	\par
	\insertimageleftboxedp[#1]{#2}{#3}{#4}{0}{#5}
}

% Insertar una imagen a la izquierda, propiedades variables
% 	#1	Label (opcional)
%	#2	Dirección de la imagen
%	#3	Ancho del objeto
%	#4	Propiedades de la imagen
%	#5	Ancho de la línea (en pt)
%	#6	Leyenda de la imagen (opcional)
\newcommand{\insertimageleftboxedp}[6][]{
	\emptyvarerr{\insertimageleftboxedp}{#2}{Direccion de la imagen no definida}
	\emptyvarerr{\insertimageleftboxedp}{#3}{Ancho del objeto no definido}
	\emptyvarerr{\insertimageleftboxedp}{#4}{Propiedades de la imagen no defindos}
	\emptyvarerr{\insertimageleftboxedp}{#5}{Ancho de la linea no definido}
	\checkoutsideenvimage
	~
	\vspace{-\baselineskip}
	\par
	\begin{wrapfigure}{l}{#3}
		\setcaptionmargincm{0}
		\ifthenelse{\equal{\figurecaptiontop}{true}}{}{
			\vspace{\marginfloatimages pt}
		}
		\begingroup
			\setlength{\fboxsep}{0 pt}
			\setlength{\fboxrule}{#5 pt}
			\centering
			\fbox{\includegraphics[#4]{#2}}
		\endgroup
		\ifx\hfuzz#6\hfuzz
			\vspace{\captionlessmarginimage cm}
		\else
			\caption{#6 #1}
		\fi
	\end{wrapfigure}
	\setcaptionmargincm{\captionlrmargin}
}

% Insertar una imagen a la izquierda, ajustada en un número de líneas, propiedades variables
% 	#1	Label (opcional)
%	#2	Dirección de la imagen
%	#3	Ancho del objeto
%	#4	Propiedades de la imagen
%	#5	Altura en líneas de la imagen
%	#6	Leyenda de la imagen (opcional)
\newcommand{\insertimageleftlinep}[6][]{
	\insertimageleftlineboxedp[#1]{#2}{#3}{#4}{0}{#5}{#6}
}

% Insertar una imagen recuadrada a la izquierda, ajustada en un número de líneas, propiedades variables
% 	#1	Label (opcional)
%	#2	Dirección de la imagen
%	#3	Ancho del objeto
%	#4	Propiedades de la imagen
%	#5	Ancho de la línea (en pt)
%	#6	Altura en líneas de la imagen
%	#7	Leyenda de la imagen (opcional)
\newcommand{\insertimageleftlineboxedp}[7][]{
	\emptyvarerr{\insertimageleftlineboxedp}{#2}{Direccion de la imagen no definida}
	\emptyvarerr{\insertimageleftlineboxedp}{#3}{Ancho del objeto no definido}
	\emptyvarerr{\insertimageleftlineboxedp}{#4}{Propiedades de la imagen no definidos}
	\emptyvarerr{\insertimageleftlineboxedp}{#5}{Ancho de la linea no definido}
	\emptyvarerr{\insertimageleftlineboxedp}{#6}{Altura en lineas de la imagen flotante izquierda no definida}
	\checkoutsideenvimage
	~
	\vspace{-\baselineskip}
	\par
	\begin{wrapfigure}[#6]{l}{#3}
		\setcaptionmargincm{0}
		\ifthenelse{\equal{\figurecaptiontop}{true}}{}{
			\vspace{\marginfloatimages pt}
		}
		\begingroup
			\setlength{\fboxsep}{0 pt}
			\setlength{\fboxrule}{#5 pt}
			\centering
			\fbox{\includegraphics[#4]{#2}}
		\endgroup
		\ifx\hfuzz#7\hfuzz
			\vspace{\captionlessmarginimage cm}
		\else
			\caption{#7 #1}
		\fi
	\end{wrapfigure}
	\setcaptionmargincm{\captionlrmargin}
}

% Insertar una imagen a la derecha, propiedades variables
% 	#1	Label (opcional)
%	#2	Dirección de la imagen
%	#3	Ancho del objeto (en cm)
%	#4	Propiedades de la imagen
%	#5	Leyenda de la imagen (opcional)
\newcommand{\insertimagerightp}[5][]{
	\xspace~\\
	\vspace{-2\baselineskip}
	\par
	\insertimagerightboxedp[#1]{#2}{#3}{#4}{0}{#5}
}

% Insertar una imagen recuadrada a la derecha, propiedades variables
% 	#1	Label (opcional)
%	#2	Dirección de la imagen
%	#3	Ancho del objeto
%	#4	Propiedades de la imagen
%	#5	Ancho de la línea (en pt)
%	#6	Leyenda de la imagen (opcional)
\newcommand{\insertimagerightboxedp}[6][]{
	\emptyvarerr{\insertimagerightboxedp}{#2}{Direccion de la imagen no definida}
	\emptyvarerr{\insertimagerightboxedp}{#3}{Ancho del objeto no definido}
	\emptyvarerr{\insertimagerightboxedp}{#4}{Propiedades de la imagen no definidos}
	\emptyvarerr{\insertimagerightboxedp}{#5}{Ancho de la linea no definido}
	\checkoutsideenvimage
	~
	\vspace{-\baselineskip}
	\par
	\begin{wrapfigure}{r}{#3}
		\setcaptionmargincm{0}
		\ifthenelse{\equal{\figurecaptiontop}{true}}{}{
			\vspace{\marginfloatimages pt}
		}
		\begingroup
			\setlength{\fboxsep}{0 pt}
			\setlength{\fboxrule}{#5 pt}
			\centering
			\fbox{\includegraphics[#4]{#2}}
		\endgroup
		\ifx\hfuzz#6\hfuzz
			\vspace{\captionlessmarginimage cm}
		\else
			\caption{#6 #1}
		\fi
	\end{wrapfigure}
	\setcaptionmargincm{\captionlrmargin}
}

% Insertar una imagen a la derecha, ajustada en un número de líneas, propiedades variables
% 	#1	Label (opcional)
%	#2	Dirección de la imagen
%	#3	Ancho del objeto (en cm)
%	#4	Propiedades de la imagen
%	#5	Altura en líneas de la imagen
%	#6	Leyenda de la imagen (opcional)
\newcommand{\insertimagerightlinep}[6][]{
	\insertimagerightlineboxedp[#1]{#2}{#3}{#4}{0}{#5}{#6}
}

% Insertar una imagen recuadrada a la derecha, ajustada en un número de líneas, propiedades variables
% 	#1	Label (opcional)
%	#2	Dirección de la imagen
%	#3	Ancho del objeto
%	#4	Propiedades de la imagen
%	#5	Ancho de la línea (en pt)
%	#6	Altura en líneas de la imagen
%	#7	Leyenda de la imagen (opcional)
\newcommand{\insertimagerightlineboxedp}[7][]{
	\emptyvarerr{\insertimagerightlineboxedp}{#2}{Direccion de la imagen no definida}
	\emptyvarerr{\insertimagerightlineboxedp}{#3}{Ancho del objeto no definido}
	\emptyvarerr{\insertimagerightlineboxedp}{#4}{Propiedades de la imagen no definidos}
	\emptyvarerr{\insertimagerightlineboxedp}{#5}{Ancho de la linea no definido}
	\emptyvarerr{\insertimagerightlineboxedp}{#6}{Altura en lineas de la imagen flotante derecha no definida}
	\checkoutsideenvimage
	~
	\vspace{-\baselineskip}
	\par
	\begin{wrapfigure}[#6]{r}{#3}
		\setcaptionmargincm{0}
		\ifthenelse{\equal{\figurecaptiontop}{true}}{}{
			\vspace{\marginfloatimages pt}
		}
		\begingroup
			\setlength{\fboxsep}{0 pt}
			\setlength{\fboxrule}{#5 pt}
			\centering
			\fbox{\includegraphics[#4]{#2}}
		\endgroup
		\ifx\hfuzz#7\hfuzz
			\vspace{\captionlessmarginimage cm}
		\else
			\caption{#7 #1}
		\fi
	\end{wrapfigure}
	\setcaptionmargincm{\captionlrmargin}
}
