% Crea una sección de referencias solo para bibtex
\newenvironment{references}{
	\ifthenelse{\equal{\stylecitereferences}{bibtex}}{ % Verifica configuraciones
	}{
		\throwerror{\references}{Solo se puede usar entorno references con estilo citas \noexpand\stylecitereferences=bibtex}
	}
	\phantomsection
	\addcontentsline{toc}{chapter}{\namereferences}
	\begin{thebibliography}{} % Inicia la bibliografía
	}
	{
	\end{thebibliography}
}

% Crea una sección de resumen
\newenvironment{resumen}{
	% Añade a los marcadores
	\setpagemargincm{\pagemarginleft}{\pagemarginright}{\pagemarginright}{\pagemarginbottom}
	\ifthenelse{\equal{\addabstracttobookmarks}{true}}{
		\pdfbookmark{\nameabstract}{contents}}{
	}
	% Inserta la tabla resumen
	\begin{flushright}
		\small
		\tablaresumen
	\end{flushright}
	\vspace{0.25\baselineskip}
	% Título
	\begin{center}
		\textcolor{\titlecolor}{\MakeUppercase{\textbf{\titulotesis}}}
	\end{center} \newp
	\vspace{-\baselineskip}
	}{\vfill\null
	\ifthenelse{\equal{\addemptypagespredoc}{true}}{
		\vfill
		\checkoddpage
		\ifoddpage
		\else
			\insertblankpage
		\fi}{
	}
	\setpagemargincm{\pagemarginleft}{\pagemargintop}{\pagemarginright}{\pagemarginbottom}
}

% Crea una sección de dedicatoria
\newenvironment{dedicatoria}{
	\cleardoublepage\null
	% \ifthenelse{\equal{\adddedictobookmarks}{true}}{
	% 	\belowpdfbookmark{\nameadedic}{contents}}{
	% }
	\vspace{\stretch{1}}
	\begin{flushright}
		\itshape}{
	\end{flushright}
	\vspace{\stretch{2}}
	\null
	\ifthenelse{\equal{\addemptypagespredoc}{true}}{
		\insertblankpage}{
	}
}

% Crea una sección de agradecimientos
\newenvironment{agradecimientos}{
	\chapter*{\nameagradec}
	\ifthenelse{\equal{\addagradectobookmarks}{true}}{
		\belowpdfbookmark{\nameagradec}{contents}}{
	}
	}{
	\newpage
	\ifthenelse{\equal{\addemptypagespredoc}{true}}{
		\insertblankpage}{
	}
}

% END