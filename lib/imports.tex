% Template:     Informe/Reporte LaTeX
% Documento:    Importación de librerías
% Versión:      3.8.2 (27/05/2017)
% Codificación: UTF-8
%
% Autor: Pablo Pizarro R.
%        Facultad de Ciencias Físicas y Matemáticas
%        Universidad de Chile
%        pablo.pizarro@ing.uchile.cl, ppizarror.com
%
% Manual template: [http://ppizarror.com/Template-Informe/]
% Licencia MIT:    [https://opensource.org/licenses/MIT/]

% LIBRERÍAS DEL NÚCLEO
\usepackage[spanish,es-nosectiondot,es-lcroman]{babel} % Idioma
\usepackage{ifthen}    % Permite el manejo de condicionales

% LIBRERÍAS INDEPENDIENTES
\usepackage{array}     % Nuevas características a las tablas
\usepackage{amsmath}   % Librerías matemáticas
\usepackage{amssymb}   % Librerías matemáticas
\usepackage{bigstrut}  % Líneas horizontales en tablas
\usepackage{bm}        % Caracteres en negrita en ecuaciones
\usepackage{booktabs}  % Permite manejar elem. visuales en tablas
\usepackage{caption}   % Leyendas
\usepackage{chngcntr}  % Añade números a las leyendas
\usepackage{colortbl}  % Administración de color en tablas
\usepackage{color}     % Colores
\usepackage{datetime}  % Fechas
\usepackage{fancyhdr}  % Encabezados y pié de páginas
\usepackage{floatrow}  % Permite adminisrar posiciones en los caption
\usepackage{gensymb}   % Simbología común
\usepackage{geometry}  % Dimensiones y geometría del documento
\usepackage{graphicx}  % Propiedades extra para los gráficos
\usepackage{lipsum}    % Permite crear textos dummy
\usepackage{listings}  % Permite añadir código fuente
\usepackage{longtable} % Permite utilizar tablas en varias hojas
\usepackage{mathtools} % Permite utilizar notaciones matemáticas
\usepackage{multicol}  % Múltiples columnas
\usepackage{needspace} % Maneja los espacios de página
\usepackage{notoccite} % Desactiva las citas en el índice
\usepackage{pdfpages}  % Permite administrar páginas en pdf
\usepackage{rotating}  % Permite rotación de objetos
\usepackage{sectsty}   % Cambia el estilo de los títulos
\usepackage{selinput}  % Compatibilidad con acentos
\usepackage{setspace}  % Cambia el espacio entre líneas
\usepackage{siunitx}   % Unidades en latex
\usepackage{soul}      % Permite subrayar texto
\usepackage{subfig}    % Permite agrupar imágenes
\usepackage{textcomp}  % Simbología común
\usepackage{ulem}      % Permite tachar, subrayar, etc
\usepackage{url}       % Permite añadir enlaces
\usepackage{wasysym}   % Contiene caracteres misceláneos
\usepackage{wrapfig}   % Permite comprimir imágenes
\usepackage{xspace}    % Adminsitra espacios en párrafos y líneas

% LIBRERÍAS CON PARÁMETROS
\usepackage[makeroom]{cancel} % Cancelar términos en fórmulas
\usepackage[inline]{enumitem} % Permite enumerar ítems
\usepackage[bottom,norule,hang]{footmisc} % Estilo pié de página
\usepackage[pdfencoding=auto,psdextra]{hyperref} % Enlaces, referencias
\usepackage[figure,table,lstlisting]{totalcount} % Contador de objetos
\usepackage[usenames,dvipsnames]{xcolor} % Paquete de colores avanzado

% LIBRERÍAS DEPENDIENTES
\usepackage{epstopdf}  % Convierte archivos .eps a .pdf
\usepackage{float}     % Administrador de posiciones de objetos
\usepackage{multirow}  % Agrega nuevas opciones a las tablas

% LIBRERÍAS CONDICIONALES
\ifthenelse{           % Agrega puntos a los títulos/subtítulos
	\equal{\showdotontitles}{true}}{
	\usepackage{secdot}
	\sectiondot{subsection}
	\sectiondot{subsubsection}}{
}