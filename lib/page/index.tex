% Template:     Informe/Reporte LaTeX
% Documento:    Índice
% Versión:      5.5.0 (01/09/2018)
% Codificación: UTF-8
%
% Autor: Pablo Pizarro R. @ppizarror
%        Facultad de Ciencias Físicas y Matemáticas
%        Universidad de Chile
%        pablo.pizarro@ing.uchile.cl, ppizarror.com
%
% Manual template: [http://latex.ppizarror.com/Template-Informe/]
% Licencia MIT:    [https://opensource.org/licenses/MIT/]

\ifthenelse{\equal{\showindex}{true}}{
	
	% Crea nueva página y establece estilo de títulos
	\newpage
	\begingroup
	\sectionfont{\color{\indextitlecolor} \fontsizetitlei \styletitlei \selectfont}
	
	% Añade la entrada del índice a los marcadores del pdf
	\ifthenelse{\equal{\addindextobookmarks}{true}}{
		\belowpdfbookmark{\nomltcont}{contents}}{
	}
	\tocloftpagestyle{fancy}
	
	% Configuración del punto en índice
	\ifthenelse{\equal{\showdotontitles}{true}}{
		\def\cftsecaftersnum {.}
		\def\cftsubsecaftersnum {.}
		\def\cftsubsubsecaftersnum {.}
		\def\cftsubsubsubsecaftersnum {.}
		}{
	}

	% Configuración del punto en número de objetos
	\def\cftfigaftersnum {\charafterobjectindex\enspace}
	\def\cftsubfigaftersnum {\charafterobjectindex\enspace}
	\def\cfttabaftersnum {\charafterobjectindex\enspace}
	\def\cftlstlistingaftersnum {\charafterobjectindex\enspace}
	
	% Configuración carácter número de página
	\renewcommand{\cftdot}{\charnumpageindex}
	
	% Desactiva los números de línea
	\ifthenelse{\equal{\showlinenumbers}{true}}{
		\nolinenumbers}{
	}

	% Cambia tabulación índice de objetos para alinear con contenidos
	\ifthenelse{\equal{\objectindexindent}{true}}{
		\def\cftlstlistingindent {1.495em}
	}{
		\setlength{\cfttabindent}{0in}
		\setlength{\cftfigindent}{0in}
		\setlength{\cftsubfigindent}{0in}
		\setlength{\cftfigindent}{0in}
		\def\cftlstlistingindent {0.01em}
	}
	
	% Iguala tamaño del margen en números de objetos
	\ifthenelse{\equal{\equalmarginnumobject}{true}}{
		
		% Calcula tamaño del margen de los números en objetos del índice dependiendo de la configuración
		\ifthenelse{\equal{\showsectioncaption}{none}}{
			\def\cftdefautnumwidth {2.3em}
		}{
		\ifthenelse{\equal{\showsectioncaption}{sec}}{
			\def\cftdefautnumwidth {3.0em}
		}{
		\ifthenelse{\equal{\showsectioncaption}{ssec}}{
			\def\cftdefautnumwidth {3.8em}
		}{
		\ifthenelse{\equal{\showsectioncaption}{sssec}}{
			\def\cftdefautnumwidth {4.3em}
		}{
			\throwbadconfig{Valor configuracion incorrecto}{\showsectioncaption}{none,sec,ssec,sssec}}}}
		}
		
		% Configuración identado de títulos de objetos después del número
		\def\cftfignumwidth {\cftdefautnumwidth}
		\def\cftsubfignumwidth {\cftdefautnumwidth}
		\def\cfttabnumwidth {\cftdefautnumwidth}
		\def\cftlstlistingnumwidth {\cftdefautnumwidth}}{
	}

	% Índice de Contenidos
	\ifthenelse{\equal{\showindexofcontents}{true}}{\tableofcontents}{}
	
	% Lista de Figuras
	\iftotalfigures
		\ifthenelse{\equal{\showindexoffigures}{true}}{
			\ifthenelse{\equal{\indexforcenewpage}{true}}{\newpage}{}
			\listoffigures
		}{}
	\fi
	
	% Lista de Tablas
	\iftotaltables
		\ifthenelse{\equal{\showindexoftables}{true}}{
			\ifthenelse{\equal{\indexforcenewpage}{true}}{\newpage}{}
			\listoftables
		}{}
	\fi
	
	% Lista del Código Fuente
	\iftotallstlistings
		\ifthenelse{\equal{\showindexofcode}{true}}{
			\ifthenelse{\equal{\indexforcenewpage}{true}}{\newpage}{}
			\lstlistoflistings
		}{}
	\fi
	
	% Termina el bloque de índice
	\endgroup
	
	% Se añade una página en blanco
	\ifthenelse{\equal{\addemptypagetwosides}{true}}{
		\vfill
		\checkoddpage
		\ifoddpage
		\else
			\newpage
			\null
			\thispagestyle{empty}
			\newpage
			\addtocounter{page}{-1}
		\fi}{
	}

}{}
