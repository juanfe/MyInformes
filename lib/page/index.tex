% Template:     Informe/Reporte LaTeX
% Documento:    Índice
% Versión:      DEV
% Codificación: UTF-8
%
% Autor: Pablo Pizarro R. @ppizarror
%        Facultad de Ciencias Físicas y Matemáticas
%        Universidad de Chile
%        pablo.pizarro@ing.uchile.cl, ppizarror.com
%
% Manual template: [https://latex.ppizarror.com/Template-Informe/]
% Licencia MIT:    [https://opensource.org/licenses/MIT/]

\ifthenelse{\equal{\showindex}{true}}{
	
	% -------------------------------------------------------------------------
	% Crea nueva página y establece estilo de títulos
	% -------------------------------------------------------------------------
	\newpage
	\begingroup
	\sectionfont{\color{\indextitlecolor} \fontsizetitlei \styletitlei \selectfont}
	
	% -------------------------------------------------------------------------
	% Salta de página si está imprimiendo por ambas caras
	% -------------------------------------------------------------------------
	\ifthenelse{\equal{\addemptypagetwosides}{true}}{
		\checkoddpage
		\ifoddpage
		\else
			\newpage
			\null
			\thispagestyle{empty}
			\newpage
			\addtocounter{page}{-1}
		\fi}{
	}
	
	% -------------------------------------------------------------------------
	% Añade la entrada del índice a los marcadores del pdf
	% -------------------------------------------------------------------------
	\ifthenelse{\equal{\addindextobookmarks}{true}}{
		\belowpdfbookmark{\nomltcont}{contents}}{
	}
	\tocloftpagestyle{fancy}
	
	% -------------------------------------------------------------------------
	% Configuración del punto en índice
	% -------------------------------------------------------------------------
	\ifthenelse{\equal{\showdotaftersnum}{true}}{		
		% Agrega los puntos
		\def\cftsecaftersnum {.}
		\def\cftsubsecaftersnum {.}
		\def\cftsubsubsecaftersnum {.}
		\def\cftsubsubsubsecaftersnum {.}
		% Modifica los márgenes
		\def\cftsecnumwidth {1.9em}
		\def\cftsubsecnumwidth {2.57em} % Incremento 0.67
		\renewcommand\cftsubsubsecnumwidth{3.35em} % Incremento 0.78
		\setlength{\cftsubsecindent}{1.91em}
		\setlength{\cftsubsubsecindent}{4.48em} % Incremento 2.57
		}{
	}
	
	% -------------------------------------------------------------------------
	% Configuración carácter número de página
	% -------------------------------------------------------------------------
	\renewcommand{\cftdot}{\charnumpageindex}
	
	% -------------------------------------------------------------------------
	% Configuración del punto en número de objetos
	% -------------------------------------------------------------------------
	\def\cftfigaftersnum {\charafterobjectindex\enspace}
	\def\cftsubfigaftersnum {\charafterobjectindex\enspace}
	\def\cfttabaftersnum {\charafterobjectindex\enspace}
	\def\cftlstlistingaftersnum {\charafterobjectindex\enspace}
	
	% -------------------------------------------------------------------------
	% Desactiva los números de línea
	% -------------------------------------------------------------------------
	\ifthenelse{\equal{\showlinenumbers}{true}}{
		\nolinenumbers}{
	}

	% -------------------------------------------------------------------------
	% Cambia tabulación índice de objetos para alinear con contenidos
	% -------------------------------------------------------------------------
	\ifthenelse{\equal{\objectindexindent}{true}}{
		\setlength{\cfttabindent}{1.9em}
		\setlength{\cftfigindent}{1.9em}
		\setlength{\cftsubfigindent}{1.9em}
		\def\cftlstlistingindent {1.9em}
	}{
		\setlength{\cfttabindent}{0em}
		\setlength{\cftfigindent}{0em}
		\setlength{\cftsubfigindent}{0em}
		\def\cftlstlistingindent {0em}
	}
	
	% -------------------------------------------------------------------------
	% Calcula tamaño del margen de los números en objetos del índice
	% -------------------------------------------------------------------------
	
	% Código fuente
	\ifthenelse{\equal{\showsectioncaptioncode}{none}}{
		\def\cftdefautnumwidthcode {3.0em} % Añade +0.7em
		\def\cftdefaultnumwidthromancode {5.25em} % Añade +0.5em para no overflow
	}{
	\ifthenelse{\equal{\showsectioncaptioncode}{sec}}{
		\def\cftdefautnumwidthcode {3.7em}
		\def\cftdefaultnumwidthromancode {5.75em}
	}{
	\ifthenelse{\equal{\showsectioncaptioncode}{ssec}}{
		\def\cftdefautnumwidthcode {4.4em}
		\def\cftdefaultnumwidthromancode {6.25em}
	}{
	\ifthenelse{\equal{\showsectioncaptioncode}{sssec}}{
		\def\cftdefautnumwidthcode {5.1em}
		\def\cftdefaultnumwidthromancode {6.75em}
	}{
	\ifthenelse{\equal{\showsectioncaptioncode}{ssssec}}{
		\def\cftdefautnumwidthcode {5.8em}
		\def\cftdefaultnumwidthromancode {7.25em}
	}{
	\ifthenelse{\equal{\showsectioncaptioncode}{chap}}{
		\def\cftdefautnumwidthcode {3.0em}
		\def\cftdefaultnumwidthromancode {5.25em}
	}{
		\throwbadconfig{Valor configuracion incorrecto}{\showsectioncaptioncode}{none,chap,sec,ssec,sssec,ssssec}}}}}}
	}
	
	% Figuras
	\ifthenelse{\equal{\showsectioncaptionfig}{none}}{
		\def\cftdefautnumwidthfig {3.0em} % Añade +0.7em
		\def\cftdefaultnumwidthromanfig {5.25em} % Añade +0.5em
	}{
	\ifthenelse{\equal{\showsectioncaptionfig}{sec}}{
		\def\cftdefautnumwidthfig {3.7em}
		\def\cftdefaultnumwidthromanfig {5.75em}
	}{
	\ifthenelse{\equal{\showsectioncaptionfig}{ssec}}{
		\def\cftdefautnumwidthfig {4.4em}
		\def\cftdefaultnumwidthromanfig {6.25em}
	}{
	\ifthenelse{\equal{\showsectioncaptionfig}{sssec}}{
		\def\cftdefautnumwidthfig {5.1em}
		\def\cftdefaultnumwidthromanfig {6.75em}
	}{
	\ifthenelse{\equal{\showsectioncaptionfig}{ssssec}}{
		\def\cftdefautnumwidthfig {5.8em}
		\def\cftdefaultnumwidthromanfig {7.25em}
	}{
	\ifthenelse{\equal{\showsectioncaptionfig}{chap}}{
		\def\cftdefautnumwidthfig {3.0em}
		\def\cftdefaultnumwidthromanfig {5.25em}
	}{
		\throwbadconfig{Valor configuracion incorrecto}{\showsectioncaptionfig}{none,chap,sec,ssec,sssec,ssssec}}}}}}
	}
	
	% Tablas
	\ifthenelse{\equal{\showsectioncaptiontab}{none}}{
		\def\cftdefautnumwidthtab {3.0em} % Añade +0.7em
		\def\cftdefaultnumwidthromantab {5.25em} % Añade +0.5em
	}{
	\ifthenelse{\equal{\showsectioncaptiontab}{sec}}{
		\def\cftdefautnumwidthtab {3.7em}
		\def\cftdefaultnumwidthromantab {5.75em}
	}{
	\ifthenelse{\equal{\showsectioncaptiontab}{ssec}}{
		\def\cftdefautnumwidthtab {4.4em}
		\def\cftdefaultnumwidthromantab {6.25em}
	}{
	\ifthenelse{\equal{\showsectioncaptiontab}{sssec}}{
		\def\cftdefautnumwidthtab {5.1em}
		\def\cftdefaultnumwidthromantab {6.75em}
	}{
	\ifthenelse{\equal{\showsectioncaptiontab}{ssssec}}{
		\def\cftdefautnumwidthtab {5.8em}
		\def\cftdefaultnumwidthromantab {7.25em}
	}{
	\ifthenelse{\equal{\showsectioncaptiontab}{chap}}{
		\def\cftdefautnumwidthtab {3.0em}
		\def\cftdefaultnumwidthromantab {5.25em}
	}{
		\throwbadconfig{Valor configuracion incorrecto}{\showsectioncaptiontab}{none,chap,sec,ssec,sssec,ssssec}}}}}}
	}
	
	% Configuración identado de títulos de objetos después del número
	\def\cftfignumwidth {\cftdefautnumwidth}
	% \def\cftsubfignumwidth {\cftdefautnumwidth}
	\def\cfttabnumwidth {\cftdefautnumwidth}
	\def\cftlstlistingnumwidth {\cftdefautnumwidth}
	
	% Código fuente
	\ifthenelse{\equal{\captionnumcode}{arabic}}{ % No hace nada (default)
		\def\cftlstlistingnumwidth {\cftdefautnumwidthcode}
	}{
		\ifthenelse{\equal{\captionnumcode}{roman}}{
			\def\cftlstlistingnumwidth {\cftdefaultnumwidthromancode}
		}{
		\ifthenelse{\equal{\captionnumcode}{Roman}}{
			\def\cftlstlistingnumwidth {\cftdefaultnumwidthromancode}
		}{
			\def\cftlstlistingnumwidth {\cftdefautnumwidthcode}
		}}
	}
	
	% Figuras
	\ifthenelse{\equal{\captionnumfigure}{arabic}}{ % No hace nada (default)
		\def\cftfignumwidth {\cftdefautnumwidthfig}
	}{
		\ifthenelse{\equal{\captionnumfigure}{roman}}{
			\def\cftfignumwidth {\cftdefaultnumwidthromanfig}
		}{
			\ifthenelse{\equal{\captionnumfigure}{Roman}}{
				\def\cftfignumwidth {\cftdefaultnumwidthromanfig}
			}{
				\def\cftfignumwidth {\cftdefautnumwidthfig}
			}}
	}

	% Tablas
	\ifthenelse{\equal{\captionnumtable}{arabic}}{ % No hace nada (default)
		\def\cfttabnumwidth {\cftdefautnumwidthtab}
	}{
		\ifthenelse{\equal{\captionnumtable}{roman}}{
			\def\cfttabnumwidth {\cftdefaultnumwidthromantab}
		}{
			\ifthenelse{\equal{\captionnumtable}{Roman}}{
				\def\cfttabnumwidth {\cftdefaultnumwidthromantab}
			}{
				\def\cfttabnumwidth {\cftdefautnumwidthtab}
			}}
	}

	% -------------------------------------------------------------------------
	% Genera las funciones para los índices
	% -------------------------------------------------------------------------
	\newcommand{\coregeneratefigureindex}{
		\iftotalfigures
			\ifthenelse{\equal{\indexnewpagef}{true}}{\newpage}{}
			\listoffigures
		\fi
	}
	\newcommand{\coregeneratetableindex}{
		\iftotaltables
			\ifthenelse{\equal{\indexnewpaget}{true}}{\newpage}{}
			\listoftables
		\fi
	}
	\newcommand{\coregeneratecodeindex}{
		\iftotallstlistings
			\ifthenelse{\equal{\indexnewpagec}{true}}{\newpage}{}
			\lstlistoflistings
		\fi
	}

	% -------------------------------------------------------------------------
	% Índice de contenidos
	% -------------------------------------------------------------------------
	\ifthenelse{\equal{\showindexofcontents}{true}}{
		\tableofcontents
	}{}
	
	% -------------------------------------------------------------------------
	% Índice de objetos
	% -------------------------------------------------------------------------
	\ifthenelse{\equal{\indexstyle}{ftc}}{
		\coregeneratefigureindex
		\coregeneratetableindex
		\coregeneratecodeindex
	}{
	\ifthenelse{\equal{\indexstyle}{f}}{
		\coregeneratefigureindex
	}{
	\ifthenelse{\equal{\indexstyle}{ft}}{
		\coregeneratefigureindex
		\coregeneratetableindex
	}{
	\ifthenelse{\equal{\indexstyle}{fc}}{
		\coregeneratefigureindex
		\coregeneratecodeindex
	}{
	\ifthenelse{\equal{\indexstyle}{fct}}{
		\coregeneratefigureindex
		\coregeneratecodeindex
		\coregeneratetableindex
	}{
	\ifthenelse{\equal{\indexstyle}{t}}{
		\coregeneratetableindex
	}{
	\ifthenelse{\equal{\indexstyle}{tf}}{
		\coregeneratetableindex
		\coregeneratefigureindex
	}{
	\ifthenelse{\equal{\indexstyle}{tfc}}{
		\coregeneratetableindex
		\coregeneratefigureindex
		\coregeneratecodeindex
	}{
	\ifthenelse{\equal{\indexstyle}{tc}}{
		\coregeneratetableindex
		\coregeneratecodeindex
	}{
	\ifthenelse{\equal{\indexstyle}{tcf}}{
		\coregeneratetableindex
		\coregeneratecodeindex
		\coregeneratefigureindex
	}{
	\ifthenelse{\equal{\indexstyle}{c}}{
		\coregeneratecodeindex
	}{
	\ifthenelse{\equal{\indexstyle}{ct}}{
		\coregeneratecodeindex
		\coregeneratetableindex
	}{
	\ifthenelse{\equal{\indexstyle}{ctf}}{
		\coregeneratecodeindex
		\coregeneratetableindex
		\coregeneratefigureindex
	}{
	\ifthenelse{\equal{\indexstyle}{cf}}{
		\coregeneratecodeindex
		\coregeneratefigureindex
	}{
	\ifthenelse{\equal{\indexstyle}{cft}}{
		\coregeneratecodeindex
		\coregeneratefigureindex
		\coregeneratetableindex
	}{
	\ifthenelse{\equal{\indexstyle}{}}{
	}{
		\throwbadconfig{Estilo desconocido del indice}{\indexstyle}{,f,ft,ftc,fc,fct,t,tf,tfc,tc,tcf,c,ct,ctf,cf,cft}}}}}}}}}}}}}}}}
	}

	% -------------------------------------------------------------------------
	% Termina el bloque de índice
	% -------------------------------------------------------------------------
	\endgroup
	
	% -------------------------------------------------------------------------
	% Se añade una página en blanco
	% -------------------------------------------------------------------------
	\newpage
	\ifthenelse{\equal{\addemptypagetwosides}{true}}{
		\vfill
		\checkoddpage
		\ifoddpage
			\newpage
			\null
			\thispagestyle{empty}
			\newpage
			\addtocounter{page}{-1}
		\else
		\fi}{
	}

}{}

% END