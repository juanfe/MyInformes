% Template:     Informe/Reporte LaTeX
% Versión:      1.8.9-2 (14/05/2016)
% Codificación: UTF-8
%
% Autor: Pablo Pizarro R.
%        Facultad de Ciencias Físicas y Matemáticas.
%        Universidad de Chile.
%
% Licencia: CC BY-NC-SA 4.0 (http://creativecommons.org/licenses/by-nc-sa/4.0/)

% CREACIÓN DEL DOCUMENTO, FUENTE E IDIOMA
\documentclass[letterpaper,11pt]{article}   % Documento clase artículo
\usepackage[utf8]{inputenc}                 % Codificación UTF-8
\usepackage[T1]{fontenc}                    % Soporta caracteres acentuados
\usepackage{lmodern}                        % Tipografía moderna
\usepackage[spanish]{babel}                 % Define el idioma del documento en español
\makeatletter

% INFORMACIÓN DEL DOCUMENTO
\newcommand{\nombredelinforme}{Titulo del informe}
\newcommand{\temaatratar}{Tema a tratar}
\newcommand{\fecharealizacion}{\today}
\newcommand{\fechaentrega}{\today}

\newcommand{\nombreuniversidad}{Universidad de Chile}
\newcommand{\nombrefacultad}{Facultad de Ciencias Físicas y Matemáticas}
\newcommand{\departamentouniversidad}{Departamento de la Universidad}
\newcommand{\imagendeldepartamento}{images/departamentos/fcfm}
\newcommand{\imagendeldepartamentoescala}{0.2}
\newcommand{\localizacionuniversidad}{Santiago, Chile}

\newcommand{\nombredelcurso}{Curso}
\newcommand{\codigodelcurso}{CO-1234}

%BEGIN_FOLD
% CONFIGURACIONES
\newcommand{\tipofuentetitulo}{\huge}               % Tamaño por defecto de los títulos
\newcommand{\tipofuentesubtitulo}{\large}           % Tamaño por defecto de los subtítulos
\newcommand{\tiporeferencias}{apa}                  % Tipo de referencias
\newcommand{\nombreltformulas}{Lista de Fórmulas}   % Nombre de la lista de fórmulas
\newcommand{\nombrelttablas}{Lista de Tablas}       % Nombre de la lista de tablas
\newcommand{\nombreltfiguras}{Lista de Figuras}     % Nombre de la lista de figuras
\newcommand{\nombreltcontend}{Índice de Contenidos} % Nombre del índice de contenidos
\newcommand{\nombreltwtablas}{Tabla}                % Nombre de las tablas
\newcommand{\nombreltwfigura}{Figura}               % Nombre de las figuras
\newcommand{\defaultcaptionmargin}{3.0}             % Márgenes de las leyendas por defecto
\newcommand{\defaultpagemarginleft}{1.5}            % Margen izquierdo de las páginas [cm]
\newcommand{\defaultpagemarginright}{1.5}           % Margen derecho de las páginas [cm]
\newcommand{\defaultpagemargintop}{2.15}            % Margen superior de las páginas [cm]
\newcommand{\defaultpagemarginbottom}{2.0}          % Margen inferior de las páginas [cm]
\newcommand{\defaultfirstpagemargintop}{3.2}        % Margen superior de la portada [cm]

% LIBRERÍAS INDEPENDIENTES
\usepackage[ampersand]{easylist}    % Listas
\usepackage{amsmath}                % Fórmulas matemáticas
\usepackage{amssymb}                % Símbolos matemáticos
%\usepackage{amsthm}                % Teoremas matemáticos
\usepackage{caption}                % Leyendas
\usepackage{color}                  % Colores
\usepackage{fancyhdr}               % Encabezados y pié de páginas
\usepackage{float}                  % Administrador de posiciones de objetos
\usepackage{geometry}               % Dimensiones y geometría del documento
\usepackage{graphicx}               % Propiedades extra para los gráficos
\usepackage[hidelinks]{hyperref}    % Permite añadir enlaces y referencias
\usepackage[makeroom]{cancel}       % Cancelar términos en fórmulas
%\usepackage[version=4]{mhchem}     % Fórmulas químicas
\usepackage{multicol}               % Múltiples columnas
\usepackage{lipsum}                 % Permite crear textos dummy
\usepackage{longtable}              % Permite utilizar tablas en varias hojas
%\usepackage{listings}              % Permite añadir código fuente
\usepackage{setspace}               % Cambia el espacio entre líneas
\usepackage{subfig}                 % Permite agrupar imágenes
\usepackage{titlesec}               % Cambia el estilo de los títulos
\usepackage{url}                    % Permite añadir enlaces
\usepackage{wrapfig}                % Permite comprimir imágenes

% LIBRERÍAS DEPENDIENTES
\usepackage{epstopdf}               % Convierte archivos .eps a pdf
\usepackage{multirow}               % Añade nuevas opciones a las tablas

% DECLARACIÓN DE FUNCIONES
\newcommand{\quotes}[1]{``#1''} % Insertar cita
\newcommand{\quotesit}[1]{\textit{\quotes{#1}}} % Insertar cita itálica
\newcommand{\setcaptionmargincm}[1]{\captionsetup{margin=#1cm}} % Cambiar el margen
\newcommand{\setpagemargincm}[4]{ % Cambia márgenes de las páginas en centímetros	
	\newgeometry{left=#1cm, top=#2cm, right=#3cm, bottom=#4cm}}
\newcommand{\newpar}[1]{\hbadness=10000 #1 ~\newline \par} % Nuevo párrafo
\newcommand{\newparnl}[1]{#1 \par}      % Nuevo párrafo sin nueva linea al final
\newcommand{\lpow}[2]{{#1}_{#2}}        % Insertar sub-índice
\newcommand{\pow}[2]{{#1}^{#2}}         % Insertar elevado
\newcommand{\newtitleanum}[1]{          % Insertar un título sin número
	\addcontentsline{toc}{section}{#1}
	\hbadness=10000 \noindent \tipofuentetitulo \textbf{#1} \\ \\ \normalsize \par}
\newcommand{\newsubtitleanum}[1]{       % Insertar un subtítulo sin número
	\addcontentsline{toc}{subsection}{#1}
	\hbadness=10000 \noindent \tipofuentesubtitulo #1 \\ \\ \normalsize \par}
\newcommand{\newtitleanumnoi}[1]{       % Insertar un título sin número sin salir en el índice
	\noindent \tipofuentetitulo \textbf{#1} \\ \\ \normalsize \par}
\newcommand{\newsubtitleanumnoi}[1]{    % Insertar un subtítulo sin número sin salir en el índice
	\noindent \tipofuentesubtitulo \textbf{#1} \\ \\ \normalsize \par}
\newcommand{\insertequation}[1]{        % Insertar una ecuación
	\vspace{-0.1cm}
	\begin{mycapequ}[H]
		\begin{equation}
		#1
		\end{equation}
	\end{mycapequ}
	\vspace{-0.4cm}}
\newcommand{\insertequationcaptioned}[2]{   % Insertar una ecuación con leyenda
	\vspace{-0.2cm}
	\begin{mycapequ}[H]
		\begin{equation}
		#1
		\end{equation}
		\vspace{-0.6cm}
		\caption{#2}
	\end{mycapequ}
	\vspace{-0.25cm}}
\newcommand{\insertimage}[3]{               % Insertar una imagen
	\vspace{-0.3cm}
	\begin{figure}[H]
		\centering
		\includegraphics[scale=#2]{#1}
		\caption{#3}
	\end{figure}
	\vspace{-0.2cm}}
\newcommand{\insertimageboxed}[3]{          % Insertar una imagen con recuadro
	\vspace{-0.3cm}
	\begin{figure}[H]
		\centering
		\fbox{\includegraphics[scale=#2]{#1}}
		\caption{#3}
	\end{figure}
	\vspace{-0.2cm}}
\newcommand{\insertdoubleimage}[7]{         % Insertar una imagen doble
	\captionsetup{margin=0.5cm}
	\vspace{-0.3cm}
	\begin{figure}[H] \centering
		\subfloat[#3]{
			\includegraphics[scale=#2]{#1}}
		\subfloat[#6]{
			\includegraphics[scale=#5]{#4}}
		\caption{#7}
	\end{figure}
	\vspace{-0.2cm}
	\setcaptionmargincm{\defaultcaptionmargin}}

% DECLARACIÓN DE AMBIENTES
\captionsetup[mycapequ]{labelformat=empty}
\DeclareCaptionType{mycapequ}[][\nombreltformulas]  % Nombre del índice de fórmulas
\newenvironment{enumeratenosep}{                    % Enumerate sin separación
	\begin{enumerate}
		\setlength{\itemsep}{0pt}
		\setlength{\parskip}{0pt}
	\end{enumerate}}
\makeatletter
\renewenvironment{thebibliography}[1] % Bibliografía en 2 columnas
{\begin{multicols}{2}[\section*{\refname}]%
		\@mkboth{\MakeUppercase\refname}{\MakeUppercase\refname}%
		\list{\@biblabel{\@arabic\c@enumiv}}%
		{\settowidth\labelwidth{\@biblabel{#1}}%
			\leftmargin\labelwidth
			\advance\leftmargin\labelsep
			\@openbib@code
			\usecounter{enumiv}%
			\let\p@enumiv\@empty
			\renewcommand\theenumiv{\@arabic\c@enumiv}}%
		\sloppy
		\clubpenalty4000
		\@clubpenalty \clubpenalty
		\widowpenalty4000%
		\sfcode`\.\@m}
	{\def\@noitemerr
		{\@latex@warning{Empty `thebibliography' environment}}%
		\endlist\end{multicols}}
\makeatother

% CONFIGURACIÓN INICIAL DEL DOCUMENTO
\setlength{\headheight}{54pt}
\bibliographystyle{\tiporeferencias}        % Estilo APA para las referencias
\setcaptionmargincm{\defaultcaptionmargin}  % Margen por defecto
%\captionsetup[table]{font={stretch=0.2}}   % Fuente de las tablas
\ListProperties(                            % Propiedades de las listas
	Hide=100,
	Hang=true,
	Progressive=3ex,
	Style*=-- ,
	Style2*=$\bullet$,
	Style3*=$\circ$,
	Style4*=\tiny$\blacksquare$)
% LENGUAJE C
%\lstset{
%	language=C,
%  	numbers=left,
%  	stepnumber=1,        
%  	numbersep=5pt,
%  	backgroundcolor=\color{white}, 
%  	showspaces=false,
%  	showstringspaces=false,
%  	showtabs=false,
%  	tabsize=2,
%  	captionpos=b,
%  	breaklines=true,
%  	breakatwhitespace=true,
%  	title=\lstname,
%}
% LENGUAJE JAVA
%\lstset{
%	frame=tb,
%   language=Java,
%   aboveskip=3mm,
%   belowskip=3mm,
%   showstringspaces=false,
%   columns=flexible,
%   basicstyle={\small\ttfamily},
%   numbers=none,
%   numberstyle=\tiny\color{gray},
%   keywordstyle=\color{blue},
%   commentstyle=\color{dkgreen},
%   stringstyle=\color{mauve},
%   breaklines=true,
%   breakatwhitespace=true,
%   tabsize=3
%}
%END_FOLD

% INICIO DEL DOCUMENTO
\begin{document}

%BEGIN_FOLD
% PORTADA
\newpage
\setpagemargincm{\defaultpagemarginleft}{\defaultfirstpagemargintop}{\defaultpagemarginright}{\defaultpagemarginbottom}
\pagestyle{fancy}
\fancyhf{}
\fancyhead[L] {\nombreuniversidad \\ \nombrefacultad \\ \departamentouniversidad}
\fancyhead[R]{\includegraphics[scale=\imagendeldepartamentoescala]{\imagendeldepartamento}}
\vspace*{5cm}
\begin{center}
	\huge  {\nombredelcurso}\\
	\vspace{1cm}
	\Huge {\nombredelinforme}\\
	\vspace{0.3cm}
	\large {\temaatratar} \\
\end{center}
\vfill

% INTEGRANTES, PROFESORES Y FECHAS
\begin{minipage}{0.965\textwidth}
	\begin{flushright}
		\begin{tabular}{ll}
			Integrantes: 
				& \begin{tabular}[t]{@{}l@{}}
					Integrante 1 \\
					Integrante 2 \\
					Integrante 3
				\end{tabular} \\
			Profesores: 
				& \begin{tabular}[t]{@{}l@{}}
					Profesor 1 \\
					Profesor 2 \\
					Profesor 3
				\end{tabular} \\
			Ayudantes: 
				& \begin{tabular}[t]{@{}l@{}}
					Ayudante 1 \\
					Ayudante 2 \\
					Ayudante 3
				\end{tabular}\\
			\multicolumn{2}{l}{Ayudante del laboratorio: Ayudante} \\
			& \\
			\multicolumn{2}{l}{Fecha de realización: \fecharealizacion} \\
			\multicolumn{2}{l}{Fecha de entrega: \fechaentrega} \\
			\multicolumn{2}{l}{\localizacionuniversidad}
		\end{tabular}
	\end{flushright}
\end{minipage}

% CONFIGURACIÓN DE PÁGINA Y ENCABEZADOS
\newpage
\setpagemargincm{\defaultpagemarginleft}{\defaultpagemargintop}{\defaultpagemarginright}{\defaultpagemarginbottom}
\renewcommand{\listfigurename}{\nombreltfiguras}    % Nombre del índice de figuras
\renewcommand{\listtablename}{\nombrelttablas}      % Nombre del índice de tablas
\renewcommand{\contentsname}{\nombreltcontend}      % Nombre del índice
\renewcommand{\tablename}{\nombreltwtablas}         % Nombre de la leyenda de las tablas
\renewcommand{\figurename}{\nombreltwfigura}        % Nombre de la leyenda de las figuras
\pagestyle{fancy}
\fancyhf{}
\fancyhead[R]{\small \rm \thepage}
\fancyfoot[L]{\small \rm \textit{\nombredelinforme}}
\fancyfoot[R]{\small \rm \textit{\codigodelcurso \ \nombredelcurso}}
\renewcommand{\sectionmark}[1]{\markright{\thesection.\ #1}}
\renewcommand{\headrulewidth}{0.5pt} % Ancho de la barra del header
\renewcommand{\footrulewidth}{0.5pt} % Ancho de la barra del footer

% PREÁMBULO DEL DOCUMENTO
\setcounter{page}{1} % Nueva indización
\renewcommand{\thepage}{\roman{page}} % Se establece la numeración del tipo romano

% ABSTRACT
\newtitleanum{Abstract}
\lipsum[1]

% TABLA DE CONTENIDOS
\newpage
\tableofcontents        % Tabla de contenidos
\listoffigures          % Índice de figuras
\listoftables           % Índice de tablas
\listofmycapequs        % Índice de fórmulas

% CONFIGURACIONES FINALES - INICIO DE LAS SECCIONES
\newpage
\fancyhead[L]{\small \rm \textit{\rightmark}}               % Configuración del header
\titleformat*{\section}{\tipofuentetitulo\bfseries}         % Tamaño de los títulos
\titleformat*{\subsection}{\tipofuentesubtitulo\bfseries}   % Tamaño de los títulos
\renewcommand{\thepage}{\arabic{page}}
\setcounter{page}{1}
%END_FOLD

% NUEVA SECCIÓN
\section{Informes con \LaTeX}

	% SUB-SECCIÓN
	\subsection{Una breve introducción}
	
		\lipsum[1]
		
		\insertequation{\pow{a}{k}=\pow{b}{k}+\pow{c}{k}, \forall k>2}
		
		\newpar{Este es un párrafo, puede contener múltiples \quotes{Expresiones} así como \quotesit{Citas en itálico}, los párrafos añaden una entrada libre por defecto, a continuación se muestra un ejemplo de inserción de imágenes:}
		
		\insertimage{images/image.png}{0.2}{Where are you? de \quotesit{Internet}}
		
		\newparnl{Este es un párrafo sin nueva linea (de ahí el \textit{nl}, esto se puede usar para terminar un tema, o si vienen imágenes nuevas), si no te gustan los comandos \textbf{newpar} o \textbf{newparnl} simplemente puedes usar los salto de línea convencionales. Además puedes cambiarle el nombre a las funciones, así puedes tener comandos más intuitivos para tí.}
		
	% SUB-SECCIÓN
	\subsection{Tablas!}

		\newparnl{También puedes usar tablas, insertarlas es muy fácil, puedes usar directamente el \quotes{conversor de tablas} \textsuperscript{\cite{conversortabla}}, ahí puedes convertir en un solo clic tablas Excel, o crearlas tú mismo sin tener que hacer todo el aburridísimo código.}
		
		\begin{longtable}{ccc}
			\caption{Esta es una tabla que se \quotes{corta} en varias páginas si es que le falta espacio, útil para cuando tienes que pegar tablas con más de 60 o 1000? filas.}\label{foo}\\
			\hline
			Columna 1 & Columna 2 & Columna 3\\\hline
			\endfirsthead
			\hline
			Columna 1 & Columna 2 & Columna 3\\
			\hline
			\endhead
			\hline
			\endfoot
			\hline
			\endlastfoot
			$\omega$ & $\nu$ & $\delta$\\     
			$\partial$ & $\nabla$ & $\mho$\\
			$\beta$ & $\gamma$ & $\epsilon$\\   
			$\varepsilon$ & $\upsilon$ & $\varphi$\\
			$\Phi$ & $\Theta$ & $\varSigma$\\
			$\omega$ & $\nu$ & $\delta$\\     
			$\partial$ & $\nabla$ & $\mho$\\ 
		\end{longtable}
		
		\newparnl{También puedes usar tablas muy aburridas (ver Tabla \ref{tablafome}), puedes hacer de todo en \LaTeX.}
		
		\begin{table}[H]
			\centering
			\caption{También puedes usar tablas así, aburridas!}
			\label{tablafome}
			\begin{tabular}{|c|c|c|c|c|c|c|c|c|c|c|c|c|}
				\hline
				Ítems & ${V}_{1}$ & ${V}_{2}$ & ${V}_{3}$ & ${V}_{4}$ & ${V}_{5}$ & ${V}_{6}$ & ${V}_{7}$ & ${V}_{8}$ & ${V}_{9}$ & ${V}_{10}$ & ${V}_{11}$ & ${V}_{12}$ \\ \hline
				${X}_{1}$             & 0  & 0  & 2  & 0  & 0  & 0  & 0  & -1 & 0  & 0   & 0   & 1   \\ \hline
				${X}_{2}$             & 0  & -2 & 1  & 0  & -1 & 0  & 0  & -1 & 0  & 0   & -1  & 1   \\ \hline
				${X}_{3}$             & 2  & 0  & 2  & 0  & 0  & 1  & 1  & 0  & 1  & 0   & 1   & 0   \\ \hline
				${X}_{4}$             & -1 & -1 & 0  & 0  & -1 & -1 & 0  & 0  & -1 & 0   & -1  & -1  \\ \hline
				${X}_{5}$             & -1 & -1 & 0  & 0  & -1 & 0  & 0  & 0  & 0  & 0   & -1  & 0   \\ \hline
				${X}_{6}$             & -1 & -1 & 0  & 0  & -1 & -1 & -1 & -1 & 0  & 0   & -1  & -1  \\ \hline
				${X}_{7}$             & 1  & 0  & 1  & 0  & 0  & 0  & 0  & 0  & 1  & 0   & 0   & 0   \\ \hline
				${X}_{8}$             & 1  & -1 & 2  & 0  & 0  & 0  & 0  & -1 & 0  & 0   & 0   & 0   \\ \hline
				${X}_{9}$             & 0  & -1 & 0  & 0  & 0  & 0  & 0  & 0  & 0  & 0   & 0   & 0   \\ \hline
				${X}_{10}$            & 1  & 0  & 2  & 0  & 0  & 0  & 0  & -1 & 0  & 0   & 0   & -1  \\ \hline
				${X}_{11}$            & -1 & -2 & -1 & 0  & -2 & 0  & 0  & -1 & -1 & 0   & -2  & -1  \\ \hline
				${X}_{12}$            & -1 & 0  & 1  & 0  & 0  & 0  & 0  & -1 & 1  & 0   & 0   & 0   \\ \hline
			\end{tabular}
		\end{table}
		
	% SUB-SECCIÓN
	\subsection{Relleno}
		
		\insertequationcaptioned{\int_{a}^{b} f(x) dx = \textstyle \sum_{x=a}^{b} f(x)\cancelto{1+\frac{\epsilon}{k}}{(1+\Delta x)}}{Ecuación sin sentido}
		
		\newpar{Test es una palabra inglesa aceptada por la Real Academia Española (RAE). Este concepto hace referencia a las pruebas destinadas a evaluar conocimientos, aptitudes o funciones. La palabra test puede utilizarse como sinónimo de examen. Los exámenes son muy frecuentes en el ámbito educativo ya que permiten evaluar los conocimientos adquiridos por los estudiantes. Los exámenes pueden ser orales o escritos, con preguntas de respuestas abiertas (donde el estudiante responde libremente) o preguntas de respuestas múltiples (el estudiante debe seleccionar la respuesta correcta de un listado).}

% REFERENCIAS
\begin{thebibliography}{99}
	\addcontentsline{toc}{section}{Referencias}
	
	\bibitem{pjd} 
		Pedro, Juan y Diego. 
		\textit{Como hacer informes en \LaTeX\.}. 
		Universidad de Chile, Facultad de Ciencias Físicas y Matemáticas, 2016.
	
	\bibitem{einstein}
		\hbadness=10000 El mismísimo Albert Einstein. 
		\textit{Otro titulo complicado e interesante}. 
		Análisis de físicas de informes, 322(10):891–921, 1905.
	
	\bibitem{ucursos} 
		También puedes agregar enlaces.
		\textit{U-Cursos al fin conoció \LaTeX.}
		2016.
		\\\url{https://www.u-cursos.cl/}
	
	\bibitem{conversortabla}
		Tables Generator.
		\textit{Convierte fácilmente tus tablas, o crea unas con un intuitivo editor de tablas.}
		\\\url{http://www.tablesgenerator.com/}
	
\end{thebibliography}

% FIN DEL DOCUMENTO
\end{document}
