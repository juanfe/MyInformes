% Template:     Informe/Reporte LaTeX
% Documento:    Archivo de testeo
% Versión:      4.0.7 (04/07/2017)
% Codificación: UTF-8
%
% Autor: Pablo Pizarro R.
%        Facultad de Ciencias Físicas y Matemáticas
%        Universidad de Chile
%        pablo.pizarro@ing.uchile.cl, ppizarror.com
%
% Manual template: [http://ppizarror.com/Template-Informe/]
% Licencia MIT:    [https://opensource.org/licenses/MIT/]

\section{Par test}

% Table generated by Excel2LaTeX from sheet 'Hoja1'
\begin{table}[htbp]
	\centering
	\caption{Add caption}
	\itemresize{1}
	{\begin{tabular}{|c|c|c|c|}
			\hline
			a     & b     & c     & d \bigstrut\\
			\hline
			1     & 2     & 2     & 4 \bigstrut\\
			\hline
			5     & 6     & 7     & 8 \bigstrut\\
			\hline
		\end{tabular}%
	}
	\label{tab:addlabel2}%
\end{table}%

Test:\\
$Hola$\\
$Arbol$\\


\insertimage{ejemplos/test-image}{width=10cm}{Pellentesque interdum sapien sed nulla. Proin tincidunt. Aliquam volutpat est vel
	massa. Sed dolor lacus, imperdiet non, ornare non, commodo eu, neque. Integer pretium semper
	justo. Proin risus. Nullam id quam. Nam neque. Duis vitae wisi ullamcorper diam congue
	ultricies. Quisque ligula. Mauris vehicula.}

% Table generated by Excel2LaTeX from sheet 'Hoja1'
\begin{table}[htbp]
	\centering
	\caption{Pellentesque interdum sapien sed nulla. Proin tincidunt. Aliquam volutpat est vel
		massa. Sed dolor lacus, imperdiet non, ornare non, commodo eu, neque. Integer pretium semper
		justo. Proin risus. Nullam id quam. Nam neque. Duis vitae wisi ullamcorper diam congue
		ultricies. Quisque ligula. Mauris vehicula.}
	\begin{tabular}{|c|c|c|c|}
		\hline
		a     & b     & c     & d \bigstrut\\
		\hline
		1     & 2     & 2     & 4 \bigstrut\\
		\hline
		5     & 6     & 7     & 8 \bigstrut\\
		\hline
	\end{tabular}%
	\label{tab:addlabel}%
\end{table}%

$\degree$


\insertequationanum{a}

\insertequationanum{a}
\insertequationanum{a}
\insertequationanum{a}
\insertequationanum{a}

\section{Sectiontest}
\subsection{Test}\subsection{Test}\subsection{Test}\subsection{Test}\subsection{Test}\subsection{Test}\subsection{Test}\subsection{Test}\subsection{Test}\subsection{Test}\subsection{Test}\subsection{Test}\subsection{Test}\subsection{Test}\subsection{Test}\subsection{Test}\subsection{Test}\subsection{Test}\subsection{Test}\subsection{Test}\subsection{Test}\subsection{Test}\subsection{Test}

\lipsum[6]
\insertequation[]{a+b}

\lipsum[6]
\insertequationcaptioned[]{a}{b}

\lipsum[6]
\insertgather{b\\c}
\insertgathercaptioned{b\\c}{d}

\lipsum[6]
\insertalign{b\\c}
\insertaligncaptioned{arg2}{arg3}
\insertaligncaptioned{arg2}{}

\lipsum[4]

\insertequation[]{3\gamma c}

% Se prueban equation
\insertequation[\label{arg1}]{arg2}
\insertequationanum{arg1}
\insertequationcaptioned[\label{arg3}]{arg2}{arg3}
\insertequationcaptionedanum{arg1}{arg2}
\insertequationcaptioned[\label{arg2}]{arg2}{}
\insertequationcaptionedanum{arg1}{}

% Pruebas con align
\newpage
\section{Align}
Linea 1, pruebas con lorem ipsum arbol tierra piedra.

\insertequationanum{arg1}

Linea 1, pruebas con lorem ipsum arbol tierra piedra.

\insertequation[]{arg2}

Linea 1, pruebas con lorem ipsum arbol tierra piedra.

\insertalign{arg2 \\ arg 4}

Linea 1, pruebas con lorem ipsum arbol tierra piedra.

\insertalignanum{arg2 \\ arg 4}

Linea 1, pruebas con lorem ipsum arbol tierra piedra.

\insertaligncaptioned{arg2 \\ \sum_{i}^{j}\frac{a}{b}}{arg3}

Linea 1, pruebas con lorem ipsum arbol tierra piedra.

\insertaligncaptionedanum{arg2 \\ arg 4}{arg2}

Linea 1, pruebas con lorem ipsum arbol tierra piedra.

\insertaligncaptioned{arg2 \\ arg 4}{}

Linea 1, pruebas con lorem ipsum arbol tierra piedra.

\insertaligncaptionedanum{arg2 \\ arg 4}{}

Linea 1, pruebas con lorem ipsum arbol tierra piedra.

\newpage
\insertemptypage
\section{Aligned}
Linea 1, pruebas con lorem ipsum arbol tierra piedra.

\insertequationanum{arg1}

Linea 1, pruebas con lorem ipsum arbol tierra piedra.

\insertequation[]{arg2}

Linea 1, pruebas con lorem ipsum arbol tierra piedra.

\insertaligned[\label{aligned1}]{arg2 \\ arg4}

Linea 1, pruebas con lorem ipsum arbol tierra piedra \eqref{aligned1}.

\insertalignedanum{arg2 \\ arg4}

Linea 1, pruebas con lorem ipsum arbol tierra piedra.

\insertalignedcaptioned[\label{aligned2}]{arg2 \\ arg4}{arg3}

Linea 1, pruebas con lorem ipsum arbol tierra piedra.

\insertalignedcaptionedanum{arg2 \\ arg4}{arg2}

Linea 1, pruebas con lorem ipsum arbol tierra piedra.

\insertalignedcaptioned[\label{aligned3}]{arg2 \\ arg4}{}

Linea 1, pruebas con lorem ipsum arbol tierra piedra.

\insertalignedcaptionedanum{arg2 \\ arg4}{}

Linea 1, pruebas con lorem ipsum arbol tierra piedra.

\newpage
\section{Equation}
Linea 1, pruebas con lorem ipsum arbol tierra piedra.

\insertequation[\label{eq1}]{arg2}

Linea 1, pruebas con lorem ipsum arbol tierra piedra.

\insertequationanum{arg1}

Linea 1, pruebas con lorem ipsum arbol tierra piedra.

\insertequationcaptioned[\label{eq2}]{arg2}{arg3}

Linea 1, pruebas con lorem ipsum arbol tierra piedra.

\insertequationcaptionedanum{arg1}{arg2}

Linea 1, pruebas con lorem ipsum arbol tierra piedra.

\insertequationcaptioned[\label{eq3}]{arg2}{}

Linea 1, pruebas con lorem ipsum arbol tierra piedra.

\insertequationcaptionedanum{arg1}{}

Linea 1, pruebas con lorem ipsum arbol tierra piedra.

\newpage
\section{Gather}

Linea 1, pruebas con lorem ipsum arbol tierra piedra.

\insertequationanum{arg1}

Linea 1, pruebas con lorem ipsum arbol tierra piedra.

\insertequation[]{arg2}

Linea 1, pruebas con lorem ipsum arbol tierra piedra.

\insertgather{arg2 \\ arg 4}

Linea 1, pruebas con lorem ipsum arbol tierra piedra.

\insertgatheranum{arg2 \\ arg 4}

Linea 1, pruebas con lorem ipsum arbol tierra piedra.

\insertgathercaptioned{arg2 \\ arg 4}{arg3}

Linea 1, pruebas con lorem ipsum arbol tierra piedra.

\insertgathercaptionedanum{arg2 \\ arg 4}{arg2}

Linea 1, pruebas con lorem ipsum arbol tierra piedra.

\insertgathercaptioned{arg2 \\ arg 4}{}

Linea 1, pruebas con lorem ipsum arbol tierra piedra.

\insertgathercaptionedanum{arg2 \\ arg 4}{}

Linea 1, pruebas con lorem ipsum arbol tierra piedra.

\newpage
\section{Gathered}

Linea 1, pruebas con lorem ipsum arbol tierra piedra.

\insertequationanum{arg1}

Linea 1, pruebas con lorem ipsum arbol tierra piedra.

\insertequation[]{arg2}

Linea 1, pruebas con lorem ipsum arbol tierra piedra.

\insertgathered[\label{gathered1}]{arg2 \\ arg 4}

Linea 1, pruebas con lorem ipsum arbol tierra piedra \eqref{gathered1}.

\insertgatheredanum{arg1}

Linea 1, pruebas con lorem ipsum arbol tierra piedra.

\insertgatheredcaptioned[\label{gathered2}]{arg2}{arg3}

Linea 1, pruebas con lorem ipsum arbol tierra piedra.

\insertgatheredcaptionedanum{arg1}{arg2}

Linea 1, pruebas con lorem ipsum arbol tierra piedra.

\insertgatheredcaptioned[\label{gathered3}]{arg2}{}

Linea 1, pruebas con lorem ipsum arbol tierra piedra.

\insertgatheredcaptionedanum{arg1}{}

Linea 1, pruebas con lorem ipsum arbol tierra piedra.

\newpage
\fracdderivat{a}{b}
$\fracderivat{a}{b}$

\newpage

\insertdoubleeqimage[\label{img-equal}]{ejemplos/test-image}{Descripción A}
{ejemplos/test-image-wrap}{Descripción B}{height=5cm}{Descripción general de las imágenes}

\insertdoubleeqimage[\label{img-equal2}]{ejemplos/test-image}{Descripción A}
{ejemplos/test-image-wrap}{Descripción B}{height=5cm}{Descripción general de las imágenes}

\insertdoubleeqimage[\label{img-equal3}]{ejemplos/test-image}{Descripción A}
{ejemplos/test-image-wrap}{Descripción B}{height=5cm}{Descripción general de las imágenes}

\inserttripleimage[\label{img-triple-b}]{ejemplos/test-image}{scale=0.4,angle=90}{ejemplos/test-image}{height=6cm}{ejemplos/test-image}{width=3cm}{Descripción general}

\inserttripleimage[\label{img-triple-1}]{ejemplos/test-image}{scale=0.4}{ejemplos/test-image}{height=6cm}{ejemplos/test-image}{width=3cm}{Descripción general}


% Nueva pagina
\newpage

\sectionanum{Test sin numerar}

\begin{lstlisting}[style=Matlab, caption={Ejemplo en Matlab.\label{codigo-matlab}}]
% Se crea gráfico
f = figure(1); hold on; movegui(f, 'center');
xlabel('td/Tn'); ylabel('FAD=Umax/Uf0');
title('Espectro de pulso de desplazamiento');

for j = 1:length(BETA)
fad = ones(1, NDATOS); % Arreglo para el FAD, uno para cada r (o td/Tn)

% Se crea el espectro de respuesta máximo para cada par de beta/r
for i = 1:NDATOS
[t, u_t, ~, ~] = main(BETA(j), r(i), M, K, F0, 0);
fad(i) = max(abs(u_t)) / uf0;
end
mx = find(fad == max(fad(:)));
fprintf('BETA=%.2f, MAX: FAD=%.3f, TD/TN=%.3f\n', BETA(j), fad(mx), tdtn(mx));
plot(tdtn, fad, 'DisplayName', strcat('\beta=', sprintf('%.2f', BETA(j))));
end
\end{lstlisting}

\subsection{Hola}