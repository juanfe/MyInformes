% Template:     Informe/Reporte LaTeX
% Documento:    Archivo principal
% Versión:      3.8.8 (01/06/2017)
% Codificación: UTF-8
%
% Autor: Pablo Pizarro R.
%        Facultad de Ciencias Físicas y Matemáticas
%        Universidad de Chile
%        pablo.pizarro@ing.uchile.cl, ppizarror.com
%
% Manual template: [http://ppizarror.com/Template-Informe/]
% Licencia MIT:    [https://opensource.org/licenses/MIT/]

% CREACIÓN DEL DOCUMENTO, FUENTE E IDIOMA
\documentclass[letterpaper,11pt]{article} % Articulo tamaño carta, 11pt
\usepackage[utf8]{inputenc}               % Codificación UTF-8
\def\templateversion{3.8.8}               % Versión del template

% INFORMACIÓN DEL DOCUMENTO
\def\nombredelinforme {Título del informe}
\def\temaatratar {Tema a tratar}

\def\autordeldocumento {Nombre del autor o grupo}
\def\nombredelcurso {Curso}
\def\codigodelcurso {CO-1234}

\def\nombreuniversidad {Universidad de Chile}
\def\nombrefacultad {Facultad de Ciencias Físicas y Matemáticas}
\def\departamentouniversidad {Departamento de la Universidad}
\def\imagendepartamento {images/departamentos/fcfm}
\def\imagendepartamentoescala {0.2}
\def\localizacionuniversidad {Santiago, Chile}

% INTEGRANTES, PROFESORES Y FECHAS
\def\tablaintegrantes {
\begin{minipage}{0.976\textwidth}
\begin{flushright}
\begin{tabular}{ll}
	Integrantes:
		& \begin{tabular}[t]{@{}l@{}}
			Integrante 1 \\
			Integrante 2
		\end{tabular} \\
	Profesores:
		& \begin{tabular}[t]{@{}l@{}}
			Profesor 1 \\
			Profesor 2
		\end{tabular} \\
	Auxiliares:
		& \begin{tabular}[t]{@{}l@{}}
			Auxiliar 1 \\
			Auxiliar 2
		\end{tabular} \\
	Ayudantes:
		& \begin{tabular}[t]{@{}l@{}}
			Ayudante 1 \\
			Ayudante 2
		\end{tabular} \\
	\multicolumn{2}{l}{Ayudante del laboratorio: Ayudante 1} \\
	& \\
	\multicolumn{2}{l}{Fecha de realización: \today} \\
	\multicolumn{2}{l}{Fecha de entrega: \today} \\
	\multicolumn{2}{l}{\localizacionuniversidad}
\end{tabular}
\end{flushright}
\end{minipage}}

% CONFIGURACIONES
% Template:     Informe/Reporte LaTeX
% Documento:    Archivo de configuraciones
% Versión:      3.5.3 (13/05/2017)
% Codificación: UTF-8
%
% Autor: Pablo Pizarro R.
%        Facultad de Ciencias Físicas y Matemáticas
%        Universidad de Chile
%        pablo.pizarro@ing.uchile.cl, ppizarror.com
%
% Sitio web del proyecto: [http://ppizarror.com/Template-Informe/]
% Licencia: MIT           [https://opensource.org/licenses/MIT]

% CONFIGURACIONES GENERALES
\def\defaultimagefolder {images/}  % Carpeta de las imágenes
\def\defaultinterline {1.0}        % Interlineado por defecto
\def\defaultnewlinesize {11pt}     % Tamaño del salto de línea
\def\showborderonlinks {false}     % Muestra un recuadro en cada enlace
\def\showdotontitles {true}        % Punto al final de cada título/subtítulo/etc
\def\tablepadding {1.0}            % Padding de las tablas

% CONFIGURACIÓN DE LAS LEYENDAS
\def\captionbottommargin {4}       % Margen inferior de las leyendas [pt]
\def\captiontopmargin {9}          % Margen superior de las leyendas [pt]
\def\centeredcaption {false}       % Leyenda centrada al tener varias líneas
\def\captionlessmargin {0.1cm}     % Margen sup/inf. de fig. si no hay leyenda
\def\captionlrmargin {2.7}         % Márgenes izq/der de las leyendas [cm]
\def\figurecaptiontop {false}      % Leyenda arriba de las imágenes
\def\showsectiononcaption {false}  % Muestra el número de sección en las leyendas

% CONFIGURACIÓN DEL ÍNDICE
\def\indextitlemargin {7pt}        % Margen títulos en índice
\def\indexdepth {3}                % Profundidad del índice
\def\showindex {true}              % Muestra el índice
\def\showindexofcontents {true}    % Muestra la lista de contenidos
\def\showindexoffigures {true}     % Muestra la lista de figuras
\def\showindexofsourcecode {false} % Muestra la lista de códigos fuente
\def\showindexoftables {true}      % Muestra la lista de tablas

% COLORES DEL DOCUMENTO
\def\citecolor {black}             % Color del número de las referencias o citas
\def\maintextcolor {black}         % Color principal del texto
\def\linkcolor {black}             % Color de los enlaces del documento
\def\urlcolor {black}              % Color de los url (insertados con \url o \href)

% MÁRGENES DE FIGURAS
\def\marginbottomimages {-0.2cm}   % Margen inferior figura
\def\marginfloatimages {-13pt}     % Margen superior figura flotante
\def\margintopimages {0.0cm}       % Margen superior figura

% REFERENCIAS
\def\referenceassection {false}    % Considera las referencias como una sección
\def\twocolumnreferences {false}   % Referencias en dos columnas
\def\typereference {apa}           % Tipo de referencias

% CONFIGURACIÓN PORTADA Y HEADERS
\def\gradecodeonportrait {false}   % Muestra el código del curso
\def\showfooter {true}             % Muestra el footer
\def\showheadertitle {true}        % Muestra título de la sección en el header

% MÁRGENES DE PÁGINA
\def\firstpagemargintop {3.8}      % Margen superior página portada [cm]
\def\pagemarginbottom {2.7}        % Margen inferior página [cm]
\def\pagemarginleft {2.54}         % Margen izquierdo página [cm]
\def\pagemarginright {2.54}        % Margen derecho página [cm]
\def\pagemargintop {3.0}           % Margen superior página [cm]

% ESTILO DE TÍTULOS
\def\etypefontsubsubtitle {\bfseries} % Estilo sub-subtítulos
\def\etypefontsubsubtitlei{\bfseries} % Estilo sub-subtítulos en el índice
\def\etypefontsubtitle {\bfseries}    % Estilo subtítulos
\def\etypefontsubtitlei {\bfseries}   % Estilo subtítulos en el índice
\def\etypefonttitle {\bfseries}       % Estilo títulos
\def\etypefonttitlei {\bfseries}      % Estilo títulos en el índice
\def\typefontsubsubtitle {\large}     % Tamaño sub-subtítulos
\def\typefontsubsubtitlei {\large}    % Tamaño sub-subtítulos en el índice
\def\typefontsubtitle {\Large}        % Tamaño subtítulos
\def\typefontsubtitlei {\Large}       % Tamaño subtítulos en el índice
\def\typefonttitle {\huge}            % Tamaño títulos
\def\typefonttitlei {\huge}           % Tamaño títulos en el índice

% NOMBRE DE OBJETOS
\def\nameportraitpage {Portada}       % Etiqueta página de la portada
\def\namereferences {Referencias}     % Nombre de la sección de referencias
\def\nomltcont{Índice de Contenidos}  % Nombre del índice de contenidos
\def\nomltfigure {Lista de Figuras}   % Nombre del índice de la lista de figuras
\def\nomltsrc{Lista de Código Fuente} % Nombre del índice de la lista de código
\def\nomlttable {Lista de Tablas}     % Nombre del índice de la lista de tablas
\def\nomltwsrc {Código}               % Etiqueta leyenda del código fuente
\def\nomltwfigure {Figura}            % Etiqueta leyenda de las figuras
\def\nomltwtable {Tabla}              % Etiqueta leyenda de las tablas

% IMPORTACIÓN DE LIBRERÍAS
% Template:     Informe/Reporte LaTeX
% Documento:    Importación de librerías
% Versión:      3.7.3 (21/05/2017)
% Codificación: UTF-8
%
% Autor: Pablo Pizarro R.
%        Facultad de Ciencias Físicas y Matemáticas
%        Universidad de Chile
%        pablo.pizarro@ing.uchile.cl, ppizarror.com
%
% Manual template: [http://ppizarror.com/Template-Informe/]
% Licencia MIT:    [https://opensource.org/licenses/MIT/]

% LIBRERÍAS DEL NÚCLEO
\usepackage[spanish,es-nosectiondot,es-lcroman]{babel} % Idioma
\usepackage{ifthen}    % Permite el manejo de condicionales

% LIBRERÍAS INDEPENDIENTES
\usepackage{array}     % Nuevas características a las tablas
\usepackage{amsmath}   % Librerías matemáticas
\usepackage{amssymb}   % Librerías matemáticas
\usepackage{bigstrut}  % Líneas horizontales en tablas
\usepackage{booktabs}  % Permite manejar elem. visuales en tablas
\usepackage{caption}   % Leyendas
\usepackage{chngcntr}  % Añade números a las leyendas
\usepackage{colortbl}  % Administración de color en tablas
\usepackage{color}     % Colores
\usepackage{datetime}  % Fechas
\usepackage{fancyhdr}  % Encabezados y pié de páginas
\usepackage{floatrow}  % Administración de floats
\usepackage{gensymb}   % Simbología común
\usepackage{geometry}  % Dimensiones y geometría del documento
\usepackage{graphicx}  % Propiedades extra para los gráficos
\usepackage{lipsum}    % Permite crear textos dummy
\usepackage{listings}  % Permite añadir código fuente
\usepackage{longtable} % Permite utilizar tablas en varias hojas
\usepackage{mathtools} % Permite utilizar notaciones matemáticas
\usepackage{multicol}  % Múltiples columnas
\usepackage{needspace} % Maneja los espacios de página
\usepackage{notoccite} % Desactiva las citas en el índice
\usepackage{pdfpages}  % Permite administrar páginas en pdf
\usepackage{rotating}  % Permite rotación de objetos
\usepackage{sectsty}   % Cambia el estilo de los títulos
\usepackage{selinput}  % Compatibilidad con acentos
\usepackage{setspace}  % Cambia el espacio entre líneas
\usepackage{siunitx}   % Unidades en latex
\usepackage{soul}      % Permite subrayar texto
\usepackage{subfig}    % Permite agrupar imágenes
\usepackage{textcomp}  % Simbología común
\usepackage{ulem}      % Permite tachar, subrayar, etc
\usepackage{url}       % Permite añadir enlaces
\usepackage{wasysym}   % Contiene caracteres misceláneos
\usepackage{wrapfig}   % Permite comprimir imágenes
\usepackage{xspace}    % Adminsitra espacios en párrafos y líneas

% LIBRERÍAS CON PARÁMETROS
\usepackage[makeroom]{cancel} % Cancelar términos en fórmulas
\usepackage[inline]{enumitem} % Permite enumerar ítems
\usepackage[bottom,norule,hang]{footmisc} % Estilo pié de página
\usepackage[pdfencoding=auto,psdextra]{hyperref} % Enlaces, referencias
\usepackage[figure,table,lstlisting]{totalcount} % Contador de objetos
\usepackage[usenames,dvipsnames]{xcolor} % Paquete de colores avanzado

% LIBRERÍAS DEPENDIENTES
\usepackage{epstopdf}  % Convierte archivos .eps a .pdf
\usepackage{float}     % Administrador de posiciones de objetos
\usepackage{multirow}  % Agrega nuevas opciones a las tablas

% LIBRERÍAS CONDICIONALES
\ifthenelse{           % Agrega puntos a los títulos/subtítulos
	\equal{\showdotontitles}{true}}{
	\usepackage{secdot}
	\sectiondot{subsection}
	\sectiondot{subsubsection}}{
}

% IMPORTACIÓN DE FUNCIONES
% Template:     Informe/Reporte LaTeX
% Documento:    Definición de funciones
% Versión:      3.0.2 (15/04/2017)
% Codificación: UTF-8
%
% Autor: Pablo Pizarro R.
%        Facultad de Ciencias Físicas y Matemáticas.
%        Universidad de Chile.
%        pablo.pizarro@ing.uchile.cl, ppizarror.com
%
% Sitio web del proyecto: [http://ppizarror.com/Template-Informe/]
% Licencia: MIT           [https://opensource.org/licenses/MIT]

\newcommand{\throwerror}[2]{
	% Lanza un mensaje de error
	% 	#1	Función del error
	%	#2	Mensaje
	\errmessage{Error: \noexpand#1 #2 (linea \the\inputlineno)}
}
\newcommand{\emptyvarerr}[3]{
	% Lanza un mensaje de error si una variable no ha sido definida
	% 	#1	Función del error
	%	#2	Variable
	%	#3	Mensaje
	\ifx\hfuzz#2\hfuzz
		\throwerror{#1}{#3}
	\fi
}
\newcommand{\quotes}[1]{
	% Insertar cita
	% 	#1	Texto
	''#1''
}
\newcommand{\quotesit}[1]{
	% Insertar cita en itálico
	% 	#1	Texto
	\textit{\quotes{#1}}
}
\newcommand{\newemptypage}{
	% Crea una página vacía
	\newpage\null\thispagestyle{empty}\newpage
	\addtocounter{page}{-1}
}
\newcommand{\setcaptionmargincm}[1]{
	% Cambiar el margen
	% 	#1	Margen en centímetros
	\captionsetup{margin=#1cm}
}
\newcommand{\setpagemargincm}[4]{
	% Cambia márgenes de las páginas [cm]
	% 	#1	Margen izquierdo
	%	#2	Margen superior
	%	#3	Margen derecho
	%	#4	Margen inferior
	\newgeometry{left=#1cm, top=#2cm, right=#3cm, bottom=#4cm}
}
\newcommand{\newp}{
	% Inserta nueva línea	
	\hbadness=10000 \vspace{\defaultnewlinesize} \par
}
\newcommand{\newpar}[1]{
	% Insertar párrafo
	% 	#1	Párrafo
	\hbadness=10000 #1 \newp
}
\newcommand{\newparnl}[1]{
	% Insertar párrafo sin nueva línea al final
	% 	#1	Párrafo
	#1 \par
}
\newcommand{\lpow}[2]{
	% Insertar sub-índice, a_b
	% 	#1	Elemento inferior (a)
	%	#2	Elemento superior (b)
	{#1}_{#2}
}
\newcommand{\pow}[2]{
	% Insertar elevado, a^b
	% 	#1	Elemento inferior (a)
	%	#2	Elemento superior (b)
	{#1}^{#2}
}
\newcommand{\fracpartial}[2]{
	% Fracción de derivadas parciales af/ax
	% 	#1	Función a derivar (f)
	%	#2	Variable a derivar (x)
	\frac{\partial #1}{\partial #2}
}
\newcommand{\fracdpartial}[2]{
	% Fracción de derivadas parciales dobles a^2f/ax^2
	% 	#1	Función a derivar (f)
	%	#2	Variable a derivar (x)
	\frac{{\partial}^{2} #1}{\partial {#2}^{2}}
}
\newcommand{\fracnpartial}[3]{
	% Fracción de derivadas parciales en n, a^nf/ax^n
	% 	#1	Función a derivar (f)
	%	#2	Variable a derivar (x)
	%	#3	Orden (n)
	\frac{{\partial}^{#3} #1}{\partial {#2}^{#3}}
}
\newcommand{\fracderivat}[2]{
	% Fracción de derivadas df/dx
	% 	#1	Función a derivar (f)
	%	#2	Variable a derivar (x)
	\frac{\text{d} #1}{\text{d} #2}
}
\newcommand{\fracdderivat}[2]{
	% Fracción de derivadas dobles d^2/dx^2
	% 	#1	Función a derivar (f)
	%	#2	Variable a derivar (x)
	\frac{{\text{d}}^{2} #1}{\text{d} {#2}^{2}}
}
\newcommand{\fracnderivat}[3]{
	% Fracción de derivadas en n d^nf/dx^n
	% 	#1	Función a derivar (f)
	%	#2	Variable a derivar (x)
	%	#3	Orden de la derivada (n)	
	\frac{{\text{d}}^{#3} #1}{\text{d} {#2}^{#3}}
}
\newcommand{\topequal}[2]{
	% Llave superior de equivalencia
	% 	#1	Elemento a igualar
	%	#2	Igualdad
	\overbrace{#1}^{\mathclap{#2}}
}
\newcommand{\underequal}[2]{
	% Llave inferior de equivalencia
	% 	#1	Elemento a igualar
	%	#2	Igualdad
	\underbrace{#1}_{\mathclap{#2}}
}
\newcommand{\topsequal}[2]{
	% Rectángulo superior de equivalencia
	% 	#1	Elemento a igualar
	%	#2	Igualdad
	\overbracket{#1}^{\mathclap{#2}}
}
\newcommand{\undersequal}[2]{
	% Rectángulo inferior de equivalencia
	% 	#1	Elemento a igualar
	%	#2	Igualdad
	\underbracket{#1}_{\mathclap{#2}}
}
\newcommand{\resizeitem}[2]{
	% Crea un resizebox de tamaño textwidth
	% 	#1	Tamaño del nuevo objeto (En textwidth)
	%	#2	Objeto a redimensionar
	\emptyvarerr{\resizeitem}{#1}{Tamano del nuevo objeto no definido}
	\emptyvarerr{\resizeitem}{#2}{Objeto a redimensionar no definido}
	\resizebox{#1\textwidth}{!}{#2}
}
\newcommand{\newtitleanum}[1]{
	% Insertar un título sin número
	% 	#1	Título
	\emptyvarerr{\newtitleanum}{#1}{Titulo no definido}
	\addcontentsline{toc}{section}{#1}
	\section*{#1}
	\ifthenelse{\equal{\showheadertitle}{true}}{
		\fancyhead[L]{\nouppercase{#1}}}{}
	\stepcounter{section}
}
\newcommand{\newtitleanumheadless}[1]{
	% Insertar un título sin número sin alterar el header
	% 	#1	Título
	\emptyvarerr{\newtitleanumheadless}{#1}{Titulo no definido}
	\addcontentsline{toc}{section}{#1}
	\section*{#1}
	\stepcounter{section}
}
\newcommand{\newsubtitleanum}[1]{
	% Insertar un subtítulo sin número
	% 	#1	Subtítulo
	\emptyvarerr{\newsubtitleanum}{#1}{Subtitulo no definido}
	\addcontentsline{toc}{subsection}{#1}
	\subsection*{#1}
	\stepcounter{subsection}
}
\newcommand{\newsubsubtitleanum}[1]{
	% Insertar un sub-subtítulo sin número
	% 	#1	Sub-subtítulo
	\emptyvarerr{\newsubsubtitleanum}{#1}{Sub-subtitulo no definido}
	\addcontentsline{toc}{subsubsection}{#1}
	\subsubsection*{#1}
	\stepcounter{subsubsection}
}
\newcommand{\newtitleanumnoi}[1]{
	% Insertar un título sin número sin indexar
	% 	#1	Título
	\emptyvarerr{\newtitleanumnoi}{#1}{Titulo no definido}
	\section*{#1}
	\ifthenelse{\equal{\showheadertitle}{true}}{
		\fancyhead[L]{\nouppercase{#1}}}{}
	\stepcounter{section}
}
\newcommand{\newtitleanumnoiheadless}[1]{
	% Insertar un título sin número sin indexar sin cambiar el header
	% 	#1	Título
	\emptyvarerr{\newtitleanumnoiheadless}{#1}{Titulo no definido}
	\section*{#1}
	\ifthenelse{\equal{\showheadertitle}{true}}{
		\fancyhead[L]{\nouppercase{#1}}}{}
	\stepcounter{section}
}
\newcommand{\newsubtitleanumnoi}[1]{
	% Insertar un subtítulo sin número sin indexar
	% 	#1	Subtítulo
	\emptyvarerr{\newsubtitleanumnoi}{#1}{Subtitulo no definido}
	\subsection*{#1}
	\stepcounter{subsection}
}
\newcommand{\newsubsubtitleanumnoi}[1]{
	% Insertar un sub-subtítulo sin número sin indexar
	% 	#1	Sub-subtítulo
	\emptyvarerr{\newsubsubtitleanumnoi}{#1}{Sub-subtitulo no definido}
	\addcontentsline{toc}{subsubsection}{#1}
	\subsubsection*{#1}
	\stepcounter{subsubsection}
}
\newcommand{\insertequation}[2][]{
	% Insertar una ecuación
	% 	#1	Label (opcional)
	%	#2	Ecuación
	\emptyvarerr{\insertequation}{#2}{Ecuacion no definida}
	\vspace{-0.1cm}
	\begin{equation}
		\text{#1} #2
	\end{equation}
	\vspace{-0.23cm}
	\par
}
\newcommand{\insertequationcaptioned}[3][]{
	% Insertar una ecuación con leyenda
	% 	#1	Label (opcional)
	%	#2	Ecuación
	%	#3	Caption
	\emptyvarerr{\insertequationcaptioned}{#2}{Ecuacion no definida}
	\ifx\hfuzz#3\hfuzz
		\insertequation[#1]{#2}
	\else
		\vspace{0cm}
		\begin{equation}
		\text{#1} #2
		\end{equation}
		\begin{center}
			\vspace{-0.15cm}
			\textit{#3} \par
			\vspace{0.05cm}
		\end{center}	
	\fi
}
\newcommand{\insertequationgathered}[2][]{
	% Insertar una ecuación con el ambiente gather
	% 	#1	Label (opcional)
	%	#2	Ecuación
	\emptyvarerr{\insertequationgathered}{#2}{Ecuacion no definida}
	\vspace{-0.4cm}
	\begin{gather}
		\text{#1} #2
	\end{gather}
	\par
	\vspace{-0.10cm}
}
\newcommand{\insertequationgatheredcaptioned}[3][]{
	% Insertar una ecuación (gather) con leyenda
	% 	#1	Label (opcional)
	%	#2	Ecuación
	%	#3	Caption
	\emptyvarerr{\insertequationgatheredcaptioned}{#2}{Ecuacion no definida}
	\ifx\hfuzz#3\hfuzz
		\insertequationgathered[#1]{#2}
	\else
		\vspace{0cm}
		\begin{gather}
			\text{#1} #2
		\end{gather}
		\begin{center}
			\vspace{-0.15cm}
			\textit{#3} \par
		\end{center}
	\fi
}
\newcommand{\insertequationalign}[2][]{
	% Insertar una ecuación con el ambiente align
	% 	#1	Label (opcional)
	%	#2	Ecuación
	\emptyvarerr{\insertequationalign}{#2}{Ecuacion no definida}
	\vspace{-0.4cm}
	\begin{align}
		\text{#1} #2
	\end{align}
	\par
	\vspace{-0.10cm}
}
\newcommand{\insertequationaligncaptioned}[3][]{
	% Insertar una ecuación (align) con leyenda
	% 	#1	Label (opcional)
	%	#2	Ecuación
	%	#3	Caption
	\emptyvarerr{\insertequationaligncaptioned}{#2}{Ecuacion no definida}
	\ifx\hfuzz#3\hfuzz
		\insertequationalign[#1]{#2}
	\else
		\vspace{0cm}
		\begin{align}
			\text{#1} #2
		\end{align}
		\begin{center}
			\vspace{-0.15cm}
			\textit{#3} \par
		\end{center}
	\fi
}
\newcommand{\insertimage}[4][]{
	% Insertar una imagen
	% 	#1	Label (opcional)
	%	#2	Dirección de la imagen
	%	#3	Parámetros de la imagen
	%	#4	Caption de la imagen
	\emptyvarerr{\insertimage}{#2}{Direccion de la imagen no definida}
	\emptyvarerr{\insertimage}{#3}{Parametros de la imagen no definidos}
	\vspace{\defaultmargintopimages}
	\begin{figure}[H]
		\centering
		\includegraphics[#3]{\defaultimagefolder#2}
		\ifx\hfuzz#4\hfuzz
			\vspace{\defaultcaptionlessmargin}
		\else
			\caption{#4 #1}
		\fi
	\end{figure}
	\vspace{\defaultmarginbottomimages}
}
\newcommand{\insertimageboxed}[4][]{
	% Insertar una imagen con recuadro
	% 	#1	Label (opcional)
	%	#2	Dirección de la imagen
	%	#3	Parámetros de la imagen
	%	#4	Caption de la imagen
	\emptyvarerr{\insertimageboxed}{#2}{Direccion de la imagen no definida}
	\emptyvarerr{\insertimageboxed}{#3}{Parametros de la imagen no definidos}
	\vspace{\defaultmargintopimages}
	\begin{figure}[H]
		\centering
		\fbox{\includegraphics[#3]{\defaultimagefolder#2}}
		\ifx\hfuzz#4\hfuzz
			\vspace{\defaultcaptionlessmargin}
		\else
			\caption{#4 #1}
		\fi
	\end{figure}
	\vspace{\defaultmarginbottomimages}
}
\newcommand{\insertimagefixed}[5][]{
	% Insertar una imagen de ancho fijo a la página
	% 	#1	Label (opcional)
	%	#2	Dirección de la imagen
	%	#3	Parámetros de la imagen
	%	#4	Tamaño de la imagen en textwidth
	%	#5	Caption de la imagen
	\emptyvarerr{\insertimagefixed}{#2}{Direccion de la imagen no definida}
	\emptyvarerr{\insertimagefixed}{#3}{Parametros de la imagen no definidos}
	\emptyvarerr{\insertimagefixed}{#4}{Tamano de la imagen (textwidth) no definida}
	\vspace{\defaultmargintopimages}
	\begin{figure}[H]
		\centering
		\resizebox{#3\textwidth}{!}{
			\includegraphics[#4]{\defaultimagefolder#2}
		}
		\ifx\hfuzz#5\hfuzz
			\vspace{\defaultcaptionlessmargin}
		\else
			\caption{#5 #1}
		\fi
	\end{figure}
	\vspace{\defaultmarginbottomimages}
}
\newcommand{\insertimageboxedfixed}[5][]{
	% Insertar una imagen recuadrada de ancho fijo
	% 	#1	Label (opcional)
	%	#2	Dirección de la imagen
	%	#3	Parámetros de la imagen
	%	#4	Tamaño de la imagen en textwidth
	%	#5	Caption de la imagen
	\emptyvarerr{\insertimageboxedfixed}{#2}{Direccion de la imagen no definida}
	\emptyvarerr{\insertimageboxedfixed}{#3}{Parametros de la imagen no definidos}
	\emptyvarerr{\insertimageboxedfixed}{#4}{Tamano de la imagen no definida}
	\vspace{\defaultmargintopimages}
	\begin{figure}[H]
		\centering
		\resizebox{#3\textwidth}{!}{
			\fbox{\includegraphics[#4]{\defaultimagefolder#2}}
		}
		\ifx\hfuzz#5\hfuzz
			\vspace{\defaultcaptionlessmargin}
		\else
			\caption{#5 #1}
		\fi
	\end{figure}
	\vspace{\defaultmarginbottomimages}
}
\newcommand{\insertdoubleimage}[8][]{
	% Insertar una imagen doble
	% 	#1	Label (opcional)
	%	#2	Dirección de la imagen 1
	%	#3	Parámetros de la imagen 1
	%	#4	Caption de la imagen 1
	%	#5	Dirección de la imagen 2
	%	#6	Parámetros de la imagen 2
	%	#7	Caption de la imagen 2
	%	#8	Caption de la imagen doble
	\emptyvarerr{\insertdoubleimage}{#2}{Direccion de la imagen 1 no definida}
	\emptyvarerr{\insertdoubleimage}{#3}{Parametros de la imagen 1 no definidos}
	\emptyvarerr{\insertdoubleimage}{#5}{Direccion de la imagen 2 no definida}
	\emptyvarerr{\insertdoubleimage}{#6}{Parametros de la imagen 2 no definidos}
	\vspace{\defaultmargintopimages}
	\captionsetup{margin=0.45cm}
	\begin{figure}[H] \centering
		\subfloat[#4]{
			\includegraphics[#3]{\defaultimagefolder#2}}
		\hspace{0.2cm}
		\subfloat[#7]{
			\includegraphics[#6]{\defaultimagefolder#5}}
		\setcaptionmargincm{\defaultcaptionmargin}
		\ifx\hfuzz#8\hfuzz
			\vspace{\defaultcaptionlessmargin}
		\else
			\caption{#8 #1}
		\fi
	\end{figure}
	\setcaptionmargincm{\defaultcaptionmargin}
	\vspace{\defaultmarginbottomimages}
}
\newcommand{\insertdoubleeqimage}[7][]{
	% Insertar una imagen doble, igual propiedades
	% 	#1	Label (opcional)
	%	#2	Dirección de la imagen 1
	%	#3	Caption de la imagen 1
	%	#4	Dirección de la imagen 2
	%	#5	Caption de la imagen 2
	%	#6	Propiedades de las imágenes
	%	#7 	Caption de la imagen doble
	\insertdoubleimage[#1]{#2}{#6}{#3}{#4}{#6}{#5}{#7}
}
\newcommand{\inserttripleimage}[8][]{
	% Insertar una imagen triple
	% 	#1	Label (opcional)
	%	#2	Dirección de la imagen 1
	%	#3	Parámetros de la imagen 1
	%	#4	Dirección de la imagen 2
	%	#5	Parámetros de la imagen 2
	%	#6	Dirección de la imagen 3
	%	#7	Parámetros de la imagen 3
	%	#8	Caption de la imagen triple
	\emptyvarerr{\inserttripleimage}{#2}{Direccion de la imagen 1 no definida}
	\emptyvarerr{\inserttripleimage}{#3}{Parametros de la imagen 1 no definidos}
	\emptyvarerr{\inserttripleimage}{#4}{Direccion de la imagen 2 no definida}
	\emptyvarerr{\inserttripleimage}{#5}{Parametros de la imagen 2 no definidos}
	\emptyvarerr{\inserttripleimage}{#6}{Direccion de la imagen 3 no definida}
	\emptyvarerr{\inserttripleimage}{#7}{Parametros de la imagen 3 no definidos}
	\vspace{\defaultmargintopimages}
	\captionsetup{margin=0.45cm}
	\begin{figure}[H] \centering
		\subfloat[]{
			\includegraphics[#3]{\defaultimagefolder#2}}
		\hspace{0.1cm}
		\subfloat[]{
			\includegraphics[#5]{\defaultimagefolder#4}}
		\hspace{0.1cm}
		\subfloat[]{
			\includegraphics[#7]{\defaultimagefolder#6}}
		\setcaptionmargincm{\defaultcaptionmargin}
		\ifx\hfuzz#8\hfuzz
			\vspace{\defaultcaptionlessmargin}
		\else
			\caption{#8 #1}
		\fi
	\end{figure}
	\setcaptionmargincm{\defaultcaptionmargin}
	\vspace{\defaultmarginbottomimages}
}
\newcommand{\inserttripleeqimage}[6][]{
	% Insertar una imagen triple, igual propiedades
	% 	#1	Label (opcional)
	%	#2	Dirección de la imagen 1
	%	#3	Dirección de la imagen 2
	%	#4	Dirección de la imagen 3
	%	#5	Propiedades de las imágenes
	%	#6	Caption de la imagen triple
	\inserttripleimage[#1]{#2}{#5}{#3}{#5}{#4}{#5}{#6}
}
\newcommand{\insertquadimage}[7][]{
	% Insertar una imagen cuádruple, igual propiedades
	% 	#1	Label (opcional)
	%	#2	Dirección de la imagen 1
	%	#3	Dirección de la imagen 2
	%	#4	Dirección de la imagen 3
	%	#5	Dirección de la imagen 4
	%	#6	Propiedades de las imágenes
	%	#7	Caption de la imagen cuádruple
	\emptyvarerr{\insertquadimage}{#2}{Direccion de la imagen 1 no definida}
	\emptyvarerr{\insertquadimage}{#3}{Direccion de la imagen 2 no definida}
	\emptyvarerr{\insertquadimage}{#4}{Direccion de la imagen 3 no definida}
	\emptyvarerr{\insertquadimage}{#5}{Direccion de la imagen 4 no definida}
	\emptyvarerr{\insertquadimage}{#6}{Propiedades de las imagenes no definidos}
	\vspace{\defaultmargintopimages}
	\captionsetup{margin=0.45cm}
	\begin{figure}[H] \centering
		\subfloat[]{
			\includegraphics[#6]{\defaultimagefolder#2}}
		\hspace{0.1cm}
		\subfloat[]{
			\includegraphics[#6]{\defaultimagefolder#3}}
		\hspace{0.1cm}
		\subfloat[]{
			\includegraphics[#6]{\defaultimagefolder#4}}
		\hspace{0.1cm}
		\subfloat[]{
			\includegraphics[#6]{\defaultimagefolder#5}}
		\setcaptionmargincm{\defaultcaptionmargin}
		\ifx\hfuzz#7\hfuzz
			\vspace{\defaultcaptionlessmargin}
		\else
			\caption{#7 #1}
		\fi
	\end{figure}
	\setcaptionmargincm{\defaultcaptionmargin}
	\vspace{\defaultmarginbottomimages}
}
\newcommand{\insertpentaimage}[8][]{
	% Insertar una imagen quíntuple, igual propiedades
	% 	#1	Label (opcional)
	%	#2	Dirección de la imagen 1
	%	#3	Dirección de la imagen 2
	%	#4	Dirección de la imagen 3
	%	#5	Dirección de la imagen 4
	%	#6	Dirección de la imagen 5
	%	#7	Propiedades de las imágenes
	%	#8	Caption de la imagen quíntuple
	\emptyvarerr{\insertpentaimage}{#2}{Direccion de la imagen 1 no definida}
	\emptyvarerr{\insertpentaimage}{#3}{Direccion de la imagen 2 no definida}
	\emptyvarerr{\insertpentaimage}{#4}{Direccion de la imagen 3 no definida}
	\emptyvarerr{\insertpentaimage}{#5}{Direccion de la imagen 4 no definida}
	\emptyvarerr{\insertpentaimage}{#6}{Direccion de la imagen 5 no definida}
	\emptyvarerr{\insertpentaimage}{#7}{Propiedades de las imagenes no definidas}
	\vspace{\defaultmargintopimages}
	\captionsetup{margin=0.45cm}
	\begin{figure}[H] \centering
		\subfloat[]{
			\includegraphics[#7]{\defaultimagefolder#2}}
		\hspace{0.1cm}
		\subfloat[]{
			\includegraphics[#7]{\defaultimagefolder#3}}
		\hspace{0.1cm}
		\subfloat[]{
			\includegraphics[#7]{\defaultimagefolder#4}}
		\hspace{0.1cm}
		\subfloat[]{
			\includegraphics[#7]{\defaultimagefolder#5}}
		\hspace{0.1cm}
		\subfloat[]{
			\includegraphics[#7]{\defaultimagefolder#6}}
		\setcaptionmargincm{\defaultcaptionmargin}
		\ifx\hfuzz#8\hfuzz
			\vspace{\defaultcaptionlessmargin}
		\else
			\caption{#8 #1}
		\fi
	\end{figure}
	\setcaptionmargincm{\defaultcaptionmargin}
	\vspace{\defaultmarginbottomimages}
}
\newcommand{\inserthexaimage}[9][]{
	% Insertar una imagen con 6 imágenes, igual propiedades
	% 	#1	Label (opcional)
	%	#2	Dirección de la imagen 1
	%	#3	Dirección de la imagen 2
	%	#4	Dirección de la imagen 3
	%	#5	Dirección de la imagen 4
	%	#6	Dirección de la imagen 5
	%	#7	Dirección de la imagen 6
	%	#8	Propiedades de las imágenes
	%	#9	Caption de la imagen global
	\emptyvarerr{\inserthexaimage}{#2}{Direccion de la imagen 1 no definida}
	\emptyvarerr{\inserthexaimage}{#3}{Direccion de la imagen 2 no definida}
	\emptyvarerr{\inserthexaimage}{#4}{Direccion de la imagen 3 no definida}
	\emptyvarerr{\inserthexaimage}{#5}{Direccion de la imagen 4 no definida}
	\emptyvarerr{\inserthexaimage}{#6}{Direccion de la imagen 5 no definida}
	\emptyvarerr{\inserthexaimage}{#7}{Direccion de la imagen 6 no definida}
	\emptyvarerr{\inserthexaimage}{#8}{Propiedades de las imagenes no definidas}
	\vspace{\defaultmargintopimages}
	\captionsetup{margin=0.45cm}
	\begin{figure}[H] \centering
		\subfloat[]{
			\includegraphics[#8]{\defaultimagefolder#2}}
		\hspace{0.1cm}
		\subfloat[]{
			\includegraphics[#8]{\defaultimagefolder#3}}
		\hspace{0.1cm}
		\subfloat[]{
			\includegraphics[#8]{\defaultimagefolder#4}}
		\hspace{0.1cm}
		\subfloat[]{
			\includegraphics[#8]{\defaultimagefolder#5}}
		\hspace{0.1cm}
		\subfloat[]{
			\includegraphics[#8]{\defaultimagefolder#6}}
		\hspace{0.1cm}
		\subfloat[]{
			\includegraphics[#8]{\defaultimagefolder#7}}
		\setcaptionmargincm{\defaultcaptionmargin}
		\ifx\hfuzz#9\hfuzz
			\vspace{\defaultcaptionlessmargin}
		\else
			\caption{#9 #1}
		\fi
	\end{figure}
	\setcaptionmargincm{\defaultcaptionmargin}
	\vspace{\defaultmarginbottomimages}
}
\newcommand{\insertimageleft}[5][]{
	% Insertar una imagen a la izquierda
	% 	#1	Label (opcional)
	%	#2	Dirección de la imagen
	%	#3	Ancho de la imagen (en textwidth)
	%	#4	Altura en líneas de la imagen
	%	#5	Caption de la imagen
	\emptyvarerr{\insertimageleft}{#2}{Direccion de la imagen no definida}
	\emptyvarerr{\insertimageleft}{#3}{Ancho de la imagen no defindo}
	\emptyvarerr{\insertimageleft}{#4}{Altura en lineas de la imagen no definida}
	\begin{wrapfigure}[#4]{l}{#3\textwidth}
		\setcaptionmargincm{0}
		\vspace{\defaultmarginfloatimages}
		\centering
		\includegraphics[width=\linewidth]{\defaultimagefolder#2}
		\ifx\hfuzz#5\hfuzz
			\vspace{\defaultcaptionlessmargin}
		\else
			\caption{#5 #1}
		\fi
		\setcaptionmargincm{\defaultcaptionmargin}
	\end{wrapfigure}
}
\newcommand{\insertimageright}[5][]{
	% Insertar una imagen a la derecha
	% 	#1	Label (opcional)
	%	#2	Dirección de la imagen
	%	#3	Ancho de la imagen (en textwidth)
	%	#4	Altura en líneas de la imagen
	%	#5	Caption de la imagen
	\emptyvarerr{\insertimageright}{#2}{Direccion de la imagen no definida}
	\emptyvarerr{\insertimageright}{#3}{Ancho de la imagen no defindo}
	\emptyvarerr{\insertimageright}{#4}{Altura en lineas de la imagen no definida}
	\begin{wrapfigure}[#4]{r}{#3\textwidth}
		\setcaptionmargincm{0}
		\vspace{\defaultmarginfloatimages}
		\centering
		\includegraphics[width=\linewidth]{\defaultimagefolder#2}
		\ifx\hfuzz#5\hfuzz
			\vspace{\defaultcaptionlessmargin}
		\else
			\caption{#5 #1}
		\fi
		\setcaptionmargincm{\defaultcaptionmargin}
	\end{wrapfigure}
}

% IMPORTACIÓN DE ENTORNOS
% Template:     Informe/Reporte LaTeX
% Documento:    Definición de entornos
% Versión:      4.2.4 (09/07/2017)
% Codificación: UTF-8
%
% Autor: Pablo Pizarro R.
%        Facultad de Ciencias Físicas y Matemáticas
%        Universidad de Chile
%        pablo.pizarro@ing.uchile.cl, ppizarror.com
%
% Manual template: [http://ppizarror.com/Template-Informe/]
% Licencia MIT:    [https://opensource.org/licenses/MIT/]

% Crea una sección de referencias solo para bibtex
\newenvironment{references}{
	\ifthenelse{\equal{\stylecitereferences}{bibtex}}{
	}{
		\throwerror{}{Solo se puede usar entorno references con estilo citas \noexpand\stylecitereferences=bibtex}
	}
	\begingroup
	% Se configura las referencias como una sección
	\ifthenelse{\equal{\referencenumsection}{true}}{
		\section{\namereferences}
	}{
		\sectionanum{\namereferences}
	}
	\renewcommand{\section}[2]{}
	\begin{thebibliography}{99}
	}
	{
	\end{thebibliography}
	\endgroup
}

% Crea una sección de anexos
\newenvironment{anexo}{
	\begingroup
	\clearpage
	\phantomsection
	\ifthenelse{\equal{\showappendixsectitle}{true}}{
		\appendixpage}{
	}
	\appendixtitleon
	\appendicestocpagenum
	\appendixtitletocon
	\bookmarksetupnext{level=part}
	\bookmarksetup{
		open,
		numbered,
		openlevel=0
	}
	\begin{appendices}
		\ifthenelse{\equal{\showappendixsecindex}{true}}{}{
			\belowpdfbookmark{\nameappendixsection}{contents}
		}
		\setcounter{secnumdepth}{3}
		\setcounter{tocdepth}{3}
		\counterwithin{equation}{section}
		\counterwithin{figure}{section}
		\counterwithin{lstlisting}{section}
		\counterwithin{table}{section}
		}
		{
	\end{appendices}
	\endgroup
}

% Crea una sección de resumen
\newenvironment{resumen}{
	% Tipo de título para abstract
	\sectionfont{\color{\titlecolor} \fontsizetitle \styletitle \selectfont}
	
	% Inserta un título sin número, sin cabecera y sin aparecer en el índice, para que aparezca en el índice utilizar la función \sectionanumheadless
	\sectionanumnoiheadless{\nameabstract}}{
	
	% Salta de página si está imprimiendo por ambas caras
	\ifthenelse{\equal{\addemptypagetwosides}{true}}{
		\checkoddpage
		\ifoddpage
			\newpage
			\null
			\thispagestyle{empty}
			\newpage
			\addtocounter{page}{-1}
		\else
		\fi
	}{}
}

% Columna centrada en tablas
\newcolumntype{P}[1]{
	>{\centering\arraybackslash}p{#1}
}


% IMPORTACIÓN DE AMBIENTES Y ESTILOS
% Template:     Informe/Reporte LaTeX
% Documento:    Definición de estilos
% Versión:      4.4.4 (26/08/2017)
% Codificación: UTF-8
%
% Autor: Pablo Pizarro R.
%        Facultad de Ciencias Físicas y Matemáticas
%        Universidad de Chile
%        pablo.pizarro@ing.uchile.cl, ppizarror.com
%
% Manual template: [http://latex.ppizarror.com/Template-Informe/]
% Licencia MIT:    [https://opensource.org/licenses/MIT/]

% Definición de colores
\definecolor{backcolour}{rgb}{0.95, 0.95, 0.92}
\definecolor{codegray}{rgb}{0.5, 0.5, 0.5}
\definecolor{codegreen}{rgb}{0, 0.6, 0}
\definecolor{codepurple}{rgb}{0.58, 0, 0.82}
\definecolor{dkgreen}{rgb}{0, 0.6, 0}
\definecolor{dgray}{RGB}{104, 108, 113}
\definecolor{gray}{rgb}{0.5, 0.5, 0.5}
\definecolor{lightyellow}{rgb}{1.0, 1.0, 0.88}
\definecolor{mauve}{rgb}{0.58, 0, 0.82}
\definecolor{mygray}{rgb}{0.5, 0.5, 0.5}
\definecolor{mygreen}{rgb}{0, 0.6, 0}
\definecolor{mylilas}{RGB}{170, 55, 241}

% Estilo de códigos
\ifthenelse{\equal{\codecaptiontop}{true}}{ % Leyenda arriba

	% Estilo de lenguaje C
	\lstdefinestyle{C}{
		language=C,
		aboveskip=3mm,
		backgroundcolor=\color{backcolour},
		basicstyle={\small\ttfamily},
		belowskip=3mm,
		breakatwhitespace=false,
		breaklines=true,
		captionpos=t,
		columns=flexible,
		commentstyle=\color{mygreen},
		keepspaces=true,
		keywordstyle=\color{magenta},
		numbers=left,
		numbersep=5pt,
		numberstyle=\tiny\color{mygray},
		showspaces=false,
		showstringspaces=false,
		showtabs=false,
		stepnumber=1,
		stringstyle=\color{mauve},
		tabsize=3
	}
	
	% Estilo de lenguaje Java
	\lstdefinestyle{Java}{
		language=Java,
		aboveskip=3mm,
		backgroundcolor=\color{backcolour},
		basicstyle={\small\ttfamily},
		belowskip=3mm,
		breakatwhitespace=true,
		breaklines=true,
		captionpos=t,
		columns=flexible,
		commentstyle=\color{dkgreen},
		keepspaces=true,
		keywordstyle=\color{blue},
		numbers=left,
		numbersep=5pt,
		numberstyle=\tiny\color{gray},
		showstringspaces=false,
		stepnumber=1,
		stringstyle=\color{mauve},
		tabsize=3
	}
	
	% Estilo de lenguaje Matlab
	\lstdefinestyle{Matlab}{
		language=Matlab,
		aboveskip=3mm,
		backgroundcolor=\color{backcolour},
		basicstyle={\small\ttfamily},
		belowskip=3mm,
		breaklines=true,
		breaklines=true,
		captionpos=t,
		columns=flexible,
		commentstyle=\color{mygreen},
		emph=[1]{for,end,break},emphstyle=[1]\color{red},
		identifierstyle=\color{black},
		keepspaces=true,
		keywordstyle=\color{blue},
		morekeywords={matlab2tikz},
		numbers=left,
		numbersep=5pt,
		numberstyle=\tiny\color{gray},
		showstringspaces=false,
		stepnumber=1,
		stringstyle=\color{mylilas},
		tabsize=3
	}
	
	% Estilo de lenguaje Python
	\lstdefinestyle{Python}{
		language=Python,
		aboveskip=3mm,
		backgroundcolor=\color{backcolour},
		basicstyle={\small\ttfamily},
		belowskip=3mm,
		breakatwhitespace=false,
		breaklines=true,
		captionpos=t,
		columns=flexible,
		commentstyle=\color{codegreen},
		keepspaces=true,
		keywordstyle=\color{magenta},
		numbers=left,
		numbersep=5pt,
		numberstyle=\tiny\color{codegray},
		showspaces=false,
		showstringspaces=false,
		showtabs=false,
		stepnumber=1,
		stringstyle=\color{codepurple},
		tabsize=3
	}
}{ % Leyenda abajo

	% Estilo de lenguaje C
	\lstdefinestyle{C}{
		language=C,
		aboveskip=3mm,
		backgroundcolor=\color{backcolour},
		basicstyle={\small\ttfamily},
		belowskip=3mm,
		breakatwhitespace=false,
		breaklines=true,
		captionpos=b,
		columns=flexible,
		commentstyle=\color{mygreen},
		keepspaces=true,
		keywordstyle=\color{magenta},
		numbers=left,
		numbersep=5pt,
		numberstyle=\tiny\color{mygray},
		showspaces=false,
		showstringspaces=false,
		showtabs=false,
		stepnumber=1,
		stringstyle=\color{mauve},
		tabsize=3
	}
	
	% Estilo de lenguaje Java
	\lstdefinestyle{Java}{
		language=Java,
		aboveskip=3mm,
		backgroundcolor=\color{backcolour},
		basicstyle={\small\ttfamily},
		belowskip=3mm,
		breakatwhitespace=true,
		breaklines=true,
		captionpos=b,
		columns=flexible,
		commentstyle=\color{dkgreen},
		keepspaces=true,
		keywordstyle=\color{blue},
		numbers=left,
		numbersep=5pt,
		numberstyle=\tiny\color{gray},
		showstringspaces=false,
		stepnumber=1,
		stringstyle=\color{mauve},
		tabsize=3
	}
	
	% Estilo de lenguaje Matlab
	\lstdefinestyle{Matlab}{
		language=Matlab,
		aboveskip=3mm,
		backgroundcolor=\color{backcolour},
		basicstyle={\small\ttfamily},
		belowskip=3mm,
		breaklines=true,
		breaklines=true,
		captionpos=b,
		columns=flexible,
		commentstyle=\color{mygreen},
		emph=[1]{for,end,break},emphstyle=[1]\color{red},
		identifierstyle=\color{black},
		keepspaces=true,
		keywordstyle=\color{blue},
		morekeywords={matlab2tikz},
		numbers=left,
		numbersep=5pt,
		numberstyle=\tiny\color{gray},
		showstringspaces=false,
		stepnumber=1,
		stringstyle=\color{mylilas},
		tabsize=3
	}
	
	% Estilo de lenguaje Python
	\lstdefinestyle{Python}{
		language=Python,
		aboveskip=3mm,
		backgroundcolor=\color{backcolour},
		basicstyle={\small\ttfamily},
		belowskip=3mm,
		breakatwhitespace=false,
		breaklines=true,
		captionpos=b,
		columns=flexible,
		commentstyle=\color{codegreen},
		keepspaces=true,
		keywordstyle=\color{magenta},
		numbers=left,
		numbersep=5pt,
		numberstyle=\tiny\color{codegray},
		showspaces=false,
		showstringspaces=false,
		showtabs=false,
		stepnumber=1,
		stringstyle=\color{codepurple},
		tabsize=3
	}	
}

% Estilo de enumeración en griego
\RequirePackage{enumitem}
\makeatletter
\def\greek#1{\expandafter\@greek\csname c@#1\endcsname}
\def\Greek#1{\expandafter\@Greek\csname c@#1\endcsname}
\def\@greek#1{\ifcase#1
	\or $\alpha$%
	\or $\beta$%
	\or $\gamma$%
	\or $\delta$%
	\or $\epsilon$%
	\or $\zeta$%
	\or $\eta$%
	\or $\theta$%
	\or $\iota$%
	\or $\kappa$%
	\or $\lambda$%
	\or $\mu$%
	\or $\nu$%
	\or $\xi$%
	\or $o$%
	\or $\pi$%
	\or $\rho$%
	\or $\sigma$%
	\or $\tau$%
	\or $\upsilon$%
	\or $\phi$%
	\or $\chi$%
	\or $\psi$%
	\or $\omega$%
	\fi}
\def\@Greek#1{\ifcase#1
	\or $\mathrm{A}$%
	\or $\mathrm{B}$%
	\or $\Gamma$%
	\or $\Delta$%
	\or $\mathrm{E}$%
	\or $\mathrm{Z}$%
	\or $\mathrm{H}$%
	\or $\Theta$%
	\or $\mathrm{I}$%
	\or $\mathrm{K}$%
	\or $\Lambda$%
	\or $\mathrm{M}$%
	\or $\mathrm{N}$%
	\or $\Xi$%
	\or $\mathrm{O}$%
	\or $\Pi$%
	\or $\mathrm{P}$%
	\or $\Sigma$%
	\or $\mathrm{T}$%
	\or $\mathrm{Y}$%
	\or $\Phi$%
	\or $\mathrm{X}$%
	\or $\Psi$%
	\or $\Omega$%
	\fi}
\makeatother
\AddEnumerateCounter{\greek}{\@greek}{24}
\AddEnumerateCounter{\Greek}{\@Greek}{12}

% Columna centrada en tablas
\newcolumntype{P}[1]{
	>{\centering\arraybackslash}p{#1}
}


% CONFIGURACIÓN INICIAL DEL DOCUMENTO
% Template:     Informe/Reporte LaTeX
% Documento:    Configuración inicial del template
% Versión:      3.5.8 (13/05/2017)
% Codificación: UTF-8
%
% Autor: Pablo Pizarro R.
%        Facultad de Ciencias Físicas y Matemáticas
%        Universidad de Chile
%        pablo.pizarro@ing.uchile.cl, ppizarror.com
%
% Sitio web del proyecto: [http://ppizarror.com/Template-Informe/]
% Licencia: MIT           [https://opensource.org/licenses/MIT]

% Definición de dimensiones
\setlength{\headheight}{64pt}  % Tamaño de la cabecera sin fancyhdr
\setcounter{MaxMatrixCols}{20} % Número máximo de columnas en matrices
\setlength{\footnotemargin}{3mm} % Margen del footnote
\renewcommand{\baselinestretch}{\defaultinterline} % Ajuste del entrelineado
\setcaptionmargincm{\captionlrmargin} % Margen por defecto

% Configuración de los colores
\color{\maintextcolor} % Color principal
\arrayrulecolor{red} % color de las lineas
\hypersetup{
	% Color de links
	colorlinks,
	citecolor=\citecolor,
	filecolor=black,
	linkcolor=\linkcolor,
	urlcolor=\urlcolor
}

% Configuración de las leyendas
\captionsetup{
	% Se actualizan los márgenes de los caption
	belowskip=\captionbottommargin pt,
	aboveskip=\captiontopmargin pt,
	labelfont={color=red,bf},
	textfont={color=green}
}
\ifthenelse{\equal{\figurecaptiontop}{true}}{
	% Se dejan los caption en la parte superior para las figuras
	\floatsetup[figure]{capposition=top}}{
}
\ifthenelse{\equal{\centeredcaption}{true}}{
	% Se centran todos los captions
	\captionsetup{justification=centering}}{
}

% Configuración de referencias
\bibliographystyle{\typereference} % Estilo de las referencias
\ifthenelse{\equal{\referenceassection}{true}}{
	% Se configura las referencias como una sección
	\patchcmd{\thebibliography}{*}{}{}{}
}{}
\makeatletter
\ifthenelse{\equal{\twocolumnreferences}{true}}{
	% Referencias en 2 columnas
	\renewenvironment{thebibliography}[1]
	{\begin{multicols}{2}[\section*{\refname}]
		\@mkboth{\MakeUppercase\refname}{\MakeUppercase\refname}
		\list{\@biblabel{\@arabic\c@enumiv}}
		{\settowidth\labelwidth{\@biblabel{#1}}
			\leftmargin\labelwidth
			\advance\leftmargin\labelsep
			\@openbib@code
			\usecounter{enumiv}
			\let\p@enumiv\@empty
			\renewcommand\theenumiv{\@arabic\c@enumiv}}
		\sloppy
		\clubpenalty 4000
		\@clubpenalty \clubpenalty
		\widowpenalty 4000
		\sfcode`\.\@m}
		{\def\@noitemerr
		{\@latex@warning{Ambiente `thebibliography' no definido}}
		\endlist\end{multicols}}}{}
\makeatother

% Configuración de acentos y carácteres especiales
\SelectInputMappings{
	aacute={á},
	Euro={€},
	Ntilde={Ñ}
}
\lstset{literate=
	{á}{{\'a}}1 {é}{{\'e}}1 {í}{{\'i}}1 {ó}{{\'o}}1 {ú}{{\'u}}1
	{Á}{{\'A}}1 {É}{{\'E}}1 {Í}{{\'I}}1 {Ó}{{\'O}}1 {Ú}{{\'U}}1
	{à}{{\`a}}1 {è}{{\`e}}1 {ì}{{\`i}}1 {ò}{{\`o}}1 {ù}{{\`u}}1
	{À}{{\`A}}1 {È}{{\'E}}1 {Ì}{{\`I}}1 {Ò}{{\`O}}1 {Ù}{{\`U}}1
	{ä}{{\"a}}1 {ë}{{\"e}}1 {ï}{{\"i}}1 {ö}{{\"o}}1 {ü}{{\"u}}1
	{Ä}{{\"A}}1 {Ë}{{\"E}}1 {Ï}{{\"I}}1 {Ö}{{\"O}}1 {Ü}{{\"U}}1
	{â}{{\^a}}1 {ê}{{\^e}}1 {î}{{\^i}}1 {ô}{{\^o}}1 {û}{{\^u}}1
	{Â}{{\^A}}1 {Ê}{{\^E}}1 {Î}{{\^I}}1 {Ô}{{\^O}}1 {Û}{{\^U}}1
	{œ}{{\oe}}1 {Œ}{{\OE}}1 {æ}{{\ae}}1 {Æ}{{\AE}}1 {ß}{{\ss}}1
	{ű}{{\H{u}}}1 {Ű}{{\H{U}}}1 {ő}{{\H{o}}}1 {Ő}{{\H{O}}}1
	{ç}{{\c c}}1 {Ç}{{\c C}}1 {ø}{{\o}}1 {å}{{\r a}}1 {Å}{{\r A}}1
	{€}{{\EUR}}1 {£}{{\pounds}}1
}

% Configuración de hbox y vbox
\hfuzz=100pt \vfuzz=100pt
\hbadness=2000 \vbadness=\maxdimen

% Se define metadata
\hypersetup{
	pdfpagemode={UseOutlines},
	pdftitle={\nombredelinforme},
	pdfauthor={\autordeldocumento},
	pdfsubject={\temaatratar},
	pdfcreator={LaTeX, pdfLaTeX, ppizarror},
	pdfproducer={Template-Informe v\templateversion\ | (Pablo Pizarro) ppizarror.com},
	pdfkeywords={\nombreuniversidad\ ,\codigodelcurso\ \nombredelcurso, \localizacionuniversidad},
	pdfinfo={
		Author={\autordeldocumento},
		Title={\nombredelinforme},
		Subject={\temaatratar},
		Template={Template-Informe},
		Template.Author={Pablo Pizarro, ppizarror.com},
		Template.Version={v\templateversion},
		Template.Website={http://ppizarror.com/Template-Informe/},
	},
	pdfview={FitH},
	pdfstartview={FitH},
	pdfstartpage={1},
	pdfdisplaydoctitle={true},
}

% INICIO DE LAS PÁGINAS
\begin{document}

% PORTADA
% Template:     Informe/Reporte LaTeX
% Documento:    Portada
% Versión:      3.6.6 (19/05/2017)
% Codificación: UTF-8
%
% Autor: Pablo Pizarro R.
%        Facultad de Ciencias Físicas y Matemáticas
%        Universidad de Chile
%        pablo.pizarro@ing.uchile.cl, ppizarror.com
%
% Manual template: [http://ppizarror.com/Template-Informe/]
% Licencia MIT:    [https://opensource.org/licenses/MIT/]

% Se escribe el header de la portada
\newpage
\renewcommand{\thepage}{\nameportraitpage}
\setpagemargincm{\pagemarginleft}{\firstpagemargintop}
{\pagemarginright}{\pagemarginbottom}
\pagestyle{fancy} \fancyhf{}
\fancyhead[L]{
	\nombreuniversidad \\ \nombrefacultad \\ \departamentouniversidad \\ \vspace{-0.43cm}}
\fancyhead[R]{
	\includegraphics[scale=\imagendepartamentoescala]{\imagendepartamento}}

% Estilo con código de curso
\ifthenelse{\equal{\gradecodeonportrait}{true}}{
	\vspace*{3cm}
	\begin{center}
		\textcolor {\portraittitlecolor}{
			\huge {\nombredelcurso} \\
			\vspace {0.3cm}
			\large {Código del curso: \codigodelcurso} \\
			\vspace {1.5cm}
			\Huge {\nombredelinforme} \\
			\vspace {0.3cm}
			\large {\temaatratar}
		}
	\end{center}

% Estilo sin código
}{
	\vspace*{5cm}
	\begin{center}
		\textcolor {\portraittitlecolor}{
			\huge {\nombredelcurso} \\
			\vspace {1cm}
			\Huge {\nombredelinforme} \\
			\vspace {0.3cm}
			\large {\temaatratar}
		}
	\end{center}
}

% Tabla de integrantes
\vfill
\tablaintegrantes

% CONFIGURACIÓN DE PÁGINA Y ENCABEZADOS
% Template:     Informe/Reporte LaTeX
% Documento:    Configuración de página
% Versión:      4.3.7 (12/08/2017)
% Codificación: UTF-8
%
% Autor: Pablo Pizarro R.
%        Facultad de Ciencias Físicas y Matemáticas
%        Universidad de Chile
%        pablo.pizarro@ing.uchile.cl, ppizarror.com
%
% Manual template: [http://latex.ppizarror.com/Template-Informe/]
% Licencia MIT:    [https://opensource.org/licenses/MIT/]

% Numeración de páginas
\newpage
\ifthenelse{\equal{\romanpageuppercase}{true}}{
	\pagenumbering{Roman}
}{
	\pagenumbering{roman}
}
\setcounter{page}{1}
\setcounter{footnote}{1}

% Márgenes de páginas y tablas
\setpagemargincm{\pagemarginleft}{\pagemargintop}{\pagemarginright}{\pagemarginbottom}
\def\arraystretch{\tablepadding} % Se ajusta el padding de las tablas

% Se define el punto decimal
\ifthenelse{\equal{\pointdecimal}{true}}{
	\decimalpoint}{
}

% Definición de nombres de objetos
\renewcommand{\appendixname}{\nomltappendixsection} % Nom. del anexo en etiq. de título
\renewcommand{\appendixpagename}{\nameappendixsection} % Nombre del anexo en índice
\renewcommand{\appendixtocname}{\nameappendixsection} % Nombre del anexo en índice
\renewcommand{\contentsname}{\nomltcont}  % Nombre del índice
\renewcommand{\figurename}{\nomltwfigure} % Nombre de la leyenda de las fig.
\renewcommand{\listfigurename}{\nomltfigure} % Nombre del índice de figuras
\renewcommand{\listtablename}{\nomlttable} % Nombre del índice de tablas
\renewcommand{\lstlistingname}{\nomltwsrc} % Nombre leyenda del código fuente
\renewcommand{\lstlistlistingname}{\nomltsrc} % Nombre índice código fuente
\renewcommand{\refname}{\namereferences} % Nombre de las referencias
\renewcommand{\tablename}{\nomltwtable} % Nombre de la leyenda de tablas

% Numeración de objetos
\ifthenelse{\equal{\showsectiononcaption}{true}}{
	\counterwithin{equation}{section}   % Añade número de sección a las ecuaciones
	\counterwithin{figure}{section}     % Añade número de sección a las figuras
	\counterwithin{lstlisting}{section} % Añade número de sección a los códigos
	\counterwithin{table}{section}      % Añade número de sección a las tablas
}{}

% Se crean los header-footer
\ifthenelse{\equal{\hfstyle}{style1}}{
	\pagestyle{fancy} \fancyhf{}
	\fancyhead[L]{\nouppercase{\rightmark}}
	\fancyhead[R]{\small \rm \thepage}
	\fancyfoot[L]{\small \rm \textit{\titulodelinforme}}
	\fancyfoot[R]{\small \rm \textit{\codigodelcurso \nombredelcurso}}
	\renewcommand{\footrulewidth}{0.5pt}
	\renewcommand{\headrulewidth}{0.5pt}
	\renewcommand{\sectionmark}[1]{\markboth{#1}{}}
}{
\ifthenelse{\equal{\hfstyle}{style2}}{
	\pagestyle{fancy} \fancyhf{}
	\fancyfoot[C]{\thepage}
	\renewcommand{\headrulewidth}{0pt}
	\renewcommand{\footrulewidth}{0pt}
	\setlength{\headheight}{49pt}
}{
\ifthenelse{\equal{\hfstyle}{style3}}{
	\pagestyle{fancy} \fancyhf{}
	\fancyfoot[L]{\departamentouniversidad}
	\fancyfoot[C]{\thepage}
	\fancyfoot[R]{\nombreuniversidad}
	\renewcommand{\headrulewidth}{0pt}
	\renewcommand{\footrulewidth}{0pt}
	\setlength{\headheight}{49pt}
}{
\ifthenelse{\equal{\hfstyle}{style4}}{
	\pagestyle{fancy} \fancyhf{}
	\fancyfoot[R]{\thepage}
	\renewcommand{\headrulewidth}{0pt}
	\renewcommand{\footrulewidth}{0pt}
	\setlength{\headheight}{49pt}
}{
\ifthenelse{\equal{\hfstyle}{style5}}{
	\pagestyle{fancy} \fancyhf{}
	\fancyhead[L]{\codigodelcurso \nombredelcurso}
	\fancyhead[R]{\nouppercase{\rightmark}}
	\fancyfoot[L]{\departamentouniversidad, \nombreuniversidad}
	\fancyfoot[R]{\small \rm \thepage}
	\renewcommand{\footrulewidth}{0pt}
	\renewcommand{\headrulewidth}{0pt}
	\renewcommand{\sectionmark}[1]{\markboth{#1}{}}
}{
\ifthenelse{\equal{\hfstyle}{style6}}{
	\pagestyle{fancy} \fancyhf{}
	\fancyhead[L]{\nouppercase{\rightmark}}
	\fancyhead[R]{}
	\fancyfoot[C]{\small \rm \thepage}
	\renewcommand{\footrulewidth}{0pt}
	\renewcommand{\headrulewidth}{0.5pt}
	\renewcommand{\sectionmark}[1]{\markboth{#1}{}}
}{
	\throwbadconfigondoc{Estilo de header-footer incorrecto}{\hfstyle}{style1,style2,style3,style4,style5,style6}}}}}}
}

% Profundidad del índice
\setcounter{tocdepth}{\indexdepth} % Se ajusta la profundidad del índice


% RESUMEN O ABSTRACT
\begin{resumen}
	\lipsum[1] % Se incluye un párrafo de resumen, se puede borrar
\end{resumen}

% TABLA DE CONTENIDOS - ÍNDICE
% Template:     Informe/Reporte LaTeX
% Documento:    Índice
% Versión:      4.4.9 (27/08/2017)
% Codificación: UTF-8
%
% Autor: Pablo Pizarro R.
%        Facultad de Ciencias Físicas y Matemáticas
%        Universidad de Chile
%        pablo.pizarro@ing.uchile.cl, ppizarror.com
%
% Manual template: [http://latex.ppizarror.com/Template-Informe/]
% Licencia MIT:    [https://opensource.org/licenses/MIT/]

\ifthenelse{\equal{\showindex}{true}}{
	
	% Crea nueva página y establece estilo de títulos
	\newpage
	\ifthenelse{\equal{\addindextobookmarks}{true}}{
		\belowpdfbookmark{\nomltcont}{contents}}{
	}
	\tocloftpagestyle{fancy}
	\sectionfont{\color{\indextitlecolor} \fontsizetitlei \styletitlei \selectfont}
	
	% Configuración del punto en índice
	\ifthenelse{\equal{\showdotontitles}{true}}{
		\def\cftsecaftersnum {.}
		\def\cftsubsecaftersnum {.}
		\def\cftsubsubsecaftersnum {.}
		}{
	}

	% Configuración del punto en número de objetos
	\ifthenelse{\equal{\showdotonobjectindex}{true}}{
		\def\cftfigaftersnum {.}
		\def\cfttabaftersnum {.}
		\def\cftsubfigaftersnum {.}
		\def\cftlstlistingaftersnum {.}
		}{
	}

	% Configuración punto entre objeto y número de página
	\ifthenelse{\equal{\showdotpagenumindex}{true}}{}{
		\renewcommand{\cftdot}{}
	}
	
	% Índice de Contenidos
	\ifthenelse{\equal{\showindexofcontents}{true}}{\tableofcontents}{}
	
	% Lista de Figuras
	\iftotalfigures
		\ifthenelse{\equal{\showindexoffigures}{true}}{\listoffigures}{}
	\fi
	
	% Lista de Tablas
	\iftotaltables
		\ifthenelse{\equal{\showindexoftables}{true}}{\listoftables}{}
	\fi
	
	% Lista del Código Fuente
	\iftotallstlistings
		\ifthenelse{\equal{\showindexofcode}{true}}{\lstlistoflistings}{}
	\fi
	
	% Se añade una página en blanco
	\ifthenelse{\equal{\addemptypagetwosides}{true}}{
		\vfill
		\checkoddpage
		\ifoddpage
		\else
			\newpage
			\null
			\thispagestyle{empty}
			\newpage
			\addtocounter{page}{-1}
		\fi
	}{}

}{}
 % Índice, se puede borrar

% CONFIGURACIONES FINALES
% Template:     Informe/Reporte LaTeX
% Documento:    Archivo principal
% Versión:      4.3.0 (13/07/2017)
% Codificación: UTF-8
%
% Autor: Pablo Pizarro R.
%        Facultad de Ciencias Físicas y Matemáticas
%        Universidad de Chile
%        pablo.pizarro@ing.uchile.cl, ppizarror.com
%
% Manual template: [http://ppizarror.com/Template-Informe/]
% Licencia MIT:    [https://opensource.org/licenses/MIT/]

% Se reestablecen headers y footers
\markboth{}{}
\newpage
\ifthenelse{\equal{\showheadertitle}{true}}{
	\fancyhead[L]{\nouppercase{\leftmark}}}{
}

% Se reestablece estilo de títulos
\sectionfont{\color{\titlecolor} \fontsizetitle \styletitle \selectfont}
\subsectionfont{\color{\subtitlecolor} \fontsizesubtitle \stylesubtitle \selectfont}
\subsubsectionfont{\color{\subsubtitlecolor} \fontsizesubsubtitle \stylesubsubtitle \selectfont}

% Se reestablecen números de página y secciones
\renewcommand{\thepage}{\arabic{page}}
\setcounter{page}{1}
\setcounter{section}{0}
\setcounter{footnote}{0}


% ========================= INICIO DEL DOCUMENTO =========================

% Template:     Informe/Reporte LaTeX
% Advertencia:  Documento generado automáticamente a partir del archivo
%               lib/example.tex
% Versión:      4.0.0 (10/06/2017)
% Codificación: UTF-8
%
% Autor: Pablo Pizarro R.
%        Facultad de Ciencias Físicas y Matemáticas
%        Universidad de Chile
%        pablo.pizarro@ing.uchile.cl, ppizarror.com
%
% Manual template: [http://ppizarror.com/Template-Informe/]
% Licencia MIT:    [https://opensource.org/licenses/MIT/]

% NUEVA SECCIÓN
% Las secciones se inician con \section, si se quiere una sección sin "número" se pueden usar las funciones \sectionanum (sección sin número) o la función \sectionanumnoi para crear el mismo título sin numerar y sin aparecer en el índice
\section{Informes con \LaTeX}
	
	% SUB-SECCIÓN
	% Las sub-secciones se inician con \subsection, si se quiere una sub-sección sin "número" se pueden usar las funciones \subsectionanum (nuevo subtítulo sin numeración) o la función \subsectionanumnoi para crear el mismo subtítulo sin numerar y sin aparecer en el índice
	\subsection{Una breve introducción}
		
		Este es un párrafo, puede contener múltiples \quotes{Expresiones} así como fórmulas o referencias \footnote{Las referencias se hacen utilizando la expresión \texttt{\textbackslash label}\{etiqueta\}} a fórmulas como \eqref{eqn:identidad-imposible}. A continuación se muestra un ejemplo de inserción de imágenes o figuras (como la Figura \ref{img:testimage}) con el comando \texttt{\textbackslash insertimage}:
		
		% Para insertar una imagen se puede usar la función \insertimage la cual toma un primer parámetro opcional para definir una etiqueta (dentro de los corchetes), luego toma la dirección de la imagen, sus parámetros (en este caso se definió la escala de 0.15) y una leyenda
		\insertimage[\label{img:testimage}]{ejemplos/test-image.png}{scale=0.15}{Where are you? de \quotes{Internet}.}
		
		A continuación \footnote{Como puedes observar las funciones \texttt{\textbackslash insert...} agregan un párrafo automáticamente.} se muestra un ejemplo de inserción de ecuaciones simples con el comando \texttt{\textbackslash insertequation}:
		
		% Se inserta una ecuación, el primer parámetro entre corchetes es opcional (permite identificar con una etiqueta para poder referenciarlo después con \ref), seguido de aquello se escribe la ecuación en modo bruto sin signos peso
		\insertequation[\label{eqn:identidad-imposible}]{\pow{a}{k}=\pow{b}{k}+\pow{c}{k} \quad \forall k>2}
		
		% Se añade parrafo de prueba, notar que no se requiere añadir un salto de línea después de insertar una función
		Nunc sed pede. Praesent vitae lectus. Praesent neque justo, vehicula eget, interdum id, facilisis et, nibh. Phasellus at purus et libero lacinia dictum. Fusce aliquet. Nulla eu ante placerat leo semper dictum. Mauris metus. Curabitur lobortis. Curabitur sollicitudin hendrerit nunc.
		
		% Los párrafos se pueden añadir con \newp, esta función se hizo para evitar errores y warnings por parte del compilador
		\newp Este es un nuevo párrafo insertado con el comando \texttt{\textbackslash newp}. Si no te gustan los comandos \texttt{\textbackslash newp}, \texttt{\textbackslash newpar} o \texttt{\textbackslash newparnl} simplemente puedes usar los salto de línea convencionales acompañado de \texttt{\textbackslash par}. Además puedes editar las funciones, definidas en el archivo \texttt{lib/functions.tex}.
		
	% SUB-SECCIÓN
	\subsection{Añadiendo tablas}
		
		\newp También puedes usar tablas, insertarlas es muy fácil, puedes usar el plugin \href{https://www.ctan.org/tex-archive/support/excel2latex/}{Excel2Latex} de Excel para convertir las tablas a \LaTeX\xspace o bien utilizar el \quotes{creador de tablas online} \textsuperscript{\cite{ref3}}.
		
		% Tabla generada con Excel2Latex
		\begin{table}[htbp]
			\centering
			\caption{Ejemplo de tablas.}
			\begin{tabular}{ccc}
				\hline
				\textbf{Columna 1} & \textbf{Columna 2} & \textbf{Columna 3} \bigstrut\\
				\hline
				$\omega$ & $\nu$ & $\delta$ \bigstrut[t]\\
				$\beta$ & $\gamma$ & $\epsilon$ \\
				$\varepsilon$ & $\upsilon$ & $\varphi$\\
				$\Phi$ & $\Theta$ & $\varSigma$ \bigstrut[b]\\
				\hline
			\end{tabular}
			\label{tab:tabla-1}
		\end{table}

% NUEVA SECCIÓN
\newpage
\section{Aquí un nuevo tema}
	
	% SUB-SECCIÓN
	\subsection{Haciendo informes como un profesional}
		
		% Se inserta una imagen flotante en la izquierda del documento con \insertimageleft, al igual que las demás funciones, el primer parámetro es opcional, luego viene la ubicación de la imagen, seguido de la escala y por último su leyenda. Para insertar una imagen flotante en la derecha se utiliza \insertimageright usando los mismos parámetros
		\insertimageleft[\label{img:imagen-izquierda}]{ejemplos/test-image-wrap}{0.3}{Apolo flotando a la izquierda.}
		
		\lipsum[1]

		% Párrafos con \newp, lipsum por defecto no añade un párrafo nuevo
		\newp \lipsum[115]
		\newp \lipsum[2]
		
		% Agrega una ecuación con leyendas
		\insertequationcaptioned[\label{eqn:formulasinsentido}]{\int_{a}^{b} f(x) \dd{x} = \fracnpartial{f(x)}{x}{\eta} \cdotp \textstyle \sum_{x=a}^{b} f(x)\cancelto{1+\frac{\epsilon}{k}}{(1+\Delta x)}}{Ecuación sin sentido.}
		
		% Aquí no es necesario usar \newp dado que todas las funciones \insert... añaden un párrafo nuevo por defecto
		\lipsum[115]
		
		% Párrafos con \newp, lipsum por defecto no añade un párrafo nuevo
		\newp \lipsum[4]
		
	% Inserta un subtítulo sin número
	\subsection{Otros párrafos más normales}
	
		% Párrafos con lipsum
		\lipsum[7]
		
		\newp \lipsum[2]
		
		% Se inserta una ecuación larga con el entorno gathered (1 solo número de ecuación)
		\insertgathered[\label{eqn:eqn-larga}]{
			\lpow{\Lambda}{f} = \frac{L\cdot f}{W} \cdot \frac{\pow{\lpow{Q}{e}}{2}}{8 \pow{\pi}{2} \pow{W}{4} g} + \sum_{i=1}^{l} \frac{f \cdot \big( M - d\big)}{l \cdot W} \cdot \frac{\pow{\big(\lpow{Q}{e}- i\cdot Q\big)}{2}}{8 \pow{\pi}{2} \pow{W}{4} g}\\
			Q_e = Q \cdot \int_{0}^{e} V(x) \dd{x}
		}
	
		% Nuevo párrafo
		\lipsum[4]
		
		% Se inserta un multicols, con esto se pueden escribir en varias columnas
		\begin{multicols}{2}
			
			% Párrafo 1
			\lipsum[4]
			
			% Ecuación
			\insertequation[]{ f(x) = \fracdpartial{u}{t} }
			
			% Párrafo 2 del multicols
			\lipsum[1]
			
		\end{multicols}
		
	% SUB-SECCIÓN
	\subsection{Ejemplos de inserción de código fuente}
		
		A continuación se presenta un ejemplo de inserción de código fuente en Python (Código \ref{codigo-python}) \footnote{El mejor lenguaje del mundo}, Java (Código \ref{codigo-java}) y Matlab (Código \ref{codigo-matlab}) utilizando el entorno \texttt{lstlisting}: \\
		
% Se define el lenguaje del código, cuidado: Los códigos en LaTeX son sensibles a las tabulaciones y espacios en blanco
\begin{lstlisting}[style=Python, caption={Ejemplo en Python.\label{codigo-python}}]
import numpy as np

def incmatrix(genl1, genl2):
	m = len(genl1)
	n = len(genl2)
	M = None # to become the incidence matrix
	VT = np.zeros((n*m, 1), int) # dummy variable
\end{lstlisting}

\begin{lstlisting}[style=Java, caption={Ejemplo en Java.\label{codigo-java}}]
import java.io.IOException; 
import javax.servlet.*;

// Hola mundo
public class Hola extends GenericServlet {
	public void service(ServletRequest request, ServletResponse response)
	throws ServletException, IOException{
		response.setContentType("text/html");
		PrintWriter pw = response.getWriter();
		pw.println("Hola, mundo!");
		pw.close();
	}
}
\end{lstlisting}

\begin{lstlisting}[style=Matlab, caption={Ejemplo en Matlab.\label{codigo-matlab}}]
% Se crea gráfico
f = figure(1); hold on; movegui(f, 'center');
xlabel('td/Tn'); ylabel('FAD=Umax/Uf0');
title('Espectro de pulso de desplazamiento');

for j = 1:length(BETA)
	fad = ones(1, NDATOS); % Arreglo para el FAD, uno para cada r (o td/Tn)
	
	% Se crea el espectro de respuesta máximo para cada par de beta/r
	for i = 1:NDATOS
		[t, u_t, ~, ~] = main(BETA(j), r(i), M, K, F0, 0);
		fad(i) = max(abs(u_t)) / uf0;
	end
	mx = find(fad == max(fad(:)));
	fprintf('BETA=%.2f, MAX: FAD=%.3f, TD/TN=%.3f\n', BETA(j), fad(mx), tdtn(mx));
	plot(tdtn, fad, 'DisplayName', strcat('\beta=', sprintf('%.2f', BETA(j))));
end
\end{lstlisting}


% NUEVA SECCIÓN
% Inserta una sección sin número
\section{Más ejemplos}
	
	% Inserta un subtítulo sin número
	\subsection{Listas y Enumeraciones}
		
		Hacer listas enumeradas con \LaTeX\ es muy fácil \footnote{También puedes revisar el manual de las enumeraciones en \url{http://www.texnia.com/archive/enumitem.pdf}}, para eso debes usar el comando \texttt{\textbackslash begin\{enumerate\}}, cada elemento empieza por \texttt{\textbackslash item}, resultando:
		
		\begin{enumerate}
			\item Ítem 1
			\item Abracadabra
			\item Manzanas
		\end{enumerate}
		
		También se puede cambiar el tipo de enumeración, se pueden usar letras, números romanos, entre otros. Esto se logra cambiando el \textbf{label} del objeto \texttt{enumerate}. A continuación se muestra un ejemplo usando letras con el estilo \texttt{\textbackslash alph} \footnote{Con \texttt{\textbackslash Alph} las letras aparecen en mayúscula}, números romanos con \texttt{\textbackslash roman} \footnote{Con \texttt{\textbackslash Roman} los números romanos salen en mayúscula} o números griegos con \texttt{\textbackslash greek} \footnote{Una característica propia del template, con \texttt{\textbackslash Greek} las letras griegas están escritas en mayúscula}:
		
		\begin{multicols}{3}
			\begin{enumerate}[label=\alph*) ,font=\bfseries] % Fuente en negrita
				\item Peras
				\item Manzanas
				\item Naranjas
			\end{enumerate}
			
			\begin{enumerate}[label=\greek*) ]
				\item Matemáticas
				\item Lenguaje
				\item Filosofía
			\end{enumerate}
		
			\begin{enumerate}[label=\roman*) ]
				\item Rojo
				\item Café
				\item Morado
			\end{enumerate}
		\end{multicols}
		
		Para hacer listas sin numerar con \LaTeX\ hay que usar el comando \texttt{\textbackslash begin\{itemize\}}, cada elemento empieza por \texttt{\textbackslash item}, resultando:
		
		\begin{multicols}{3}
			\begin{itemize}[label={--}]
				\item Peras
				\item Manzanas
				\item Naranjas
			\end{itemize}
			
			\begin{enumerate}[label={*}]
				\item Rojo
				\item Café
				\item Morado
			\end{enumerate}
			
			\begin{itemize}
				\item Árboles
				\item Pasto
				\item Flores
			\end{itemize}
		\end{multicols}
		
	% Inserta un subtítulo sin número
	\subsection{Otros}
		
		Recuerda revisar el manual de todas las funciones de este template visitando el siguiente link: \url{http://ppizarror.com/Template-Informe/}. Además si necesitas una ayuda muy específica sobre el template me puedes enviar un correo a \insertemail{pablo.pizarro@ing.uchile.cl}.
		
\newpage % Salto de página % Ejemplo, se puede borrar


% REFERENCIAS
\begin{references}
	\bibitem{ref1}
	Template Informe en \LaTeX.
	\textit{¡Revisa el manual online de este template!} \\
	\url{http://ppizarror.com/Template-Informe/}
	
	\bibitem{ref2}
	Excel2Latex.
	\textit{Importa de forma sencilla tus tablas de Excel a \LaTeX.} \\
	\url{https://www.ctan.org/tex-archive/support/excel2latex/}
	
	\bibitem{ref3}
	ShareLatex.
	\textit{Uno de los mejores editores online para \LaTeX\.} \\
	\href{https://www.sharelatex.com/?r=298b935f&rm=d&rs=b}{\texttt{http://www.tablesgenerator.com/}}
\end{references}

% FIN DEL DOCUMENTO
\end{document}